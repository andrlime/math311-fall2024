% packages
\documentclass[b5paper]{article}
\usepackage[utf8]{inputenc}
\usepackage[margin=1.1in]{geometry} % 0.9 for pset, 1.4 for livetex notes

% Math libraries
\usepackage{amsmath}
\usepackage{amssymb}
\usepackage{amsthm}
\usepackage{amsthm}

% Formatting
\usepackage{titling}
\usepackage{fancyhdr}
\usepackage{graphicx}
\usepackage{float}
\usepackage{titlesec}
\usepackage{setspace}
\usepackage[titles]{tocloft}

% Additional commands
\usepackage{cancel}
\usepackage{enumitem}
\usepackage{xfrac}
\usepackage{mathtools}

% Graphcs
\usepackage{tikz}
\usepackage{tcolorbox}
\usepackage{xcolor}
\usepackage{booktabs}
\usepackage{longtable}

% Content
\usepackage{physics}
\usepackage[version=4]{mhchem}
\usepackage{siunitx}
\usepackage{sectsty}
\usepackage{hyperref}

\hypersetup{
    colorlinks=true,
    linktoc=all,
    linkcolor=black,
}

\allsectionsfont{\sffamily}
\renewcommand{\cftsecfont}{\normalfont\bfseries\sffamily}
\renewcommand*{\contentsname}{Table of Contents}

\newcommand*\circled[1]{\tikz[baseline=(char.base)]{
            \node[shape=circle,draw,inner sep=2pt] (char) {#1};}}
\renewcommand\real{\mathbb{R}}
\newcommand\complex{\mathbb{C}}
\newcommand\integer{\mathbb{Z}}
\newcommand\rational{\mathbb{Q}}
\renewcommand\natural{\mathbb{N}}
\newcommand\probability{\mathbb{P}}
\newcommand\field{\mathbb{F}}
\newcommand\qsp{\qq{}}
\newcommand\curlyf{\mathcal{F}}
\newcommand\psp{(\Omega, \curlyf, \probability)}
\DeclarePairedDelimiter{\ceil}{\lceil}{\rceil}

\definecolor{DARKBLUE}{HTML}{005E7A}
\definecolor{BLACK}{HTML}{121212}
\definecolor{DARKRED}{HTML}{83223C}
\definecolor{ORANGE}{HTML}{F0B12C}
\definecolor{CYAN}{HTML}{2CF0CB}
\definecolor{RED}{HTML}{C41E3A}

\newcounter{lemmacounter}
\newcounter{definitioncounter}
\newtcolorbox{squarebox}{
  sharp corners,
  colback=yellow!5!white,
  colframe=DARKBLUE!75!BLACK,
  boxrule=0.5px
}

\newtcolorbox{squarebox2}{
  sharp corners,
  colback=yellow!5!white,
  colframe=ORANGE!75!BLACK,
  boxrule=0.5px
}

\newtcolorbox{roundbox1}{
  colback=yellow!5!white,
  colframe=BLACK!75!BLACK,
  boxrule=0.5px
}

\newtcolorbox{roundbox2}{
  colback=yellow!5!white,
  colframe=DARKRED!75!BLACK,
  boxrule=0.5px
}

\newtcolorbox{roundbox3}{
  colback=yellow!5!white,
  colframe=CYAN!75!BLACK,
  boxrule=0.5px
}

\newtcolorbox{roundbox4}{
  sharp corners,
  colback=RED!5!white,
  colframe=RED!75!BLACK,
  boxrule=0.5px
}

\newenvironment{problem}
  {\begin{roundbox1}}
  {\end{roundbox1}}
  
\newenvironment{solution}
  {\begin{proof}[Solution]}
  {\end{proof}}

\newenvironment{definition}
  {\begin{roundbox1}
  \stepcounter{lemmacounter}
  \textcolor{BLACK}{\textsf{\textbf{Definition \thesection.\arabic{lemmacounter}}}}}
  {\end{roundbox1}}

\newenvironment{example}
  {\begin{roundbox2}
  \stepcounter{lemmacounter}
  \textcolor{DARKRED}{\textsf{\textbf{Example \thesection.\arabic{lemmacounter}}}}}
  {\end{roundbox2}}

\newenvironment{lemma}
  {\begin{squarebox2}
  \stepcounter{lemmacounter}
  \textcolor{ORANGE}{\textsf{\textbf{Lemma \thesection.\arabic{lemmacounter}}}}}
  {\end{squarebox2}}

\newenvironment{aside}
  {\begin{roundbox4}
  \textcolor{RED}{\textsf{\textbf{Note}}}}
  {\end{roundbox4}}
  
\newenvironment{theorem}
  {\begin{squarebox}
  \stepcounter{lemmacounter}
  \textcolor{DARKBLUE}{\textsf{\textbf{Theorem \thesection.\arabic{lemmacounter}}}}}
  {\end{squarebox}}
  
\newenvironment{proposition}
  {\begin{roundbox3}
  \stepcounter{lemmacounter}
  \textcolor{DARKBLUE}{\textsf{\textbf{Proposition \thesection.\arabic{lemmacounter}}}}}
  {\end{roundbox3}}

\makeatletter
\DeclareRobustCommand{\@seccntformat}[1]{%
  \def\temp@@a{#1}%
  \def\temp@@b{subsubsection}%
  \ifx\temp@@a\temp@@b
  \textcolor{DARKRED}{\csname the#1\endcsname}%
  \quad
  \else
  \textcolor{DARKBLUE}{§\csname the#1\endcsname}%
  \quad
  \fi
} 
\makeatother

\numberwithin{equation}{section}
\numberwithin{lemmacounter}{section}
\numberwithin{definitioncounter}{section}

\begin{document}
    \newcommand{\courseshort}{\textbf{Math 311-1}}
\newcommand{\coursetitle}{MENU Probability and Stochastic Processes}
\newcommand{\term}{Fall Quarter 2024}
\newcommand{\prof}{Benjamin Weinkove}
\newcommand{\textbook}{Basic Probability Theory by Robert Ash}
\newcommand{\isbn}{978-0-4886-46628-6}

    \pagestyle{fancy}
    \fancyfoot[C]{}
    \fancyhead[L]{\textbf{Andrew Li} (\term)}
    \fancyfoot[L]{\courseshort \: \coursetitle}
    \fancyfoot[R]{\thepage}

    \title{\textbf{\textsf{\courseshort \, Lecture Notes}}}
    \author{Andrew Li}
    \date{\term}
    \clearpage\maketitle
    \noindent Original lecture notes for \textbf{\courseshort: \coursetitle}, from \term, taught by Professor \prof. This course follows \textbook, ISBN \isbn.
    \thispagestyle{empty}

    \tableofcontents
    \AddToHook{cmd/section/before}{\clearpage}
    
    \section{September 24, 2024}
\subsection{Introduction}
Introducing a probability course first requires a rigourous definition of a probability space, and some brief review of set theory.

\begin{proposition}
    Under the classical definition of probability, the probability of some event is defined as
    \begin{align}
        \mathbb{P}(\mathrm{event}) = \dfrac{\#\,\mathrm{favourable\,outcomes}}{\#\,\mathrm{total\,outcomes}}
    \end{align}
    For example, for rolling a (fair) six-sided dice, the probability of each of the six sides landing up is 
    \begin{align}
        \mathbb{P}(\{1\}) = \mathbb{P}(\{2\}) = ... = \mathbb{P}(\{6\}) = 1/6
    \end{align}
    For flipping two coins, the notation more clearly implicates why events are defined as sets as opposed to distinct elements.
    \begin{align}
        \mathbb{P}(\{ HH, HT, TH \}) = 3/4
    \end{align}
    or phrased in English, the probability of flipping at least one head after flipping two (fair) coins is $3/4$. Note here that
    \begin{align}
        \{ HH, HT, TH \}^C = \{ \mathrm{TT} \} 
    \end{align}
    which obviously has a $\left( \frac{1}{2} \right)^2 = \frac{1}{4}$ probability.
\end{proposition}

\subsection{Probability Spaces}
\begin{definition}
    A \textbf{sample space} $\Omega$ is defined as the set of possible outcomes of some random experiment.
\end{definition}
\begin{definition}
    An \textbf{event space} $\mathcal{F}$ is defined as some set of events, which are subsets of some sample space $\Omega$. That is, an event is some set of outcomes, such as $\{ 2, 4, 6 \}$ being the even sides of a dice.
\end{definition}
\begin{definition}
    A \textbf{probability space} is a triplet
    \begin{align}
        \left( \Omega, \mathcal{F}, \mathbb{P} \right)
    \end{align}
    consisting of $\Omega$ a sample space (a set), $\mathcal{F}$ an event space (a $\sigma$-algebra of subsets/events), and $\mathbb{P}: \mathcal{F} \to [0,1]$ (a probability measure on $\mathcal{F}$).
\end{definition}

\subsubsection{Sigma Algebra}
\begin{definition}
    A collection of subsets ($\mathcal{F}$) of some set ($\Omega$) is a \textbf{$\sigma$-algebra} if
    \begin{align}
        \begin{cases}
            \Omega \in \mathcal{F}\\
            A_1, ..., A_\infty \in \mathcal{F} \implies \bigcup_i^\infty A_i \in \mathcal{F}\\
            A \in \mathcal{F} \implies A^C \in \mathcal{F}
        \end{cases}
    \end{align}
\end{definition}
It follows trivially that
\begin{lemma}
    A $\sigma$-algebra always contains $\emptyset$
\end{lemma}
\begin{proof}
    Suppose some $\sigma$-algebra $\mathcal{F}$ does not contain the empty set. By definition, $\Omega \in \mathcal{F}$, and by definition, $\Omega^C \in \mathcal{F}$. However, $\Omega^C = \emptyset$, which is a contradiction.
\end{proof}
It follows slightly less trivially that
\begin{example}
    It is not necessarily true that $\mathcal{F}$ contains \textbf{all} subsets of $\Omega$. As a trivial example, let $\Omega = \{ 1, 2, 3, 4, 5, 6 \}$. Then,
    \begin{align}
        \mathcal{F} = \{ \emptyset, \{ 1 \}, \{ 2, 3, ..., 6 \}, \{ 1, 2, ..., 6 \} \}
    \end{align}
    is clearly a $\sigma$-algebra, and is easy to see per Definition 1.5.
\end{example}

\subsubsection{Probability Measure}
\begin{definition}
    A \textbf{probability measure} is some function
    \begin{align}
        \mathbb{P}: \mathcal{F} \to [0,1]
    \end{align}
    such that
    \begin{align}
        \mathbb{P}(\Omega) = 1
    \end{align}
    where $\Omega$ is a sample space, i.e. all possible outcomes. 
\end{definition}
The probability of an event (a set) corresponds to the sum of all outcomes within that event. Suppose $\Omega = \{ 1, 2, 3, ..., N \}$; then
\begin{align}
    \mathbb{P}(\Omega) = \mathbb{P}(1) + \mathbb{P}(2) + ... + \mathbb{P}(N)
\end{align}
\begin{proposition}
    In general, let $A_1, A_2, ...$ be disjoint subsets of $\Omega$; then
    \begin{align}
        \mathbb{P}\left(\bigcup_i^\infty A_i\right) &= \sum_i^\infty \mathbb{P}(A_i)
    \end{align}
\end{proposition}
\begin{definition}
    Two sets $A_i$ and $A_j$ are disjoint if
    \begin{align}
        A_i \cap A_j = \emptyset \impliedby i \ne j
    \end{align}
\end{definition}
A simple example for this is rolling a six-sided (fair) dice. In this case, $\mathcal{F}$ is the set of all $2^6 = 64$ subsets of $\Omega$. We can easily see that
\begin{align}
    \mathbb{P}(\{1\}) &= 1/6\\
    \mathbb{P}(\{2\}) &= 1/6\\
    \mathbb{P}(\{1, 2\}) &= 1/6 + 1/6 = 2/6
\end{align}
Note that the probabilities are not derived based on anything (though we could use physics); we use probability as a model for the world based on how we define the probabilities of certain events.

\subsection{Digression on Set Theory}
Suppose $A, B, C$ are sets. The operations $\cup$, $\cap$, and $^C$ are closed and have the following properties:
\begin{enumerate}
    \item Commutativity
    \begin{align}
        A \cap B &= B \cap A\\
        A \cup B &= B \cup A \notag
    \end{align}
    \item Associativity
    \begin{align}
        A \cap (B\cap C) &= (A \cap B) \cap C\\
        A \cup (B\cup C) &= (A \cup B) \cup C \notag
    \end{align}
    \item Distributivity
    \begin{align}
        A \cup (B \cap C) &= (A \cup B) \cap (B \cup C)\\
        A \cap (B \cup C) &= (A \cap B) \cup (B \cap C) \notag
    \end{align}
\end{enumerate}

\subsubsection{De Morgan's Laws}
Suppose $A_1, A_2, ...$ are sets. Then,
\begin{lemma}
    \begin{align}
        \left( \bigcap_n^\infty A_n \right)^C = \bigcup_n^\infty A_n^C
    \end{align}
\end{lemma}
\begin{proof}
    For some element $x$,
    \begin{align}
        x \in \left( \bigcap_n^\infty A_n \right)^C &\iff \exists n \mid x \notin A_n\\
        &\iff \exists n \mid x \in A_n^C \\
        &\iff x \in \bigcup_n^\infty A_n^C
    \end{align}
    (1.20) follows because if $x$ is in the complement of the intersection of all of the sets, that necessarily means it must not be in that intersection, i.e. not be in at least one set. (1.21) follows trivially: given the previous statement, $x$ must be in the complement of one of the sets. So, (1.22) follows because $x$ is in at least one of the complements which is a subset of the union of all of them.
\end{proof}
\begin{lemma}
    \begin{align}
        \left( \bigcup_n^\infty A_n \right)^C = \bigcap_n^\infty A_n^C
    \end{align}
\end{lemma}
\begin{proof}
    \begin{align}
        x \in \left( \bigcup_n^\infty A_n \right)^C &\iff x \notin A_n \mid \forall n\\
        &\iff x \in A_n^C \mid \forall n \\
        &\iff x \in \bigcap_n^\infty A_n^C
    \end{align}
    (1.24) follows because for $x$ to not be in the union of all of these sets, then $x$ cannot be an element of any of them, which implies (1.25) because that means $x$ must simultaneously be an element of the complement of all of the sets. For that to be true requires $x$ to be an element of the intersection of all $A_n^C$.
\end{proof}

% \subsection{Exercises (1.2)}
% \begin{enumerate}
%     \item[1.] An experiment involves choosing an integer N between 0 and 9 (the sample space consists of the integers from 0 to 9, inclusive). Let $A = \{ N \le 5 \}$, $B = \{ 3 \le N \le 7 \}$, $C = \{ N \text{ is even and } N > 0 \} $. List the points that belong to the following events.
%     \begin{align}
%         A \cap B \cap C \qsp A \cup (B \cap C^C) \qsp (A \cup B) \cap C^C \qsp (A \cap B) \cap [(A \cup C)^C]
%     \end{align}
%     \begin{solution}
%         It's trivial.
%     \end{solution}
%     \item[5.] Let $\Omega$ be the reals. Prove the following:
%     \begin{itemize}
%         \item[i.] Open intervals \begin{align}
%             (a,b) = \underbracket{\bigcup_{n=1}^\infty \left( a, b - \frac{1}{n} \right]}_{\circled{A}} = \bigcup_{n=1}^\infty \left[ a + \frac{1}{n}, b \right)
%         \end{align}
%         \begin{solution}
%             The open interval $(a,b) = \{ N \mid a < N < b \}$. Claim that $x \in (a,b) \iff x \in \circled{A}$. For some element $x \in (a,b)$ to be in $\bigcup \cdots$, there must exist some $n'$ such that $b - \frac{1}{n'} > x$. Then, because $x \in \left(a, b - \frac{1}{n'}\right]$, it must be true that $x$ is also in the union of all of these intervals. Set
%             \begin{align}
%                 n' \equiv \left\lceil\frac{1}{b - x}\right\rceil
%             \end{align}
%             Then, for $n = \left\lceil\frac{1}{b - x}\right\rceil$, due to taking the ceiling in (1.30), the inverse of $\lceil1/(b-x)\rceil$ is something less than $b - x$, i.e. the interval becomes, for some small $\varepsilon > 0$,
%             \begin{align}
%                 &\left(a, b - (b - x) + \varepsilon \right] \to \left( a, x + \varepsilon \right] \supset (a,x)
%             \end{align}
%             This is true for all $x$, i.e. for all elements $x \in (a,b)$, there exists some interval being unioned that contains that element. Conversely, it is trivial to show that all elements $x$ in at least one of the intervals being unioned necessarily lie in the $(a,b)$ interval.
%         \end{solution}
%         \item[ii.] Closed intervals \begin{align}
%             [a,b] = \bigcap_{n=1}^\infty \left[ a, b + \frac{1}{n} \right) = \bigcap_{n=1}^\infty \left( a - \frac{1}{n}, b \right]
%         \end{align}
%         \begin{solution}
%             Same chain of reasoning as above but for a slightly different scenario.
%         \end{solution}
%     \end{itemize}
%     \item[9.] If $A, B_1, B_2, \ldots$ are arbitrary events, show that
%     \begin{align}
%         A \cap \left( \bigcup_i^N B_i \right) = \bigcup_i^N \left( A \cap B_i \right)
%     \end{align}
%     holds as $N \to \infty$.
%     \begin{solution}
%         We can prove by induction. The base case is for $N=2$, i.e.
%         \begin{align}
%             A \cap (B_1 \cup B_2) = (A \cap B_1) \cup (A \cap B_2)
%         \end{align}
%         This is obvious. For some element $x$ to be in $A \cap (B_1 \cup B_2)$ means $x \in A$ and $x$ is in at least one of the $B_{\ldots}$ sets. That implies $x \in (A \cap B_1)$ or $x \in (A \cap B_2)$ or both, which is equivalent to the right side. We claim that (1.34) is true for some $N$. Then, to show it holds for $N+1$,
%         \begin{align}
%             A \cap \left( \bigcup_i^{N+1} B_i \right) &= A \cap \left( \left[\bigcup_i^{N} B_i\right] \cup B_{N+1}\right) = \left( A \cap \left[\bigcup_i^{N} B_i\right] \right) \cup \left( A \cap B_{N+1} \right)\\
%             &= \bigcup_i^N \left( A \cap B_i \right) \cup \left( A \cap B_{N+1} \right) = \bigcup_i^{N+1} \left( A \cap B_i \right)
%         \end{align}
%     \end{solution}
% \end{enumerate}

    \section{September 25, 2024}
Recall that a probability space is defined as a triplet $(\Omega, \mathcal{F}, \mathbb{P})$. Further recall the three conditions that define a $\sigma$-algebra. Finally, recall the definition of a probability measure. Using these, we can define some properties.

\subsection{Probability Space Properties}

\subsubsection{Properties of a Sigma Algebra}
Let $\mathcal{F}$ be a $\sigma$-algebra. Then,
\begin{enumerate}
    \item $\emptyset \in \mathcal{F}$ which is proven in Lemma 1.6.
    \item Closedness under union
    \begin{align}
        A_1, \ldots, A_N \in \curlyf \implies \bigcup_i^N A_i \in \curlyf
    \end{align}
    \begin{proof}
        Using $A_{N+1}, ... \equiv \emptyset$, the union of all of these must be an element of $\curlyf$
    \end{proof}
    \item Closedness under intersection
    \begin{align}
        A_1, \ldots, A_N \in \curlyf \implies \bigcap_i^N A_i \in \curlyf
    \end{align}
    \begin{proof}
        Take
        \begin{align}
            A_1, \ldots, A_N \in \curlyf 
        \end{align}
        Then, by definition,
        \begin{align}
            A_1^C, \ldots, A_N^C \in \curlyf 
        \end{align}
        By definition and then De Morgan's Laws,
        \begin{align}
            A_1^C, \ldots, A_N^C \in \curlyf  \implies  \bigcup A_i^C \in \curlyf \implies \left( \bigcap A_i \right)^C \in \curlyf \implies \bigcap A_i \in \curlyf & \qedhere
        \end{align}
    \end{proof}
\end{enumerate}

\subsubsection{Generated Sigma Algebras}
Let $\Omega \equiv \natural$. Then,
\begin{align}
    \curlyf \equiv \{ \emptyset, \{ 1 \}, \{ 1, 2 \}, ... \}
\end{align}
is not a $\sigma$-algebra because $\{ 1, 2 \} - \{ 1 \} = \{ 2 \}$ is not an element of $\curlyf$. In spite of this, we can define a $\sigma$-algebra $\widetilde{\curlyf}$ to be the intersections of all $\sigma$-algebras that contain $\curlyf$, i.e. for some non-sigma-algebra subset of sets $\curlyf$,
\begin{align}
    \widetilde{\curlyf} \equiv \bigcap \mathcal{G} \qsp \forall \mathcal{G}_\sigma \supset \mathcal{F}
\end{align}
\begin{proposition}
    $\widetilde{\curlyf}$ is a $\sigma$-algebra.
\end{proposition}
For the case in (2.5) specifically, we assert that $\widetilde{\curlyf} \equiv 2^\natural$, i.e. the power set of natural numbers. To show this, see that if $\{ 1 \} \in \curlyf$ then $\{ 2 \} \in \curlyf$ for $\{ 1, 2 \} \in \curlyf$. For similar reasons,
\begin{align}
    \{ n \} \in \curlyf \qsp \forall n \in \natural
\end{align}
and by taking the union of these sets, all subsets of $\natural$ can be composed of these singleton subsets.

\subsubsection{Properties of a Probability Measure}
\begin{enumerate}
    \item The probability of nothing... is nothing!
    \begin{align}
        \probability(\emptyset) = 0
    \end{align}
    \begin{proof}
        \begin{align}
            \probability(\emptyset \union \Omega) = \probability(\Omega) = \probability(\emptyset) + \probability(\Omega) \implies \probability(\emptyset) = 0
        \end{align}
    \end{proof}
    \item Probability of unions
    \begin{align}
        \probability(A \cup B) = \probability(A) + \probability(B) - \probability(A \cap B)
    \end{align}
    To prove this without using pictures, we can express $A$, $B$, and $A \cup B$ as disjoint sets
    \begin{align}
        A \cup B &= (A - B) \cup (A \cap B) \cup (B - A)\\
        A &= (A - B) \cup (A \cap B)\\
        B &= (B - A) \cup (A \cap B)
    \end{align}
    This means that
    \begin{align}
        \probability(A) + \probability(B) = \probability(A - B) + \probability(B - A) + 2 \probability(A \cap B)
    \end{align}
    and clearly, there is one extra $\probability(A \cap B)$.
    \item Subset probability. Given $B \subseteq A$, $\probability(A - B) = \probability(A) - \probability(B)$ which implies $\probability(B) \le \probability(A)$ due to non-negative probabilities being not possible by definition. We can write $A$ and $B$ as disjoint sets,
    \begin{align}
        A &= B \cup (A-B)\\
        B &= B
    \end{align}
    and then
    \begin{align}
        \probability(A) &= \probability(B) + \probability(A - B)\\
        \probability(B) &= \probability(B)\\
        \therefore \probability(A) - \probability(B) &= \probability(A - B) (\ge 0) & \qedhere
    \end{align}
    \item Union probability less than sum
    \begin{align}
        \probability\left( \bigcup A_i \right) \le \sum \probability(A_i)
    \end{align}
    \begin{proof}
        We can again write these in a different way
        \begin{align}
            \bigcup A_i = (A_1) \cup (A_1^C \cap A_2) \cup (A_1^C \cap A_2^C \cap A_3) \cup \cdots
        \end{align}
        Note that by Property 3, $\probability(A_1^C \cap A_2) \le \probability(A_2)$ and the same applies to all of the other ones. So,
        \begin{align}
            \probability\left( \bigcup A_i \right) \le \probability(A_1) + \probability(A_2) + \cdots & \qedhere
        \end{align}
    \end{proof}
\end{enumerate}


\subsection{Combinatorics and Counting}
Take $\Omega = \{ a_1, \cdots, a_N \}$ as some sample space with $\abs{\Omega} = N$. Take $\curlyf \equiv 2^\Omega$ as all subsets of $\Omega$ and define
\begin{align}
    \probability(a_i) = 1/N \qsp \forall i
\end{align}
which generalizes to
\begin{align}
    \probability(A) = \dfrac{\abs{A}}{\abs{\Omega}}
\end{align}
To compute $\probability$, we have to be able to count $\abs{A}$, which requires an overview of combinatorics. 

\subsubsection{Ordered with Replacement}
Suppose we have a license plate with five letters. Then, there are $26^5$ possible combinations because we can reuse letters, and the order matters. In general, for a set of size $N$ and $R$ repeats, there are
\begin{align}
    N^R \qsp \text{permutations}
\end{align}

\subsubsection{Ordered without Replacement}
If we do not replace then we cannot reuse letters. So, for the license plate we have $26 \cdot 25 \cdots 22$ combinations. In general, for $N$ and $R$, we have
\begin{align}
    \frac{N!}{(N-R)!} = \, ^NP_R = (N)_R \qsp \text{permutations}
\end{align}

% \subsubsection{Unordered with Replacement}
% Unordered means there are $N!$ permutations with the same set of characters, i.e. we are overcounting by $N!$. So, we have
% \begin{align}
%     \frac{N^R}{R!} \qsp \text{permutations}
% \end{align}

\subsubsection{Unordered without Replacement}
By the same logic, we now have
\begin{align}
    \frac{N!}{R!(N-R)!} = \binom{N}{R} \qsp \text{permutations}
\end{align}


    \section{September 27, 2024}

\subsection{Counting Problems (cont)}
Recall counting formulas:
\begin{enumerate}
    \item Ordered samples of $r$ objects out of $n$ with replacement is $n^r$
    \item Ordered samples of $r$ objects out of $n$ with\textit{out} replacement is $\frac{n!}{(n-r)!}$
    \item Unordered samples of $r$ objects out of $n$ without replacement is $\frac{n!}{r!(n-r)!} = \binom{n}{r}$
\end{enumerate}
\begin{example}
    There are 10 balls in an urn numbered 1 through 10. Randomly draw 5 balls without replacement. What is the probability of the second largest number being 8?
    \begin{solution}
        Ways to choose 5 balls out of 10 is
        \begin{align}
            \binom{10}{5}
        \end{align}
        if we do not care about the order. How many of these combinations have the second largest number of 8? There are two possibilities: largest number is 9 or largest number is 10. So there are
        \begin{align}
            \underset{9 \text{ or } 10}{2} \times \underset{\text{choose 3 from 7 remaining}}{\binom{7}{3}}
        \end{align}
        This sets one choice as 8, one as one of 9 or 10, and the rest as arbitrary picks that are not 8, 9, or 10. So the probability is
        \begin{align}
            \probability = \dfrac{2 \times \binom{7}{3}}{\binom{10}{5}}
        \end{align}
    \end{solution}
\end{example}
\subsubsection{Unordered Samples \textit{With} Replacement}
How many Scrabble combinations of 7 letters are there if there are only A, B, and C? It is not as simple as $N^R/R!$ because there can be repeated elements which adds a degree of nuance. We can rewrite some sequence using ``stars and bars'' into
\begin{align}
    AABCABA \implies ***\,* \mid *\,* \mid *
\end{align}
To count the number of combinations, there is one slot for each star and one slot for each bar. Each slot can either be a star or a bar. So we pick 7 positions from 9 positions:
\begin{align}
    \binom{9}{7} \equiv \binom{9}{2} = 36
\end{align}
ways to arrange these stars and bars.
\begin{proposition}
    The number of unordered samples of $R$ objects out of $N$ is
    \begin{align}
        \binom{R+N-1}{R} = \binom{R+N-1}{N-1}
    \end{align}
\end{proposition}

\subsection{Independence}
\begin{proposition}
    Let $A$ and $B$ be two independent events. We say $A$ and $B$ are independent if $\probability(A \text{ and } B) = \probability(A) \cdot \probability(B)$
\end{proposition}
What if we have $N$ events? How can this definition be generalized?
\begin{definition}
    A family of events $\mathcal{A} \equiv \{ A_i \}_{i \in I}$ (where $I$ is some set of indices such as $\natural$) are \textbf{independent} if and only if for every finite subset $A' \subseteq \mathcal{A}$,
    \begin{align}
        \probability\left(\bigcap A'_i\right) = \prod \probability(A'_i)
    \end{align}
\end{definition}

\subsubsection{Properties of Independent Events}
Recall the properties of a sample space defined previously.
\begin{lemma}
    Let $\Omega$ be a sample space and let $A$ be an event with probability $\probability(A)$. Then, $\probability(A^C) = 1- \probability(A)$.
    \begin{proof}
        It must be true that
        \begin{align}
            A \cup A^C = \Omega
        \end{align}
        so
        \begin{align}
            \probability(A) + \probability(A^C) = \probability(A \cup A^C) = \probability(\Omega) = 1
        \end{align}
    \end{proof}
\end{lemma}
\begin{enumerate}
    \item If $A$ and $B$ are independent, then $\probability(A \cap B^C) = \probability(A) \cdot \probability(B^C)$. That is, if an event is independent with another, then the first event is independent with the other not happening as well.
    \begin{align}
        A, B \text{ indep.} \implies \left\{ A, A^C \right\} \times \left\{ B, B^C \right\} \text{ all indep.}
    \end{align}
    \begin{proof}
        Given $A$ and $B$ are independent, $\probability(A \cap B) = \probability(A)\probability(B)$. Then,
        \begin{align}
            \probability(A \cap B^C) = \probability(A - B) = \probability(A - (A \cap B))
        \end{align}
        But, if $B \subseteq A$, then $\probability(A - B) = \probability(A) - \probability(B)$. So,
        \begin{align}
            \probability(A \cap B^C) \stackrel{\ldots}{=} \probability(A - (A \cap B)) &= \probability(A) - \probability(A)\probability(B)\\
            &= \probability(A) \left[ 1 - \probability(B) \right]\\
            &= \probability(A) \probability(B^C)
        \end{align}
    \end{proof}
    This works for the other two non-trivial elements of (3.10) as well.
\end{enumerate}
\begin{proposition}
    For some family $\{ A_i \}_{i \in I}$ of independent events, define $B_\alpha \equiv (A_\alpha \text{ or } A_\alpha^C)$. Then,
    \begin{align}
        \probability\left( \bigcap B_i \right) = \prod \probability(B_i)
    \end{align}
    where here, each event is either some event in $A$ or its complement.
\end{proposition}

\subsection{Bernoulli Trials}
Suppose some factory produces batteries and 5\% of all batteries are defective. These are independent defective batteries which makes the QA job difficult. Suppose the factory makes 10 batteries. What is the probability that exactly 3 of them are defective? The first three being defective has probability
\begin{align}
    \probability\left[ \text{first 3 are defective; rest are ok} \right] = \left( \dfrac{1}{20} \right)^3 \cdot \left( \dfrac{19}{20} \right)^7
\end{align}
But, there are multiple ways to pick three, so the probability there is
\begin{align}
    \binom{10}{3} \cdot \probability\left[ \text{first 3 are defective; rest are ok} \right]
\end{align}
\begin{definition}
    For repeating some binary event with probability of success $p$ for $N$ independent trials, the probability of succeeding exactly $k$ times is
    \begin{align}
        \probability(k \text{ successes}) = \binom{N}{k} \cdot p^k \cdot (1-p)^{N-k}
    \end{align}
\end{definition}

    \section{September 30, 2024}

\subsection{Generalized Bernoulli Trials}
Recall for $n$ independent trials and the probability of success for each trial is $p$. Then,
\begin{align}
    \probability(k\text{ successes}) = \binom{n}{k} p^{k} (1-p)^{n-k}
\end{align}
What if there are more than two outcomes? Let there be $n$ independent trials and $k$ possible outcomes $b_1, \ldots, b_k$ for each trial with associated probabilities $p_1, \ldots, p_k$ ($\sum p_i = 1$), e.g. rolling a dice $n$ times.\\

The sample space $\Omega = $ set of all finite sequences of length $n$ where each entry is one of $b_1,\ldots,b_k$ (there are $k^n$ such sequences). We want to compute the probability of exactly $n_1$ occurrences of $b_1$, $n_2 \to b_2, \ldots, n_k\to b_k$. Define this as
\begin{align}
    p(n_1, n_2, \ldots, n_k) \qsp \sum n_i = n
\end{align}

First, we can compute the probability of the first $n_1$ being $b_1$; next $n_2$ being $b_2$; etc. That equals
\begin{align}
    p_1^{n_1} \cdots p_k^{n_k}
\end{align}
We also need to scale by the total number of arrangements, i.e. number of ways to get $n_1$ occurrences of $b_1, \ldots, n_k$ of $b_k$. That equals
\begin{align}
    \binom{n}{n_1}\binom{n-n_1}{n_2}\binom{n-n_1-n_2}{n_3}\cdots\binom{n_k}{n_k}
\end{align}
Can we simplify this? After some trivial arithmetic (write it out), this reduces to
\begin{align}
    \dfrac{n!}{n_1! n_2! n_3! \cdots n_k!} = n! \cdot \prod_i (n_i)!^{-1}
\end{align}
so the total probability is
\begin{definition}
    \textbf{Generalized Bernoulli Trials}
    \begin{align}
        p(n_1, n_2, \ldots, n_k) = n! \cdot \left[\prod_i (n_i)!^{-1}\right] \cdot p_1^{n_1} \cdots p_k^{n_k}
    \end{align}
\end{definition}

\begin{example}
    Take an urn with black, white, red, and green balls. Randomly and independently draw four balls with replacement. What is the probability that I have exactly two distinct colors?
\end{example}
\begin{solution}
    Let $(b_1, b_2, b_3, b_4)$ = (black, white, red, green). Each is equally likely. What is the probability I have exactly two black and two white?
    \begin{align}
        \probability(b_1=2, b_2=2, 0, 0) = \dfrac{4!}{2!2!} \cdot (1/4)^2 \cdot (1/4)^2 = 3/128
    \end{align}
    That is a simple case. For two of one color and two of another, there are $\binom{4}{2}$ ways to pick two colors, which is six. So, that sub-probability is $18/128 = 9/64$ (two of one color; two of another). We still have to do one of one color and three of another.
    \begin{align}
        \probability(3, 1, 0, 0) = \dfrac{4!}{3!1!} \cdot (1/4)^3 \cdot (1/4)^1 = 1/64
    \end{align}
    There are twelve ways to do this (six in one way; six in the other), so this sub-probability is $12/64$. The total is $21/64 \approx 1/3$.
\end{solution}

\subsection{Conditional Probability}
% bayes?
Take two events $A, B$ in some probability space. What is
\begin{align}
    \probability(A \mid B) \qand \probability(B \mid A)
\end{align}
or the probabilities of some event happening given another has happened? Intuitively,
\begin{definition}
    The probability $B$ occurs \textit{given} $A$ occurs is
    \begin{align}
        \probability(B \mid A) = \dfrac{\probability(A \cap B)}{\probability(A)}
    \end{align}
\end{definition}
\begin{lemma}
    If $A$ and $B$ are independent,
    \begin{align}
        \probability(B \mid A) = \dfrac{\probability(A \cap B)}{\probability(A)} = \dfrac{\probability(A) \probability(B)}{\probability(A)} = \probability(B)
    \end{align}
    which is quite intuitive.
\end{lemma}
\begin{example}
    Roll a fair die once. $A$ is rolling an odd number and $B$ is rolling a 5.
    \begin{align}
        \probability(B \mid A) = \dfrac{1/6}{1/2} = \dfrac{1}{3}
    \end{align}
\end{example}
\begin{example}
    Throw two dice. Let $A$ be that the highest roll is a six; let $B$ be that the sum is a ten.
    \begin{align}
        \probability(B \mid A) = \dfrac{2/36}{11/36} = 2/11 \qsp \probability(A \mid B) = \dfrac{2/36}{3/36} = 2/3
    \end{align}
\end{example}
\noindent Notice that
\begin{align}
    \probability(A \cap B) = \probability(A) \probability(B \mid A) = \probability(B) \probability(A \mid B)
\end{align}
This yields
\begin{theorem}
    \textbf{Bayes' Theorem}
    \begin{align}
        \probability(A \mid B) = \dfrac{\probability(A) \probability(B \mid A)}{\probability(B)}
    \end{align}
\end{theorem}
What happens when there are multiple events, namely $A, B, C, D$?
\begin{align}
    \probability(A \cap B \cap C) &= \probability(A \cap B) \probability(C \mid A\cap B)\\
    &= \probability(A)\probability(B \mid A) \probability(C \mid A\cap B)
\end{align}
We can keep going.
\begin{proposition}
    \begin{align}
        \probability\left(\bigcap A_i\right) = \prod_i \probability\left(A_i \mid \bigcap_j^{i-1} A_j\right)
    \end{align}
\end{proposition}
\begin{example}
    Draw three cards randomly from a regular deck without replacment. Find the probability that there is no ace in the three cards. Let $A_i = i-\text{th card is not an ace}$
    \begin{align}
        \probability(A_1 \cap A_2 \cap A_3) = \probability(A_1) \probability(A_2 \mid A_1) \probability(A_3\mid A_1 \cap A_2)
    \end{align}
    We can enumerate some probabilities
    \begin{align}
        A_1 &= 48/52 & \text{52 total cards; 4 aces}\\
        A_2 &= 47/51\\
        A_3 &= 46/50
    \end{align}
\end{example}

\subsection{Law of Total Probability}
\begin{definition}
    Events $B_1, B_2, \ldots$ are \textbf{mutually exclusive} if they are disjoint.
\end{definition}
\begin{definition}
    Events $B_1, B_2, \ldots$ are exhaustive if $\Omega = \bigcup_i B_i$
\end{definition}
Combining these two yields the law of total probability
\begin{theorem}
    Let $B_1, B_2, \ldots$ be mutually exclusive and exhaustive. Then,
    \begin{align}
        \probability(A) = \sum_i \probability(A \cap B_i)
    \end{align}
    and
    \begin{align}
        \probability(A) = \sum_i \probability(B_i) \probability(A \mid B_i)
    \end{align}
    for all $i \mid \probability(B_i) > 0$.
\end{theorem}

    \section{October 2, 2024}

\subsection{Law of Total Probability (properly)}
\begin{theorem}
    Take some probability space $(\Omega, \curlyf, \probability)$. If $B_1, B_2, \ldots \in \curlyf$ are mutually exclusive and exhaustive (see 4.3), for $A \in \curlyf$
    \begin{align}
        \probability(A) = \sum_{i} \probability(A \cap B_i)
    \end{align}
\end{theorem}
\begin{proof}
    Trivially,
    \begin{align}
        \probability(A) = \probability(A \cap \Omega)
    \end{align}
    But, $B_1 \cap \cdots = \Omega$. So this equals
    \begin{align}
        \probability(A) &= \probability\qty(A \cap (B_1 \cap \cdots))\\
        &= \probability\qty(\bigcup(A \cap B_i))
    \end{align}
    Since $B_i$ are disjoint, these are all disjoint, so this becomes
    \begin{align}
        \probability(A) = \sum_i \probability(A \cap B_i)
    \end{align}
\end{proof}

Recall conditional probability if $\probability(A) \ne 0$
\begin{align}
    \probability(B \mid A) = \dfrac{\probability(A \cap B)}{\probability(A)}
\end{align}

\begin{theorem}
    For $\probability(B_i) > 0$, Theorem 5.1 using conditional probabilities equals
    \begin{align}
        \probability(A) = \sum_i \probability(B_i) \probability(A \mid B_i)
    \end{align}
\end{theorem}

\subsection{Bayes' Formula}
\begin{proposition}
    For $B_1, B_2, \ldots$ mutually exclusive and exhaustive,
    \begin{align}
        \probability(B_k \mid A) = \dfrac{\probability(A \cap B_k)}{\probability(A)}
    \end{align}
    which can be rewritten into
    \begin{align}
        \boxed{\probability(B_k \mid A) = \dfrac{\probability(B_k) \probability(A \mid B_k)}{\sum_i \probability(B_i) \probability(A \mid B_i)}}
    \end{align}
\end{proposition}

\begin{example}
    Throw a die with outcome $i \in \{1, \ldots, 6\}$. Then, flip a coin $i$ times. Find the conditional probability that the dice landed on $3$ given at least one head was obtained.
    \begin{proof}[Solution]
        The professor drew a beautiful, branching tree. Each outcome of a dice corresponds to some probabilities in terms of number of heads. This can be used to easily compute the conditional probability, using Equation 5.9.
    \end{proof}
\end{example}

\subsection{Borel Sets}
Take $\Omega = \real$.
\begin{definition}
    The Borel $\sigma$-algebra on $\real$, $\mathcal{B}$, is the $\sigma$-algebra of subsets of $\real$ generated by\footnote{smallest $\sigma$-algebra containing} the closed intervals for all $[a,b]$ where $a \le b$.
\end{definition}
The elements of $\mathcal{B}$ are Borel sets.
\begin{proposition}
    Some facts
    \begin{align}
        (a,b) = \bigcup_{n=1}^\infty [a + 1/n, b - 1/n] \implies {(a,b) \in \mathcal{B}}
    \end{align}
    \begin{align}
        [a,\infty) = \bigcup_{n=1}^\infty [a, a + n] \implies {[a,\infty) \in \mathcal{B}}
    \end{align}
    \begin{align}
        [z,z] \in \mathcal{B} \, \forall z \in \integer \implies \integer \in \mathcal{B}
    \end{align}
\end{proposition}

\subsubsection{Example of a probability space involving $\mathcal{B}$}
Let $\Omega = \real$ and $\curlyf = \mathcal{B}$. Let $f(x)$ be a normalized and non-negative (Riemann) integrable function on $\real$. Then, define the probability measure
\begin{align}
    \probability(B) = \int_{B} f(x) \dd{x}
\end{align}
But how do we integrate over a Borel set? We use the unique probability measure $\probability$ on $\mathcal{B}$ and redefine
\begin{align}
    \probability([a,b]) = \int_a^b f(x) \dd{x}
\end{align}
\begin{proof}
    This is measure theory so the professor refused.
\end{proof}

\subsection{Random Variables}
A random variable is a function on the sample space $\Omega$ which we want to measure.
\begin{definition}
    Let $(\Omega, \curlyf, \probability)$ be a probability space. A random variable is a real-valued function
    \begin{align}
        R: \Omega \to \real
    \end{align}
    such that for $a,b\in\real$ with $a\le b$, the set $\{ \omega \in \Omega \mid a \le R(\omega) \le b \}$ is an element of $\curlyf$. This is equivalent to the inverse image of $R^{-1}([a,b])$.
\end{definition}
\begin{proposition}
    If $\curlyf = 2^\Omega$, then every function $R: \Omega \to \real$ is a random variable.
\end{proposition}
\begin{example}
    Flip a coin 6 times. $\Omega$ is all of the possible outcomes. The function
    \begin{align}
        R = \text{number of heads}
    \end{align}
    is a random variable. Then, the probability
    \begin{align}
        \probability(0 \le R \le 1) = \frac{1}{2^6} + \frac{6}{2^6}
    \end{align}
\end{example}
\begin{example}
    Roll two dice. $\Omega$ is all of the possible outcomes. So
    \begin{align}
        \Omega = \left\{ (i,j) \mid i, j \in \{ 1, \ldots, 6 \} \right\}
    \end{align}
    Define a random variable $R = i + j$. We can compute
    \begin{align}
        \probability(0 \le R \le 3) = 0 + 0 + 1/36 + 2/36 = 1/12
    \end{align}
\end{example}
Then, a less trivial example:
\begin{example}
    Take $\Omega = \real$ and $\curlyf = \mathcal{B}$ and probability measure on function $f(x)$. Define
    \begin{align}
        R: \real \to \real \qsp R(x) = x + 1
    \end{align}
    Then,
    \begin{align}
        R^{-1}([a,b]) = [a-1,b-1]
    \end{align}
    which is obviously an element of $\mathcal{B}$.
\end{example}
Most functions (and most continuous ones) are random variables.

    \section{October 4, 2024}
\subsection{Random Variables}
Take $(\Omega, \curlyf, \probability)$ as a probability space. Recall

\begin{definition}
    A random variable is some function
    \begin{align}
        R : \Omega \to \real
    \end{align}
    such that $R^{-1}([a,b]) \in \curlyf$. Note that
    \begin{align}
        R^{-1}([a,b])
    \end{align}
    describes all points $\chi \in \Omega$ such that $R(\chi)$ maps into $[a,b] \in \real$.
\end{definition}
\begin{definition}
    A random variables is some function
    \begin{align}
        R: \Omega \to \real
    \end{align}
    such that $R^{-1}(B) \in \curlyf$ for all Borel sets $B \in \mathcal{B}$ on $\real$.
\end{definition}

\begin{proposition}
    The above two definitions are equivalent.
\end{proposition}
\begin{proof}
    Suppose $R$ is a random variable as defined in Definition 6.1. Define
    \begin{align}
        \mathcal{G} = \{ G\subset \real \mid R^{-1}(G) \in \curlyf \}
    \end{align}
    By Definition 6.1, all closed intervals are in $\mathcal{G}$. We now claim that $\mathcal{G}$ is a $\sigma$-algebra on $\real$. If this is true, then $\mathcal{G}$ being a $\sigma$-algebra containing all closed intervals means it contains all Borel sets $B \in \mathcal{B}$, i.e. $\mathcal{B} \subseteq \mathcal{G}$. This means Definition 6.2 $\subseteq$ Definition 6.1.
    \begin{lemma}
        \textbf{The Claim} $\mathcal{G}$ is a $\sigma$-algebra on $\real$
    \end{lemma}
    \begin{proof}[Proof of Lemma 6.4]
        \begin{align}
            R^{-1}(\real) &= \Omega \in \curlyf \implies \real \in \mathcal{G}\\
            G_i \in \mathcal{G} &\implies R^{-1}(G_i) \in \curlyf \implies \bigcup_i R^{-1}(G_i) \in \curlyf\\
            &\implies R^{-1}\qty(\bigcup_i G_i) \in \curlyf \implies \qty(\bigcup_i G_i) \in \mathcal{G}\\
            G \in \mathcal{G} &\implies R^{-1}(G) \in \curlyf \implies \qty(R^{-1}(G))^C \in \curlyf \implies G^C \in G
        \end{align}
    \end{proof}
    On the converse, if we have a random variable as defined in Definition 6.2, then that implies that
    \begin{align}
        R^{-1}([a,b]) \in \curlyf
    \end{align}
    for all $a \le b$ because all closed intervals are Borel sets. Then that trivially proves the definition, i.e. Definition 6.1 $\subseteq$ Definition 6.2. Thus,
    \begin{align}
        \text{Definition 6.1} \equiv \text{Definition 6.2} & \qedhere
    \end{align}
\end{proof}

\subsection{Discrete Random Variables}
Take some probability space. Take some random variable
\begin{align}
    R: \Omega \to \real
\end{align}
\begin{definition}
    A random variable $R: \Omega \to \real$ is discrete if $\Im(R)$ is a finite or countably infinite set of points, i.e. $R$ hits a countable number of points.
\end{definition}
\begin{example}
    Flip a coin six times. $\Omega$ is the set of all possible outcomes. Take $R: \Omega \to \natural$ as defined as the number of heads. The image of $R$ is
    \begin{align}
        \kappa := \{ 0, \ldots, 6 \}
    \end{align}
    which is discrete.
\end{example}
\noindent Define a probability function $\probability_R(x)$ as
\begin{align}
    \probability(R = x)
\end{align}
Then, $\probability(R \in B)$ where $B \in \mathcal{B}$ is the sum
\begin{align}
    \sum_{x \in B} \probability_R(x)
\end{align}
Define a distribution function $F_R(x)$ as
\begin{align}
    F_R(x) = \probability(\{ R \le x \}) = \sum_{t \le x} \probability_R(t)
\end{align}
In Example 6.6,
\begin{align}
    \probability_R(k \in \kappa) = \binom{6}{k}\left( \frac{1}{2} \right)^6
\end{align}
For $k \notin \kappa$, $\probability(k) = 0$. The distribution function is then
\begin{align}
    F_R(x) = \sum_{t=0}^x \probability_R(t) \dd{t}
\end{align}

\subsection{Absolutely Continuous Random Variables}
Take a probability space $(\real, \mathcal{B}, \probability)$. Define
\begin{align}
    \probability(B \in \mathcal{B}) = \int_B f(x) \dd{x}
\end{align}
such that $f$ is a given probability density function which is \textbf{non-negative}, \textbf{integrable}, and
\begin{align}
    \int_\real f(x) \dd{x} = 1
\end{align}
Define a random variable $R(\omega) \equiv \omega$. As in the same way we defined a distribution function,
\begin{align}
    F_R(x) \equiv \probability(R \le x)
\end{align}
Based on the integral above,
\begin{align}
    F_R(x) = \int_{-\infty}^x f(\omega) \dd{\omega}
\end{align}
\begin{definition}
    Given a probability space $(\Omega, \curlyf, \probability)$ and a random variable $R: \Omega \to \real$. $R$ is absolutely continuous if there exists an integrable density function $f_R \ge 0$ such that $F_R(x) = \int_{-\infty}^x f_R(t) \dd{t}$
\end{definition}
\begin{proposition}
    Let $R$ be absolutely continuous with a density function $f_R$. Then,
    \begin{align}
        \probability(R \in (a,b]) = \int_{a}^b f_R(t) \dd{t}
    \end{align}
    \begin{proof}
        Trivial
    \end{proof}
    This is equivalent to
    \begin{align}
        \probability(R \in B) = \int_B f(x) \dd{x}
    \end{align}
    for Borel set $B \in \mathcal{B}$.
\end{proposition}
\begin{lemma}
    For an \textbf{absolutely continuous} random variable $R$ with density function $f_R$, $\probability\qty(R = \chi)$ for some fixed value $\chi$ equals zero. This does not mean $\chi$ \textit{never} occurs, but the probability is zero.
    \begin{proof}
        Trivial
    \end{proof}
\end{lemma}
\begin{example}
    \textbf{Uniform Distribution}
    Take probability space $(\real, \mathcal{B}, \probability)$ where $\probability$ is some density function.
    \begin{align}
        \probability(B) = \int_B f(x) \dd{x}
    \end{align}
    such that the density function is defined as
    \begin{align}
        f_R(x) = \begin{cases}
            \frac{1}{b-a} & a \le x \le b\\
            0 & \text{else}
        \end{cases}
    \end{align}
    for some constants $a,b$. This has uniform probability for a subset of $\real$. Then, the cumulative distribution function equals
    \begin{align}
        F_R(x) = \probability(R \le x) = \begin{cases}
            0 & x < a\\
            \int_a^{x} \frac{1}{b-a} \dd{t} = \frac{x-a}{b-a} & x \in [a,b]\\
            1 & x > b
        \end{cases}
    \end{align}
\end{example}

% \subsection{Functions of a Random Variable}



    \section{October 7, 2024}

\subsection{Recap: Absolutely Continuous Random Variables}
Given some probability space $\psp$ and a random variable $R: \Omega \to \real$, recall the distribution function
\begin{align}
    F_R(x) = \probability(R \le x)
\end{align}
$R$ is absolutely continuous if there exists a non-negative, integrable density function $f_R: \real \to \real$
\begin{align}
    F_R(x) = \int_{-\infty}^x f_R(x) \dd{x}
\end{align}
In this case, then
\begin{align}
    \probability(R \in B) = \int_B f_R(x) \dd{x}
\end{align}
Some remarks:
\begin{enumerate}
    \item If $R$ is a absolutely continuous random variable, then $F_R$ is continuous.
    \begin{proof}[Hand Waving]
        We need to show that
        \begin{align}
            \lim_{x\to a} F_R(x) = F_R(a)
        \end{align}
        But, $F_R(x) = \int_{-\infty}^x f(t) \dd{t}$, so
        \begin{align}
            \lim_{x\to a} F_R(x) = \underbracket{\cdots\cdots}_{\text{measure theory}} = F_R(a)
        \end{align}
        $\varepsilon-\delta$ was not done.
    \end{proof}
    \item If the density function $f_R$ is continuous, then the fundamental theorem of calculus yields
    \begin{align}
        \dv{F_R(x)}{x} = f_R(x)
    \end{align}
    That is, the (probability) density function is the derivative of the (cumulative) distribution function.
    \item ``Let $R$ be an absolutely continuous random variable with density function $f(x)$'' means that we can construct $R$ using the probability space $(\real, \mathcal{B}, \probability)$ where
    \begin{align}
        \probability(B) = \int_B f(x) \dd{x}
    \end{align}
    using the given density function $f(x)$. Then, $R$ is a random variable with density $f$. The density and distribution functions well-define $R$. We don't care about $R(\Omega)$ as much as $f_R$ and $F_R$.
\end{enumerate}

\subsection{Functions of Random Variables}
Let $R_1$ be a random variable uniformly distributed on $[0,1]$, i.e.
\begin{align}
    f_1(x) = \begin{cases}
        1 & 0 \le x \le 1\\
        0 & \text{else}
    \end{cases}
\end{align}
Define $R_2$ as
\begin{align}
    R_2 = \qty(R_1)^2
\end{align}
Then, we want to find $F_2(x)$. First, note
\begin{align}
    F_1(x) = \probability(R_1 \le x) = \begin{cases}
        0 & x < 0\\
        x & x \in [0,1]\\
        1 & x > 1
    \end{cases}
\end{align}
To find the distribution function $(F_2)_{R_2}$,
\begin{align}
    F_2(u) = \probability(R_2 \le u) = \probability(R_1^2 \le u)
\end{align}
If $u \ge 0$, then
\begin{align}
    F_2(u) &= \probability\qty(R_1 \in [-\sqrt{u}, \sqrt{u}])\\
    &= \int_{-\sqrt{u}}^{\sqrt{u}} f_1(t) \dd{t}\\
    &= \begin{cases}
        0 & u < 0 \\
        \sqrt{u} & 0 \le u \le 1\\
        1 & \text{else}
    \end{cases}
\end{align}
The derivative, i.e. the density function, equals
\begin{align}
    f(u) = \begin{cases}
        0 & u < 0\\
        \frac{1}{2\sqrt{u}} & u \in [0,1] \\
        0 & u > 1
    \end{cases}
\end{align}
\begin{example}
    A slightly trickier example. Take $R_1$ with density function
    \begin{align}
        f_1(x) = \begin{cases}
            0 & x < 0\\
            e^{-x} & x \ge 0
        \end{cases}
    \end{align}
    and $R_2$ as
    \begin{align}
        R_2 = \begin{cases}
            R_1 & R_1 \le 1\\
            1/R_1 & R_1 > 1
        \end{cases}
    \end{align}
    What is $\text{cdf}(R_2)$ (or $F_2$)?
\end{example}
\begin{solution}
    We can split into ranges. When $y \le 0$,
    \begin{align}
        F_2(y) = \probability(R_2 \le y) = \int_{(-\infty,y]} f_1(x) \dd{x} = 0
    \end{align}
    For $y > 0$,
    \begin{align}
        F_2(y) = \probability(R_2 \le y)
    \end{align}
    Now, there are two cases: one where $y \le 1$ and one where $y > 1$.
    \begin{align}
        \probability(R_2 \le y \qand R_1 \le 1) \qand \probability(R_2 \le y \qand R_1 > 1)
    \end{align}
    So,
    \begin{align}
        F_2(y \mid y > 0) &= \probability(R_1 \le y \text{ and } R_1 \le 1) + \probability\qty(\frac{1}{R_1} \le y \text{ and } R_1 > 1)\\
        &= F_1(y) + \probability(R_1 > 1/y)\\
        &= \left[ 1 - e^{-y} \right] + \left[ 1 - \probability(R_1 \le 1/y) \right]\\
        &= \cdots = 1 - e^{-y} + e^{-1/y}
    \end{align}
    As $y \to 1$, $F_2(y) \to 1$. So, other than this case,
    \begin{align}
        F_2(y) = \begin{cases}
            1 & y > 1\\
            0 & y < 0
        \end{cases} & \qedhere
    \end{align}
\end{solution}

\subsection{Properties of Distributions Functions}
Take $\psp$ as some probability space and $R$ as a random variable. Take $F(x)$ as
\begin{align}
    \probability(R \le x)
\end{align}
Then,
\begin{proposition}
    \begin{enumerate}
        \item Let $A_1, A_2, \ldots \in \curlyf$ be an expanding sequence (i.e. $A_m \supseteq A_n$ for all $m > n$). Then,
        \begin{align}
            \probability\qty(\bigcup_{n=1}^\infty A_n) = \lim_{n\to\infty} \probability(A_n)
        \end{align}
        \item Let $A_1, A_2, \ldots \in \curlyf$ be a contracting sequence (i.e. $A_m \subseteq A_n$ for all $m > n$). Then,
        \begin{align}
            \probability\qty(\bigcap_{n=1}^\infty A_n) = \lim_{n\to\infty} \probability(A_n)
        \end{align}
    \end{enumerate}
\end{proposition}
\begin{proof}[Proof of the first]
    Take $A = \bigcup_{n} A_n$. Then,
    \begin{align}
        A = A_1 \cup (A_2 - A_1) \cup (A_3 - A_2) \cup \cdots
    \end{align}
    These are clearly disjoint. It follows that
    \begin{align}
        \probability(A) = \sum_{i=1} \probability(A_i - A_{i-1})
    \end{align}
    where $A_0 \equiv \emptyset$. This equals
    \begin{align}
        \probability(A_1) + \lim_{n\to\infty} \sum_{i=1}^\infty \qty[\probability(A_{i+1}) - \probability(A_i)]
    \end{align}
    But there are cascading cancellations, so
    \begin{align}
        \probability(A) = \probability(A_1) + \lim_{n \to \infty} \left( \probability(A_{n+1}) - \probability(A_1) \right)
    \end{align}
    Thus,
    \begin{align}
        \probability(A) = \lim_{n \to \infty} \probability \qty(A_{n+1 \equiv n})
    \end{align}
\end{proof}

Some properties of distribution functions
\begin{enumerate}
    \item $F$ is non decreasing, i.e. if $a < b$ then $F(a) \le F(b)$.
    \begin{proof}
        \begin{align}
            a < b \implies \{ R \le a \} \subset \{ R \le b \}
        \end{align}
    \end{proof}
    \item Infinite limit
    \begin{align}
        \lim_{x \to \infty} F(x) = 1
    \end{align}
    For the absolutely continuous case this is trivial. Let $x_n$ be a sequence of real numbers with $x_n \to \infty$. We want to show that
    \begin{align}
        \lim_{n \to\infty} F(x_n) = 1
    \end{align}
    This is equivalent to showing that $F(x \to \infty) \to 1$.
    \begin{proof}
        Define $A_n = \{ R \le x_n \}$. Then,
        \begin{align}
            F(x_n) = \probability(R \le x_n) = \probability(A_n)
        \end{align}
        But, this is an expanding sequence because $A_1 \subset A_2 \subset \cdots$. Thus, applying Proposition 7.2,
        \begin{align}
            F\qty(x_n \ce{->[n \to \infty]} \infty) = \probability\qty(\bigcup_{i=1}^\infty A_i) = \probability(\Omega) = 1
        \end{align}
    \end{proof}
\end{enumerate}

    \section{October 9, 2024}

\subsection{Properties of Distribution Functions}
Take some $\psp$ probability space, $R$ random variable, and $F(x) = \probability(R \le x)$ distribution function. Recall Lemma 7.2,
\begin{align}
    A_1 \subset A_2 \subset \cdots \implies \lim_{n \to \infty} \probability(A_n) = \probability\qty(\bigcup_{n=1}^\infty A_n)
\end{align}
and a similar Lemma for contracting sets. Now, we can continue proving some properties
\begin{enumerate}
    \item $F$ non decreasing. Proven last time.
    \item $\lim_{x \to\infty} F(x) = 1$. Proven last time.
    \item $\lim_{x \to{-\infty}} F(x) = 0$
    \begin{proof}
        Let $x_n \to -\infty$. For every sequence of real numbers tending to $-\infty$,
        \begin{align}
            \lim_{n \to \infty} F(x_n) = 0
        \end{align}
        Then, define $A_n \equiv \{ R \le x_n \}$. Then, the sets $A_n$ are contracting, and we can apply the Lemma.
        \begin{align}
            \lim_{n \to \infty} \probability(A_n) = \probability\qty(\bigcap_{n=1}^\infty A_n)
        \end{align}
        Because $A_n$ are contracting as $x \to -\infty$, this means $A_n \to \emptyset$, and therefore
        \begin{align}
            \lim_{n \to \infty} F(x_n) \to F(\emptyset) = 0
        \end{align}
    \end{proof}
    \item Limit probability from the right
    \begin{align}
        \lim_{x \to x_0^+} F(x) = F(x_0)
    \end{align}
    \begin{proof}    
        Let $x_n$ be a monotonically decreasing sequence of real numbers such that $x_n > x_0$ but $x_n \to x_0$. Let
        \begin{align}
            A_n = \{ R \le x_n \}
        \end{align}
        The sets $A_n$ are contracting because $x_n$ are decreasing on $x \to x_0^+$. Then,
        \begin{align}
            \lim_{n \to \infty} F(x_n) = \lim_{n \to \infty} \probability(A_n) = \probability\qty(\bigcap_{n=1}^\infty A_n)
        \end{align}
        The intersection of all of these is $(-\infty, x_0)$. If $\omega \in A_n$ for all $n$, that implies $R(\omega) \le x_n$ for all $n$. As $x_n \to x_0$, this means
        \begin{align}
            R(\omega) \le x_0
        \end{align}
        for all $\omega$. Conversely,
    \end{proof}
    \item Limit probability from the left
    \begin{align}
        \lim_{x \to x_0^-} F(x) = \probability(R < x_0)
    \end{align}
    (this is not the same as $\le x_0$).
    \begin{proof}
        The proof is pragmatically the same as above. Let $x_n$ be a monotonically increasing sequence. Then, set $A_n$ as an increasing sequence. As $n\to\infty$, $F(x_n) \to \probability(R < x_0)$.
        \begin{align}
            \lim_{n \to \infty}\probability(A_n) = \probability\qty(\bigcup_{n=1}^\infty A_n)
        \end{align}
        We want to show that
        \begin{align}
            \bigcup_n A_n \equiv \{ R < x_0 \}
        \end{align}
        For this to be true, pick some $\omega \in \bigcup_n A_n$; that means $\omega$ must be in any of them. All $A_n$ are some $\{ R < x_n \}$. But, $x_n \to x_0$ implies that $\omega < x_0 \implies \omega \in A_n$ for some $n$. Since the limit $x_n$ does not equal $x_0$, this effectively means
        \begin{align}
            \bigcup_n A_n \equiv \{ R < x_0 \}
        \end{align}
        This can be done in reverse too to complete the $\iff$ proof, but that proof is quite trivial.
    \end{proof}
    \item Probability of equality
    \begin{align}
        \probability(R = x_0) = F(x_0^+) - F(x_0^-)
    \end{align}
    \begin{proof}
        Trivially,
        \begin{align}
            \probability(R = x_0) = \probability(R \le x_0) - \probability(R < x_0)
        \end{align}
        By previous properties, this equation is pretty obviously equal to Equation 8.13.
    \end{proof}
\end{enumerate}

\begin{example}
    Pick some discrete probability space where $R$ is the value of a fair die. Then, the probability
    \begin{align}
        \probability(R = 2) = \probability(R \le 2) - \probability(R < 2) = \frac{2}{6} - \frac{1}{6} = \frac{1}{6}
    \end{align}
    and the rest of the distribution looks like a staircase for obvious reasons. This also satisfies the other properties, which is trivial.
\end{example}

\subsection{Joint Density Functions}
\begin{definition}
    Take some probability space $\psp$ and random variables $R_1, R_2$. We say the pair $(R_1, R_2)$ is absolutely continuous if there exists some integrable function $f_{12}(x,y) > 0$ such that the joint distribution function
    \begin{align}
        F(x,y) = \probability(R_1 \le x \cap R_2 \le y) = \iint_{(-\infty,-\infty)}^{(x,y)} f_{12}(x,y) \dd{A}
    \end{align}
    Then, $f_{12}$ is called the density of the pair $(R_1, R_2)$ or the joint density of $R_1$ \textbf{and} $R_2$.    
\end{definition}

\subsubsection{Borel Sets Tangent}
Borel sets can be done on $\real^2$.
\begin{definition}
    On $\real^2$, the Borel $\sigma$-algebra $\mathcal{B}$ is the $\sigma$-algebra of subsets of $\real^2$ generated by rectangles
    \begin{align}
        [x_1, x_2] \times [y_1, y_2]
    \end{align}
    where rectangle bounds can also have open-interval bounds.
\end{definition}
This definition can be generalized to $\real^n$.
\begin{proposition}
    Take $\Omega \equiv \real^n$. If $f$ is non-negative and integrable on $\real^n$ such that
    \begin{align}
        \int_{\real^n} f(\vb{x}) \dd{\vb{x}} = 1
    \end{align}
    then if $\curlyf \equiv \mathcal{B}^n$, then there exists a unique probability measure $\probability$ on $\mathcal{B}^n$ such that
    \begin{align}
        \probability([x_1^i, x_1^f] \times \cdots \times [x_n^i, x_n^f]) = \int_{\real^n} f(\vb{x}) \dd{\vb{x}}
    \end{align}
\end{proposition}

\begin{lemma}
    A ball is an element of $\mathcal{B}^3$; a disk is an element of $\mathcal{B}^2$.
\end{lemma}

There were a few more examples, but they did not say anything new.\footnote{``Once you see some examples it becomes more tractable'' but it already is???}

\begin{example}
    Let $f(x,y)$, $R_1 \equiv x$, $R_2 \equiv y$. Set
    \begin{align}
        f(x,y) = \begin{cases}
            1 & 0 \le x \le 1 \cup 0 \le y \le 1\\
            0 & \text{else}
        \end{cases}
    \end{align}
    Let $R_1, R_2$ have joint density $f$. Find
    \begin{align}
        \probability(2 R_1 \le R_2)
    \end{align}
    That just equals
    \begin{align}
        \probability(2x \le y)
    \end{align}
    and finding this can be done with a very simple double integral (or just finding the area of a triangle).
\end{example}
    \section{October 11, 2024}

\subsection{Recap: Joint Distribution Functions}
Recall that for $R_1,R_2$ random variables, we say that a pair $(R_1,R_2)$ is absolutely continuous if there exists a joint density function $f_{12}(x,y)$ such that
\begin{align}
    F(x,y) \equiv \probability(R_1 \le x \cap R_2 \le y) = \iint_{\real^2}^{[x,y]} f_{12}(\vb{x}) \dd{\vb{x}}
\end{align}
This can be rather trivially extrapolated to $\real^N$.
\begin{example}
    Suppose $(R_1,R_2)$ has density
    \begin{align}
        f_{12}(x,y) = \begin{cases}
            2e^{-x - 2y} & 0 < x,y < \infty\\
            0 & \text{else}
        \end{cases}
    \end{align}
    Find $\probability(R_1 > 1 \cap R_2 < 1)$.
\end{example}
\begin{solution}
    We can simply integrate the density over this space
    \begin{align}
    \hat{x} = [1,\infty) \cap \hat{y} = (0,1]    
    \end{align}
    noting that $y < 0$ has density $0$. Thus, we integrate over $\hat{x} \times \hat{y}$
    \begin{align}
        \probability = \iint_{\hat{x} \times \hat{y}} 2e^{-x - 2y} \dd{\qty(\hat{x} \times \hat{y})}
    \end{align}
    Pretty trivial from here.
\end{solution}
Again, this can be trivially extrapolated to $N$ random variables. Treat
\begin{align}
    \vb{R} &= (R_1, \ldots, R_N)\\
    \vb{x} &= (x_1, \ldots, x_N)
\end{align}
as a vector of random variables. Then,
\begin{align}
    F_{12\cdots N}(\vb{x}) \iint_{\real^N} f_{12\cdots N} \dd{\vb{x}} \equiv \probability\left[ R_1 \le x_1, \ldots, R_N \le x_N \right]
\end{align}
\begin{proposition}
    Given $F_{12}$, we can find $f_{12}$ (assuming continuous) by differentiating
    \begin{align}
        \pdv{x} \pdv{y} F_{12}
    \end{align}
    Recall that taking partial derivatives is a commutative operation.
\end{proposition}

\subsection{Joint vs Individual Density Functions}
\begin{proposition}
    If $(R_1,R_2)$ is a absolutely continuous pair, then $R_1$ and $R_2$ are absolutely continuous.
\end{proposition}
\begin{proof}
    Let $(R_1, R_2)$ have density $f_{12}(x,y)$. Without loss of generality,
    \begin{align}
        F_1(x) \equiv \probability(R_1 \le x)
    \end{align}
    But this just $\probability(R_1 \le x \cap R_2 \in \real)$, which just equals
    \begin{align}
        \int_{-\infty}^x \int_\real f_{12} \dd{v}\dd{u}
    \end{align}
    The inside can be written as
    \begin{align}
        f_1(u) \equiv \int_\real f_{12} \dd{v}
    \end{align}
    Thus, $R_1$ (and symmetrically, $R_2$) are absolutely continuous.
\end{proof}

\begin{example}
    Suppose $(R_1,R_2)$ has density
    \begin{align}
        f_{12}(x,y) = \begin{cases}
            x + y & 0 \le x,y \le 1\\
            0 & \text{else}
        \end{cases}
    \end{align}
    Then computing $f_1$ and $f_2$ are quite trivial using Proposition 9.3.
\end{example}

\begin{proposition}
    The converse of Proposition 9.3 is \textit{not} true. Given absolutely continuous random variables $R_1,R_2$, it is not necessarily true that the pair $(R_1,R_2)$ is absolutely continuous. Moreover, there is not necessarily a unique $f_{12}$ given $f_1,f_2$.
    \begin{proof}
        There can be integration constants that make the map $R_1,R_2 \to (R_1,R_2)$ not injective.
    \end{proof}
\end{proposition}

\subsection{Independent Random Variables}
Recall that events $A,B$ are independent if
\begin{align}
    \probability(A \cap B) = \probability(A) \probability(B)
\end{align}
\begin{definition}
    Take $R_1,R_2$ random variables. We say $R_1$ and $R_2$ are independent if
    \begin{align}
        \probability(R_1 \in B_1, R_2 \in B_2) = \probability(R_1 \in B_1) \probability(R_2 \in B_2)
    \end{align}
    where $B_1, B_2 \in \mathcal{B}$ are Borel sets.
\end{definition}
\noindent This can be extrapolated to a family of random variables, even uncountably many.
\begin{definition}
    A family $\mathcal{R} = \{ R_i \}_{i \in I}$ (where $I$ is some index set) is independent if for every finite subset $\{ R_{i_1}, \ldots, R_{i_k} \} \subset \mathcal{R}$,
    \begin{align}
        \probability(R_{i_1} \in B_{i_1}, \ldots, R_{i_k} \in B_{i_k}) = \prod_{j} \probability(R_{i_j} \in B_{i_j})
    \end{align}
    for all $B_{\ldots} \in \mathcal{B}$.
\end{definition}

\begin{proposition}
    Let $R_1, \ldots, R_n$ be independent and individually absolutely continuous with densities $f_1, \ldots, f_n$. Then, the joint random variable $(R_1, \ldots, R_n)$ is absolutely continuous with density
    \begin{align}
        f_{1\cdots n} \equiv \prod_i^n f_i(x_i)
    \end{align}
\end{proposition}
\begin{proof}
    By definition
    \begin{align}
        F_{12 \cdots n}(x_1, \ldots, x_n) = \probability(R_1 \le x_1 \cap \cdots \cap R_n \le x_n)
    \end{align}
    Independence tells us this equals
    \begin{align}
        F_{12 \cdots n}(x_1, \ldots, x_n) = \probability(R_1 \le x_1) \cdots \probability(R_n \le x_n)
    \end{align}
    and those can trivially be written as
    \begin{align}
        F_{12 \cdots n}(x_1, \ldots, x_n) = F_1(x_1) \cdots F_n(x_n)
    \end{align}
    Then,
    \begin{align}
        \pdv{x_1}\cdots\pdv{x_n} F_{12 \cdots n} = f_{12 \cdots n} = f_1(x_1) \cdots f_n(x_n)
    \end{align}
    because when taking $\pdv{x_j}$, all $f_{k \ne j}$ are treated as constants.
\end{proof}

\subsection{Functions of a Random Variable}
\begin{definition}
    A function
    \begin{align}
        g : \real \to \real
    \end{align}
    is Borel measurable if the inverse image
    \begin{align}
        \forall B \in \mathcal{B}, g^{-1}(B) \in \mathcal{B}
    \end{align}
\end{definition}

\begin{proposition}
    Let $R$ be a random variable and $g: \real \to \real$ be Borel measurable. Then, $g(R)$ is a random variable.
\end{proposition}
\begin{proof}
    Recall a random variable on $\psp$ is some map
    \begin{align}
        R : \Omega \to \real
    \end{align}
    such that for all $B \in \mathcal{B}$
    \begin{align}
        R^{-1}(B) \in \curlyf
    \end{align}
    Let $B \in \mathcal{B}$. Write
    \begin{align}
        \widetilde{R} \equiv g(R) : \Omega \to \real
    \end{align}
    Then,
    \begin{align}
        \widetilde{R}^{-1}(B) &= \{ \omega \in \Omega \mid g(R(\omega)) \in B \}\\
        &= \{ \omega \in \Omega \mid R(\omega) \in g^{-1}(B) \}\\
        &= R^{-1}(g^{-1}(B)) \in \curlyf
    \end{align}
\end{proof}

\begin{proposition}
    Let $R_1, \ldots, R_n$ be independent random variables and $g_1, \ldots, g_n: \real \to \real$ be Borel measurable functions\footnote{anything continuous is Borel measurable}. Then,
    \begin{align}
        g_1(R_1), \ldots, g_n(R_n)
    \end{align}
    are independent, i.e. functions of independent random variables are independent.
\end{proposition}

\begin{proof}
    Let $\widetilde{R_i} \equiv g_i(R_i)$. Let $B_1, \ldots B_n \in \mathcal{B}$. Then
    \begin{align}
        \probability(\widetilde{R_1} \in B_1, \ldots, \widetilde{R_n} \in B_n) &= \probability(g_1(R_1) \in B_1, \ldots, g_n(R_n) \in B_n)\\
        &= \probability(R_1 \in g_1^{-1}(B_1), \ldots, R_n \in g_n^{-1}(B_n))
    \end{align}
    By independence of $\{ R_i \}$, this equals
    \begin{align}
        &= \probability(R_1 \in g_1^{-1}(B_1)) \cdots \probability(R_n \in g_n^{-1}(B_n))\\
        &= \probability(g_1(R_1) \in B_1) \cdots \probability(g_n(R_n) \in B_n)
    \end{align}
    which means $\widetilde{R_1}, \ldots, \widetilde{R_n}$ are independent.\footnote{a LiTtLe BiT ThEoReTiCaL ToDaY}
\end{proof}

    \section{October 14, 2024}

\subsection{Functions of More Than One Random Variable}
\begin{example}
    Take $R_1,R_2$ as independent, uniformly distributed random variables between $0$ and $1$. Find the density of $R_3 = \frac{R_2}{R_1^2}$
\end{example}
\begin{solution}
    The density of the pair $(R_1,R_2)$ is
    \begin{align}
        f_{12}(x,y) = \begin{cases}
            1 & 0 \le x,y \le 1\\
            0 & \text{else}
        \end{cases}
    \end{align}
    which is the uniform square $[0,1] \times [0,1]$. We can compute the distribution function for $R_3$:
    \begin{align}
        F_3(z) = \probability(R_3 \le z)
    \end{align}
    This just equals
    \begin{align}
        F_3(z) = \probability\qty(R_2 \le (R_1)^2z)
    \end{align}
    Note that we do not care about $R_1 = 0$ because it has zero probability. This equals
    \begin{align}
        \iint_{y \le x^2 z} f_{12}(x,y) \dd{x}\dd{y}
    \end{align}
    We can take $z \ge 0$ because negative quotients are not possible. Then, the cases $z \in [0,1]$ and $z > 1$ are separate in evaluating this double integral.
    \begin{enumerate}
        \item Take $z \in [0,1]$. Then, we are integrating
        \begin{align}
            \int_0^1 x^2 z \dd{x} = \frac{z}{3}
        \end{align}
        \item Take $z > 1$. Then, we are integrating
        \begin{align}
            \int_0^{\sqrt{1/z}} x^2 z \dd{z} + \int_{\sqrt{1/z}}^1 1 \dd{z}
        \end{align}
        which equals
        \begin{align}
            \int_0^{\sqrt{1/z}} x^2 z \dd{z} + \qty(1 - \sqrt{1/z})
        \end{align}
        which trivially equals
        \begin{align}
            1 - \frac{2}{3\sqrt{z}}
        \end{align}
    \end{enumerate}
    Thus, the distribution function equals
    \begin{align}
        F_3(z) = \begin{cases}
            0 & z < 0\\
            \frac{z}{3} & z \in [0,1]\\
            1 - \frac{2}{3\sqrt{z}} & z > 1
        \end{cases}
    \end{align}
    The density is the derivative
    \begin{align}
        f_3(z) = \begin{cases}
            0 & z < 0\\
            \frac{1}{3} & z \in [0,1]\\
            \frac{1}{3}z^{-3/2} & z > 1
        \end{cases}
    \end{align}
\end{solution}

\begin{definition}
    A random variable $R_1$ with density
    \begin{align}
        f(z) = \frac{1}{\sqrt{2\pi}\sigma}e^{-\frac{(x-\mu)^2}{2\sigma^2}}
    \end{align}
    where $\mu$ is the mean and $\sigma > 0$ is the standard deviation of a Gaussian/normal distribution.
\end{definition}
Everybody knows what a normal distribution, its mean, and its standard deviation are.

\begin{example}
    Let $R_1,R_2$ be independent normal distributions with $\mu = 0, \sigma = 1$. Find the density of $R_3 = \sqrt{R_1^2 + R_2^2}$
\end{example}
\begin{solution}
    For $z > 0$,
    \begin{align}
        F_3(z) = \probability(R_3 \le z)
    \end{align}
    This is the probability that $R_1^2 + R_2^2 \le z^2$. The joint density of $(R_1,R_2)$ is the product, i.e.
    \begin{align}
        f_{12} = \frac{1}{2\pi} e^{\frac{-(x^2+y^2)}{2}}
    \end{align}
    Then,
    \begin{align}
        F_3(z) &= \iint_{x^2+y^2 \le z^2} \frac{1}{2\pi} e^{\frac{-(x^2+y^2)}{2}} \dd{x}\dd{y}\\
        &= \int_0^{2\pi} \int_0^z \frac{1}{2\pi} e^{\frac{-r^2}{2}} r\dd{r}\dd{\theta}\\
        &= \int_0^z e^{\frac{-r^2}{2}} r\dd{r}
    \end{align}
    It's pretty trivial to compute this, but we want the density. The inside of this integral \textit{is} the density.
\end{solution}

\subsection{Poisson Distribution}
Recall the Binomial distribution. Suppose we have $n$ Bernoulli trials, for each of which $p$ is the probability of success. Let $R$ be the discrete random variable where $R$ is the number of successes after $n$ trials. Obviously,
\begin{align}
    \probability_R(k) = \binom{n}{k} p^k (1-p)^{n-k}
\end{align}
What happens when we have many, many trials? Suppose $n$ is very large and $p$ is very small. Also assume that $np \to \lambda$ where $\lambda > 0$ is some positive constant. ($p \sim \lambda/n$)
\begin{proposition}
    This distribution converges to the discrete \textbf{Poisson Distribution} with probability function
    \begin{align}
        p(k) = e^{-\lambda} \frac{\lambda^k}{k!}
    \end{align}
\end{proposition}
\begin{proof}
    Write $R_n$ for this random variable. Using
    \begin{align}
        \probability(R_n = k) = \binom{n}{k}p^k (1-p)^{n-k}
    \end{align}
    As $n \to \infty$,
    \begin{align}
        \probability(R_n = k) &= \frac{n(n-1)\cdots(n-k+1)}{k!} (np)^k \frac{1}{n^k} \qty(1 - \frac{np}{n})^{n-k}\\
        &= \frac{1(1-1/n)\cdots(1-(k-1)/n)}{k!} \lambda^k \qty(1 - \frac{np}{n})^n \qty(1 - \frac{np}{n})^{-k}\\
        &= \frac{1}{k!} \lambda^k \qty(1 - \frac{np}{n})^n \qty(1 - \frac{np}{n})^{-k}
    \end{align}
    But, $\qty(1 - \frac{np}{n})^{-k} \to 1$ because $np \to \lambda$, and $\lim_{n\to\infty}\qty(1 + \frac{x}{n})^n = e^x$. Thus, our $R_n$ approaches
    \begin{align}
        e^{-\lambda} \frac{\lambda^k}{k!}
    \end{align}
\end{proof}

\begin{proposition}
    The Poisson Distribution is well defined, i.e.
    \begin{align}
        \sum_0^\infty e^{-\lambda} \frac{\lambda^k}{k!} \dd{k} = 1
    \end{align}
\end{proposition}
\begin{proof}
    This just equals
    \begin{align}
        e^{-\lambda} \sum_0^\infty \frac{\lambda^k}{k!}
    \end{align}
    which equals
    \begin{align}
        e^{-\lambda}e^{\lambda}
    \end{align}
    which is obviously $1$.
\end{proof}
\begin{example}
    Poisson Distributions can model the number of typos on the page of a printed book.
\end{example}
\begin{example}
    Poisson Distributions can model the number of radioactive emissions / alpha particles in an hour.
\end{example}

\subsubsection{Sum of Poisson Distributions}
\begin{proposition}
    Let $R_1,R_2$ be two Poisson distributions with parameters $\lambda_1,\lambda_2$. Then, the random variable
    \begin{align}
        R' = R_1 + R_2
    \end{align}
    is a Poisson distribution with parameter $\lambda' = \lambda_1+\lambda_2$.
\end{proposition}
\begin{lemma}
    Let $R_1,R_2, \ldots, R_n$ be independent, discrete random variables with probability functions $P_1, P_2, \ldots, P_n$. Write $P_{12\cdots n}$ for the joint probability function of the $n$ random variables, i.e.
    \begin{align}
        P_{12\cdots n}(x_1, x_2, \ldots, x_n) = \probability(R_1 = x_1, \ldots)
    \end{align}
    Then, $R_1, R_2, \ldots, R_n$ are independent if and only if
    \begin{align}
        \probability(R_1 = x_1, \ldots, R_n = x_n) = \probability(R_1 = x_2) \cdots \probability(R_n = x_n)
    \end{align}
\end{lemma}
\begin{proof}[Proof of Proposition]
    The joint probability function $f_{12}(i,j)$ equals
    \begin{align}
        f_{12}(i,j) &= \probability(R_1 = i \qand R_2 = j)\\
        &= \probability(R_1 = i) \cdot \probability(R_2 = j)
    \end{align}
    That is just the product of the two Poisson distributions
    \begin{align}
        e^{-\lambda_1} \frac{\lambda_1^k}{k!} e^{-\lambda_2} \frac{\lambda_2^k}{k!}
    \end{align}
    Then,
    \begin{align}
        \probability(R_1 + R_2 = k) = \sum_{i = 0}^k \probability(R_1 = i)\probability(R_2 = k - i)
    \end{align}
    which equals
    \begin{align}
        \sum_{i = 0}^k \frac{1}{i!}\lambda_1^i e^{-\lambda_1} \frac{1}{(k-i)!} \lambda_2^{k-i} e^{-\lambda_2}
    \end{align}
    After multiplying by $k!/k!$, we get
    \begin{align}
        \sum_{i = 0}^k \binom{k}{i} \frac{1}{k!} \lambda_1^i \lambda_2^{k-i} e^{-(\lambda_1 + \lambda_2)}
    \end{align}
    which simplifies into
    \begin{align}
        \frac{(\lambda_1 + \lambda_2)^k}{k!} e^{-(\lambda_1 + \lambda_2)}
    \end{align}
    which is clearly a Poisson Distribution with parameter $\lambda_1 + \lambda_2$.
\end{proof}

    \section{October 16, 2024}
`new' stuff lol

\subsection{Expectation}

\subsubsection{Discrete Expectation}

Throw a fair six-sided die. Let $R$ be the result. What is the expected value of $R$? Obviously, in this case, $R = 3.5$. What if $R$ is any \textit{simple} random variable?

\begin{definition}
    A \textbf{simple random variable} takes at most \textit{finitely} many values, such as rolling a dice with $6$ possible values.
\end{definition}

\begin{definition}
    Let $R$ be a simple random variable. Then, the expectation of $R$ equals
    \begin{align}
        E(R) = \sum_x x \cdot \probability(x)
    \end{align}
    the weighted average of the probabilities.
\end{definition}

\begin{example}
    Take a biased coin
    \begin{align}
        \probability(\text{heads}) = \frac{3}{4} && \probability(\text{tails}) = \frac{1}{4}
    \end{align}
    and flip it twice. Define $R \equiv \text{number of heads}$. What is $E(R)$?
\end{example}
\begin{solution}
    $E(R)$ equals
    \begin{align}
        \sum_x x \cdot \probability(x)
    \end{align}
    which equals
    \begin{align}
        0 \cdot \probability(0) + 1 \cdot (2(3/4)(1/4)) + 2 \cdot (9/16) = \frac{3}{2}
    \end{align}
\end{solution}

\begin{definition}
    Let $R$ be a simple random variable, and let
    \begin{align}
        g: \real \to \real
    \end{align}
    Then, the expectation of this is
    \begin{align}
        E(g(R)) = \sum_x g(x) \cdot \probability(x)
    \end{align}
\end{definition}

\begin{definition}
    Let $R$ be a discrete random variable (it could have countably infinitely many values). Then, (again)
    \begin{align}
        E(g(R)) = \sum_x g(x) \probability(x)
    \end{align}
    This could be an infinite sum, which indicates a possibility of divergence. Thus, we define Equation 11.7 as long as
    \begin{enumerate}
        \item $g \ge 0$ (in this case, $E(R)$ could diverge)
        \item or the sum is absolutely convergent
    \end{enumerate}
\end{definition}

\subsubsection{Absolutely Continuous Expectation}

\begin{definition}
    Let $R$ be an absolutely continuous random variable with density $f_R(x)$. Then, define the expectation
    \begin{align}
        E(R) = \int_\real x \cdot f_R(x) \dd{x}
    \end{align}
    Additionally, if $g$ is a Borel function, define
    \begin{align}
        E(g(R)) = \int_\real g(x) f_R(x) \dd{x}
    \end{align}
    as long as
    \begin{enumerate}
        \item $g(x) \ge 0$
        \item or $\int \ldots$ is absolutely convergent
    \end{enumerate}
\end{definition}

\begin{lemma}
    This above definition can be extended to finitely many random variables $R_1, \ldots, R_n$. Take $R_1, \ldots, R_n$ discrete. Then,
    \begin{align}
        E(g(R_1, \ldots, R_n)) = \sum_{x_1, \ldots, x_n} g(x_1, \ldots, x_n) \cdot \probability(R_1 = x_1, \ldots, R_n = x_n)
    \end{align}
    This same thing can be done to absolutely continuous random variables:
    \begin{align}
        E(g(R_1, \ldots, R_n)) = \int_{\real^n} g(x_1, \ldots, x_n) f_{12\cdots n}(x_1, \ldots x_n) \dd{\vb{x}_n}
    \end{align}
    where $g$ is some Borel function and we make similar assumptions as above.
\end{lemma}

\begin{example}
    Let $R$ be a random variable with density
    \begin{align}
        f_R(x) = \begin{cases}
            e^{-x} & x \ge 0\\
            0 & \text{else}
        \end{cases}
    \end{align}
    Find $E(R)$.
\end{example}
\begin{solution}
    Since $f$ is zero on $x < 0$, the expectation equals
    \begin{align}
        E(R) = \int_0^\infty x \cdot e^{-x} \dd{x}
    \end{align}
    which is a fairly trivial integration by parts. It equals
    \begin{align}
        E(R) = \cdots = 1
    \end{align}
\end{solution}

We were then shown \textit{another} example.
\begin{example}
    Let $R_1,R_2$ be independent random variables each with density
    \begin{align}
        f_R(x) = \begin{cases}
            e^{-x} & x \ge 0\\
            0 & \text{else}
        \end{cases}
    \end{align}
    Find $E(\max(R_1,R_2))$
\end{example}
\begin{solution}
    First, the joint density is just the product, i.e.
    \begin{align}
        f_{12}(x,y) = \begin{cases}
            e^{-x-y} & (x \ge 0) \cap (y \ge 0)\\
            0 & \text{else}
        \end{cases}
    \end{align}
    Using Equation 11.11, we get
    \begin{align}
        E(\max(R_1,R_2)) &= \iint_{\real^2_+} \max(x,y) \cdot e^{-x-y} \dd{\vb{x}_2}\\
        &= \int_0^\infty \int_0^\infty \max(x,y) \cdot e^{-x-y} \dd{x}\dd{y}
    \end{align}
    Using some inequalities, the $\max(\cdots)$ becomes
    \begin{align}
        \max(x,y) = \begin{cases}
            x & x \ge y\\
            y & x \le y
        \end{cases}
    \end{align}
    So we can write our integral as
    \begin{align}
        \underbracket{\int_0^\infty \int_0^x x \cdot e^{-x-y} \dd{y}\dd{x}}_{\circled{1}} + \underbracket{\int_0^\infty \int_x^\infty y \cdot e^{-x-y} \dd{y}\dd{x}}_{\circled{2}}
    \end{align}
    which can be quite trivially integrated by parts. The answer is
    \begin{align}
        E(\max(\cdots)) = \frac{3}{2}
    \end{align}
\end{solution}

\subsection{Moments}
\begin{definition}
    Let $R$ be a random variable. Then, for $k > 0$ ($k$ is not necessarily an integer). Then, the $k$-th moment of $R$ is defined as
    \begin{align}
        \alpha_k = E(R^k)
    \end{align}
    Note that
    \begin{align}
        \alpha_1 = E(R) = \mu && \text{mean}
    \end{align}
\end{definition}
Let $R$ be an absolutely continuous random variable with density $f(x)$. Consider the centroid of the mass defined by $0 \le y \le f(x)$ (the ``center of mass''), located at $(x_c, y_c)$. Then
\begin{align}
    x_c = \alpha_1(R) = E(R) = \mu_R
\end{align}
Given $N$ points
\begin{align}
    (x_1, y_1), \cdots, (x_N, y_N)
\end{align}
the centroid is the point 
\begin{align}
    \frac{1}{N} \sum_{i=1}^N (x_i, y_i)
\end{align}
The same thing is true for infinitely many points:
\begin{align}
    \frac{\iint (x,y) \dd{x}\dd{y}}{\iint (1,1) \dd{x}\dd{y}} && \text{odd notation}
\end{align}
In the case of some random mass (possibly a nose, which we can assume to be spherical), the denominator may be complicated and annoying. In the case of an absolutely continuous random variable, that denominator just equals $1$. So, for our random variable,
\begin{align}
    \iint (x,y) \dd{x}\dd{y}
\end{align}
In the specific case of a shaded region under some function $f(x)$, this is
\begin{align}
    \int_\real \int_0^{f(x)} (x,y) \dd{y}\dd{x}
\end{align}

\subsubsection{Central Moments}
\begin{definition}
    Let $R$ be a random variable; let $k > 0$ be positive. Then, the $k$-th central moment of $R$ is
    \begin{align}
        \beta_k = E\qty[(R - \mu)^k]
    \end{align}
    such that the mean is zero. In particular,
    \begin{align}
        \beta_1 = E(R - \mu) = 0
    \end{align}
    Then, $\beta_2$ is the variance, which is the square of standard deviation.
\end{definition}
\begin{example}
    Take $R = N(\cdots)$, so
    \begin{align}
        f_R(x) = \frac{1}{\sigma\sqrt{2\pi}} e^{-\frac{(x-\mu)^2}{2\sigma^2}}
    \end{align}
    Then, using the first moment and second central moment,
    \begin{align}
        \alpha_1 = \mu && \beta_2 = \sigma^2
    \end{align}
\end{example}
\begin{solution}
    I've done this twice already and will not do it a third time.
\end{solution}



    \section{October 18, 2024}

\subsection{Recap: Random Variables}
Recall that a random variable $R$ on some probability space $\psp$ is a function
\begin{align}
    R: \Omega \to \real
\end{align}
\begin{example}
    Flip one coin. Then
    \begin{align}
        R = \begin{cases}
            1 & \text{heads}\\
            0 & \text{tails}
        \end{cases}
    \end{align}
    Then, the probability distribution is
    \begin{align}
        F_R(x) = \probability(R \le x) = \begin{cases}
            0 & x < 0\\
            1/2 & x \in [0,1)\\
            1 & x \ge 1
        \end{cases}
    \end{align}
    This can also be thought of as $\Omega = \{ H, T \}$, and
    \begin{align}
        R(\Omega) = \{ H \to 1, T \to 0 \}
    \end{align}
\end{example}

\begin{example}
    Roll a dice. Let
    \begin{align}
        R = \begin{cases}
            1 & \text{even}\\
            0 & \text{odd}
        \end{cases}
    \end{align}
    Then, $R$ can also be expressed as
    \begin{align}
        R = \{ 1 \to 0, 2 \to 1, \cdots \}
    \end{align}
    The probability distribution is the same as in Equation 12.3.
\end{example}

\noindent Take $R$ as a random variable with a density $f(x)$. Note that the probability space here is not specified, because there are many random variables with the same density $f$. \textbf{Choose} the probability space $(\real, \mathcal{B}, \probability)$ such that
\begin{align}
    \probability(B \in \mathcal{B}) = \int_B f(x) \dd{x}
\end{align}
Then we can define
\begin{align}
    R: \real \to \real \equiv R(x) = x
\end{align}
Then the probability
\begin{align}
    \probability(R \le z) = \int_{x \le z} f(x) \dd{x}
\end{align}

\begin{aside}
    If a random variable has a density function, it is assumed to be absolutely continuous.
\end{aside}

Let $R_1, \ldots, R_n$ be independent, absolutely continuous random variables with densities $f_1(x), \ldots, f_n(x)$. Take the probability space $(\real^n, \mathcal{B}, \probability)$ where
\begin{align}
    \probability(B) = \int\cdots\int_B f_1(x_1)\cdots f_n(x_n) \dd{x_1}\cdots\dd{x_n}
\end{align}
Then,
\begin{align}
    R_1(x_1, \ldots, x_n) &= x_1\\
    R_n(x_1, \ldots, x_n) &= x_n
\end{align}

\subsection{Median}
\begin{definition}
    Let $R$ be a random variable with distribution function $F_R(x)$. Assume absolute continuity. Then, $F_R(x)$ is continuous. Then, the median is defined as
    \begin{align}
        m \in \real \mid f_R(m) = \frac{1}{2}
    \end{align}
    which is equivalent to
    \begin{align}
        \probability(R \le m) = \frac{1}{2} = \probability(R > m)
    \end{align}
\end{definition}

\noindent For general, not-necessarily-absolutely-continuous $R$,
\begin{definition}
    The median $m \in \real$ of a random variable $R$ satisfies
    \begin{align}
        F_R(m) \ge 1/2 \qsp F_R(\ce{$m$-}) \le 1/2
    \end{align}
\end{definition}
\begin{example}
    Take Example 12.1. The median is any number $m \in [0,1]$.
    \begin{proof}
        Let $x \in [0,1]$. Then,
        \begin{align}
            F_R(x) \ge 1/2 \qand F_R(x^-) \le 1/2 & \qedhere
        \end{align}
    \end{proof}
\end{example}


\subsection{Properties of Expectation}
\begin{aside}
    Assume expectations all exist and are finite, i.e. do not diverge.
\end{aside}
\begin{enumerate}
    \item Closure under addition. Let $R_1, \ldots, R_n$ be random variables. Then,
    \begin{align}
        E\qty(\sum_i R_i) = \sum_i E(R_i)
    \end{align}
    \item For $a \in \real$,
    \begin{align}
        E(aR) = a \cdot E(R)
    \end{align}
    \item If $R_1 \le R_2$ for all $\omega \in \Omega$, then $E(R_1) \le E(R_2)$.
    \item If $R \ge 0$ and $E(R) = 0$, then $\probability(R = 0) = 1$, i.e. $R$ is essentially zero.
    \begin{proof}
        It's in the textbook.
    \end{proof}
    \begin{aside}
        If for some random variable $R$, $\text{Var}(R) = 0$, then $R$ is essentially constant.
    \end{aside}
    \item Let $R_1, \ldots, R_n$ be independent random variables. Then,
    \begin{align}
        E(R_1\cdots R_n) = E(R_1) \cdots E(R_n)
    \end{align}
    \item Let $R$ be a random variable with mean $\mu$ and variance $\sigma^2$. Then,
    \begin{align}
        \text{var}(aR + b) = a^2\sigma^2
    \end{align}
    \begin{proof}
        It's in the textbook.
    \end{proof}
    \item Variance is closed under addition, i.e. for independent random variables $R_1, \ldots, R_n$,
    \begin{align}
        \text{var}\qty(\sum_i R_i) = \sum_i \text{var}\qty(R_i)
    \end{align}
    \begin{proof}
        It's in the textbook.
    \end{proof}
\end{enumerate}


    \section{October 23, 2024}

\subsection{Review of Moments}
Let $X$ be a random variable. Recall that
\begin{align}
    \alpha_k = E(X^k) && k > 0
\end{align}
is the $k$-th moment. It follows that $E(X) = \text{mean}(X)$, so recall that
\begin{align}
    \beta_k = E((X - m)^k) && k > 0
\end{align}
is the $k$-th central moment (for $m < \infty$).

\subsection{Properties of Expectation Continued}
\begin{enumerate}
    \item[8.] The central moments $\beta_1, \beta_2, \ldots$ of a random variable can be obtained from the moments $\alpha_1, \alpha_2, \ldots$, assuming $\alpha_i < \infty$ for $i < n$ and $\alpha_n$ exists. The result is
    \begin{align}
        \beta_n = E\qty[(R - m)^n] = E \qty[\sum_{k = 0}^n \binom{n}{k} (-m)^{n-k} R^k]
    \end{align}
    assuming $m < \infty$. This simplifies into
    \begin{align}
        \sum_{k=0}^n \binom{n}{k}(-m)^{n-k} \alpha_k
    \end{align}
    because $E(R^k) \equiv \alpha_k$. It follows that for $n = 2$,
    \begin{align}
        \text{var}(R) = E(R^2) - 2mE(R) + m^2 \implies \boxed{\sigma_R^2 = E(R^2) - \qty[E(R)]^2}
    \end{align}
    \item[9.] If $0 < j < k$, then
    \begin{align}
        E(\abs{R}^j) \le 1 + E(\abs{R}^k)
    \end{align}
    \begin{proof}
        For $\omega \in \Omega$,
        \begin{align}
            \abs{R(\omega)}^j \le \begin{cases}
                \abs{R(\omega)}^k & \abs{R(\omega)} \ge 1\\
                1 & \text{else}
            \end{cases}
        \end{align}
        i.e. finiteness is a logical consequence of $0 < j < k$. Thus,
        \begin{align}
            \abs{R(\omega)}^j \le 1 + \abs{R(\omega)}^k
        \end{align}
        for all $\omega \in \Omega$. This extends to the expectation:
        \begin{align}
            E(\abs{R}^j) \le 1 + E(\abs{R}^k)
        \end{align}
    \end{proof}
    It follows that if some higher order expectation is finite, then lower order expectations are also finite.
\end{enumerate}

\begin{example}
    Let $R \sim \exp(\lambda)$ be
    \begin{align}
        f_R(x) = \begin{cases}
            \lambda e^{-\lambda x} & x \ge 0\\
            0 & \text{else}
        \end{cases}
    \end{align}
    Find $\alpha_i$ and $\text{var}(R)$.
\end{example}
\begin{solution}
    By definition,
    \begin{align}
        \alpha_j = E\qty(R^j) = \int_0^\infty x^j \lambda e^{-\lambda x} \dd{x}
    \end{align}
    This integration could clearly be done by parts. But this can also be done through a substitution. Set $y = \lambda x$. Then, $\dd{y} = \lambda \dd{x}$. We get
    \begin{align}
        \dfrac{1}{\lambda^j} \cdot \int_0^\infty y^j \cdot e^{-y} \dd{y}
    \end{align}
    \begin{aside}
        Define the Gamma function
        \begin{align}
            \Gamma\qty(j) \equiv \int_0^\infty y^{j-1} \cdot e^{-y} \dd{y}
        \end{align}
        We can do some induction on $j$:
        \begin{align}
            \Gamma(1) = \int_0^\infty e^{-x} = 1
        \end{align}
        For $\Gamma(r)$,
        \begin{align}
            \int_0^\infty x^{r-1} e^{-x} \dd{x}
        \end{align}
        By parts, this becomes
        \begin{align}
            \cancel{\eval{\frac{x^r}{r} e^{-x}}_0^\infty} + \frac{1}{r} \int_0^\infty x^r e^{-x} \dd{x}
        \end{align}
        This indicates
        \begin{align}
            \Gamma(r) = \dfrac{1}{r} \Gamma(r+1)
        \end{align}
        Inductively, for $n \in \natural$, $\Gamma(n + 1) = n!$.
    \end{aside}
    It follows that Equation 13.12 simplifies into
    \begin{align}
        \alpha_j = \dfrac{j!}{\lambda^j}
    \end{align}
    So it follows that
    \begin{align}
        \text{var}\qty(\exp(\lambda)) = \alpha_2 - \alpha_1^2 = \dfrac{2}{\lambda^2} - \qty(\dfrac{1}{\lambda})^2 = \boxed{\dfrac{1}{\lambda^2}} & \qedhere
    \end{align}
\end{solution}

\subsection{Covariance}
With multiple random variables, we can define covariance for two variables at a time. Let $R_1,R_2$ be two random variables.
\begin{definition}
     The $(i,j)$-th joint moment for the pair of random variables is
    \begin{align}
        \alpha_{i,j} = E\qty[R_1^i \cdot R_2^j] && i,j \ge 1
    \end{align}
\end{definition}
Assume that the moments are finite, i.e. $m_1 = E(R_1) < \infty$ and $m_2 = E(R_2) < \infty$.
\begin{definition}
    The $(i,j)$-th joint central moment for the pair of random variables is
    \begin{align}
        \beta_{i,j} = E\qty[(R_1 - m_1)^i \cdot (R_2 - m_2)^j]
    \end{align}
\end{definition}
\begin{definition}
    The \textbf{covariance} of a pair of random variables $R_1,R_2$ is defined as
    \begin{align}
        \beta_{1,1} = \text{cov}(R_1, R_2) = E((R_1 - E(R_1)) (R_2 - E(R_2)))
    \end{align}
    Note that if $R_1 = R_2$, then
    \begin{align}
        \text{cov}(R_1,R_2) = \text{var}(R_1)
    \end{align}
\end{definition}
\begin{proposition}
    An easier way to compute covariance is given by
    \begin{align}
        \text{cov}(R_1,R_2) &= E(R_1R_2 - R_2m_1 - R_1m_2 + m_1m_2)\\
        &= E(R_1R_2) - m_1 \cdot E(R_2) - m_2 \cdot E(R_1) + m_1m_2\\
        &= E(R_1R_2) - m_1 m_2 - m_1 m_2 + m_1m_2\\
        &= E(R_1R_2) - m_1 m_2
    \end{align}
\end{proposition}

\subsection{Preview of Correlation}
Covariance is not normalized and can take any value $r \in \real$. However, that may be unintuitive, so we want a normalized correlation constant $R \in [-1,1]$
% \begin{aside}
%     If a plot looks like
%     % figure
%     then we say the two random variables are positively correlated and has positive covariance. Similarly,
%     % figure
%     are negatively correlated and has negative covariance. And,
%     % figure
%     are not correlated and has zero covariance.
% \end{aside}
\begin{proposition}
    If $R_1,R_2$ are independent, then $\text{cov}(R_1,R_2) = 0$ and the two variables are uncorrelated.
    \begin{proof}
        $R_1,R_2$ are independent, so
        \begin{align}
            E\qty[(R_1-m_1)(R_2-m_2)] &= E\qty[(R_1-m_1)] \cdot E\qty[(R_2-m_2)]\\
            &= 0 \cdot 0 = 0 && \qedhere
        \end{align}
    \end{proof}
    This is not an iff; functions may be uncorrelated, but not independent.
\end{proposition}
This will be completed in Lecture 14.

    \section{October 25, 2024}

\subsection{Cauchy Schwarz}
\begin{theorem}
    For an inner product space $V$, let $u,v\in V$; then
    \begin{align}
        \norm{\langle u, v \rangle} \le \langle u, u \rangle \cdot \langle v, v \rangle
    \end{align}
\end{theorem}
\begin{proof}
    Straightforward from the definition of inner product.
\end{proof}

\begin{proposition}
    For random variables $R_1,R_2$ (assume $E(R_1^2),E(R_2^2)$ are finite), then
    \begin{align}
        \qty[ E(R_1R_2) ]^2 \le E(R_1^2) \cdot E(R_2^2)
    \end{align}
    with equality if and only if there exist $a_1,a_2 \in \real \ne 0$ such that
    \begin{align}
        a_1R_1 + a_2R_2 = 0
    \end{align}
    i.e. $R_1$ and $R_2$ are linearly dependent, which is equivalent to saying that the probability that this linearly combined random variable equalling $0$ is $1$.
\end{proposition}
\begin{proof}
    Suppose then that $R_1,R_2$ are linearly dependent, which implies
    \begin{align}
        a_1R_1 + a_2R_2 = 0
    \end{align}
    Then,
    \begin{align}
        \qty[E(R_1R_2)]^2 &= \qty(E\qty(-\frac{a_2}{a_1}R_2^2))^2 = \qty(\frac{a_2}{a_1})^2 \qty(E\qty(R_2^2))^2\\
        E(R_1^2) \cdot E(R_2^2) &= E\qty(\qty(-\frac{a_2}{a_1}R_2^2)^2) \cdot E(R_2^2) = \qty(\frac{a_2}{a_1})^2 \qty(E\qty(R_2^2))^2
    \end{align}
    Now, define
    \begin{align}
        S = aR_1 - R_2
    \end{align}
    Then,
    \begin{align}
        0 &\le E(S^2) = E(a^2R_1^2 + R_2^2 - 2aR_1R_2)\\
        &\le a^2 E(R_1^2) + E(R_2^2) - 2aE(R_1R_2)
    \end{align}
    Pick $a = E(R_1R_2)/E(R_1^2)$. Then, plugging in $a$ and rearranging yields
    \begin{align}
        \qty[E(R_1R_2)]^2 \le E(R_1^2) \cdot E(R_2^2) & \qedhere
    \end{align}
    Tracing back, when Equation 14.7 is zero, $R_1$ and $R_2$ are linearly dependent, and $S$ is essentially zero.
\end{proof}
% This was ugly

\subsection{Correlation Coefficient}
\begin{definition}
    Recall that
    \begin{align}
        \text{cov}(R_1R_2) = E\qty((R_1 - E(R_1))(R_2 - E(R_2)))
    \end{align}
    Define the correlation coefficient of $R_1$ with $R_2$ to be
    \begin{align}
        \rho(R_1,R_2) = \text{corr}(R_1,R_2) \equiv \dfrac{\text{cov}(R_1,R_2)}{\sigma_1\sigma_2}
    \end{align}
    where $\sigma_1,\sigma_2 \in \real > 0$ are the standard deviation of $R_1$ and $R_2$, and $\text{cov}(\cdots)$ is assumed to exist.
\end{definition}

\begin{proposition}
    Assume $E(R_1^2),E(R_2^2) < \infty$. Then,
    \begin{align}
        \abs{\rho} \le 1
    \end{align}
    with equality if and only if $R_1 - E(R_1)$ and $R_2 - E(R_2)$ are linearly dependent.
\end{proposition}
\begin{proof}
    By definition,
    \begin{align}
        \qty(\text{cov}(R_1,R_2))^2 = \qty[E\qty((R_1 - E(R_1))(R_2 - E(R_2)))]^2
    \end{align}
    Cauchy Schwarz tells us that
    \begin{align*}
        \qty[E\qty((R_1 - E(R_1))(R_2 - E(R_2)))]^2 \le E\qty(R_1 - E(R_1)) \cdot E\qty(R_2 - E(R_2))
    \end{align*}
    Thus,
    \begin{align}
        \abs{\dfrac{\qty(\text{cov}(R_1,R_2))^2}{\sigma_1\sigma_2}} \le 1 & \qedhere
    \end{align}
    The equality holds when $R_1 - E(R_1)$ and $R_2 - E(R_2)$ are linearly dependent as a consequence of Proposition 14.2.
\end{proof}
    
\subsection{Method of Indicators}
Take some probability space $\psp$ and some $A \in \curlyf$.
\begin{definition}
    The \textit{indicator of $A$} is the discrete random variable $I_A$ given by
    \begin{align}
        I_A(\omega) = \begin{cases}
            1 & \omega \in A\\
            0 & \omega \notin A
        \end{cases}
    \end{align}
    which are analogous to characteristic functions in analysis.
\end{definition}
\begin{proposition}
    The expectation of an indicator is
    \begin{align}
        E(I_A) = \probability(I_A = 1) = \probability(A) = \probability\qty(\{ \omega \in \Omega \mid I_A(\omega) = 1 \})
    \end{align}
    The expectation of an indicator equals the probability of the event.
\end{proposition}

\begin{example}
    Take $R$ is the number of successes in $n$ Bernoulli trials where the probability of success is $p$. Find $E(R)$ and $\text{var}(R)$.
\end{example}
\begin{solution}
    Define $A_i$ as the event where the $i$-th trial is successful. Then,
    \begin{align}
        R = \sum_j\qty(I_{A_j}) = \text{number of successes}
    \end{align}
    The expectation of $R$ is the sum of expectations of each indicator, i.e.
    \begin{align}
        E(R) = \sum_j E(I_{A_j}) = \sum_j \probability(A_j) = np & \qedhere
    \end{align}
    The variance then equals
    \begin{align}
        E(R^2) - E(R)^2
    \end{align}
    which equals $E(R^2) - (np)^2$. That first term equals
    \begin{align}
        E\qty[\qty(I_{A_1} + \cdots I_{A_n})^2]
    \end{align}
    which equals
    \begin{align}
        E\qty[
            \sum_j I_{A_j}^2 + \sum_{j \ne k} I_{A_j}I_{A_k}
        ]   
    \end{align}
    \begin{aside}
        For random variable $A$,
        \begin{align}
            I_A^2 = I_A
        \end{align}
        \begin{proof}
            $1^2 = 1$ and $0^2 = 0$
        \end{proof}
    \end{aside}
    \begin{aside}
        If $A$ and $B$ are independent events, then $I_A$ and $I_B$ are independent.
    \end{aside}
    Thus, $E(R^2)$ becomes
    \begin{align}
        &= \sum_j E\qty[
            I_{A_j}
        ] + \sum_{j \ne k} E\qty[
            I_{A_j}I_{A_k}
        ]\\
        &= \sum_j E\qty[
            I_{A_j}
        ] + \sum_{j \ne k} E\qty[
            I_{A_j \cap A_k}
        ]\\
        &= np + (n^2 - n)p^2
    \end{align}
    Thus,
    \begin{align}
        \text{var}(R) = \qty[np + (n^2 - n)p^2] - (np)^2 = np(1-p)
    \end{align}
    \begin{example}
        Suppose $N$ people throw their hat into the middle of a room then randomly each pick a hat. Find $E(R)$ where $R$ is the number of people who pick their own hat.
    \end{example}
    \begin{solution}
        Define $A_i$ as the $i$-th person getting their own hat. Then,
        \begin{align}
            R = I_{A_1} + \cdots + I_{A_N}
        \end{align}
        Note that these are not independent events! Thus,
        \begin{align}
            E(R) = E(I_{A_1}) + \cdots + E(I_{A_N})
        \end{align}
        For each person,
        \begin{align}
            P(A_i) = 1/N
        \end{align}
        Then,
        \begin{align}
            E(R) = N \cdot 1/N = 1 & \qedhere
        \end{align}
    \end{solution}
\end{solution}




    \section{October 28, 2024}

\subsection{Chebyshev's Inequality}
Stating this theorem is harder than proving it.
\begin{theorem}
    \textbf{Chebyshev's Inequality}
    \begin{enumerate}
        \item[a.] $R \ge 0$ r.v.; $b > 0$ random number
        \begin{align}
            \probability(R \ge b) \le E(R)/b && \text{assume expectation exists}
        \end{align}
        \item[b.] $R$ r.v.; $c$ constant, $l, \varepsilon > 0$ constants
        \begin{align}
            \probability(\abs{R - c} \ge \varepsilon) \le \frac{E(\abs{R-c})^l}{\varepsilon^l}
        \end{align}
        \item[c.] $R$ r.v. finite mean $m$, finite variance $\sigma^2 > 0$, $k > 0$
        \begin{align}
            \probability(\abs{R-m} \ge k\sigma) \le 1/k^2
        \end{align}
    \end{enumerate}
\end{theorem}

\subsubsection{Proof of (a)}
First, assuming $R$ is absolutely continuous with density $f(x)$. Then,
\begin{align}
    E(R) = \int_{-\infty}^\infty x f(x) \dd{x}
\end{align}
Since $R \ge 0$, this is equivalent to
\begin{align}
    E(R) = \int_{0}^\infty x f(x) \dd{x}
\end{align}
We can shove an inequality at this
\begin{align}
    E(R) = \int_{0}^\infty x f(x) \dd{x}\ge \int_b^\infty x f(x) \dd{x}
\end{align}
Note that we are throwing away $\int_0^b$, so this is not a very sharp estimate. Note that within this integral, $x \ge b$, so this previous line is greater than
\begin{align}
    E(R) \ge \int_b^\infty x f(x) \dd{x} \ge b\int_b^\infty f(x) \dd{x} = b \probability(R \ge b)
\end{align}
Rearranging,
\begin{align}
    \probability(R \ge b) \le E(R)/b
\end{align}
For the general case, we claim
\begin{align}
    R \ge b I_{A_b}
\end{align}
where $A_b \equiv \{ R \ge b \} \subset \Omega$. Recall that $I$ is the indicator function, i.e.
\begin{align}
    I_{A_b}(\omega) = \begin{cases}
        1 & \omega \in A_b\\
        0 & \omega \notin A_b
    \end{cases}
\end{align}
To prove this, let $\omega \in \Omega$. We want $R(\omega) \ge b I_{A_b}(\omega)$. We can split into cases
\begin{enumerate}
    \item Case 1: $\omega \notin A_b$. Then, we want to show that $R(\omega) \ge 0$, which is true by assumption that $R \ge 0$.
    \item Case 2: $\omega \in A_b$. Then $b I_{A_b}(\omega) = b$. We need to show that $R(\omega) \ge b$. This is true by how $A_b$ is defined, i.e. as $\{ R \ge b \}$.
\end{enumerate}
This proves the claim, i.e. for all $\omega$, this statement is true:
\begin{align}
    R \ge b I_{A_b}
\end{align}
This implies that $E(R) \ge E(b I_{A_b})$. By linearity,
\begin{align}
    E(R) \ge b E\qty(I_{A_b}) = b\probability(A_b) = b \probability(R \ge b)
\end{align}
which can be rearranged to complete the proof for the general, non-absolutely-continuous case.
\qed

\subsubsection{Proof of (b)}
This follows from $A$. The probability
\begin{align}
    \probability(\abs{R-c} \ge \varepsilon) = \probability(\abs{R-c}^l \ge \varepsilon^l)
\end{align}
Then we apply (a) where $\abs{R - c}^l$ is our new, non-negative random variable and $\varepsilon^l$ is our number. Thus,
\begin{align}
    \probability(\abs{R-c}^l \ge \varepsilon^l) \le \frac{E\qty(\abs{R-c}^l)}{\varepsilon^l}
\end{align}
\qed

\subsubsection{Proof of (c)}
The probability $\probability(\abs{R-m} \ge k\sigma)$, by (b),
\begin{align}
    \probability(\abs{R-m} \ge k\sigma) \le \frac{E(\abs{R-m}^2)}{(k\sigma)^2} = \frac{\sigma^2}{k^2\sigma^2} = 1/k^2
\end{align}
\qed

\begin{aside}
    This is not a very sharp inequality. Take $R \sim \exp(1)$ (mean and variance are both $1$). By (c), (assume $k$ large)
    \begin{align}
        \probability(\abs{R-m} \ge k\sigma) \le 1/k^2
    \end{align}
    For our specific distribution, $m= \sigma = 1$, so we get 
    \begin{align}
        \probability(\abs{R - m} \ge k\sigma) = \probability(\abs{R - 1} \ge k) = \probability(R \ge k + 1) = e^{-(k+1)}
    \end{align}
    For $k$ large,
    \begin{align}
        e^{-(k+1)} \ll 1/k^2
    \end{align}
    so the inequality is not sharp, and quite wasteful.
\end{aside}


\subsection{Weak Law of Large Numbers}
\begin{theorem}
    Let $R_1, R_2, \ldots$ be a whole bunch of independent random variables on a given probability space. Assume the means and variances are finite, and shared (equal) across all $R_{1 \cdots N}$. And assume $\sigma_i^2 \le M$, i.e. the variances are bounded by some constant $M \in \real$ for all $i$. Define
    \begin{align}
        S \equiv \sum_{i} R_i
    \end{align}
    and let $n$ be the number of random variables being summed over. Then, for all $\varepsilon > 0$,
    \begin{align}
        \probability\qty[\frac{\abs{S - E(S)}}{n} \ge \varepsilon] \ce{->[$n \to \infty$]} 0
    \end{align}
    or
    \begin{align}
        \probability\qty[ \abs{\frac{R_1 + \cdots + R_n}{n} - m} \ge \varepsilon ] \ce{->[$n \to \infty$]} 0
    \end{align}
    i.e. the mean of the random variables tends towards the actual mean as $n \to \infty$
\end{theorem}
\begin{example}
    Flip $N$ coins. As $N \to \infty$, the expected number of heads is $N/2$. As $N \to \infty$, the probability of deviation from the expected value tends towards zero.
\end{example}
\begin{proof}[Solution]
    Obvious.
\end{proof}
Why is this law of large numbers \textit{weak}? Because it is stating that the probability of being greater than $\varepsilon$ goes to zero, as opposed to an actual probability going to zero.

\subsubsection{Proof}
Take
\begin{align}
    \probability\qty(\frac{\abs{S - E(S)}}{n} \ge \varepsilon)
\end{align}
By Part (b) of Chebyshev, we can write
\begin{align}
    \probability\qty(\frac{\abs{S - E(S)}}{n} \ge \varepsilon) &\le \frac{E\qty(\abs{\frac{S-E(S)}{n}}^2)}{\varepsilon^2}\\
    &\le \frac{1}{n^2\varepsilon^2} E\qty[(S - E(S))^2]\\
    &\le \frac{\text{var}(S)}{n^2\varepsilon^2}
\end{align}
By linearity on the independent variables,
\begin{align} 
    \probability\qty(\frac{\abs{S - E(S)}}{n} \ge \varepsilon)
    &\le \frac{\sum_i^n \text{var}(R_i)}{n^2\varepsilon^2}\\
    &\le \frac{Mn}{n^2\varepsilon^2} = \frac{M}{n\varepsilon^2}\\
    &\le 0\\
    \implies \probability\qty(\frac{\abs{S - E(S)}}{n} \ge \varepsilon) &= 0
\end{align}


\subsection{Conditional Probability}
Let $R_1,R_2$ be discrete r.v. with probability functions $P_{R_1}(x)$ and $P_{R_2}(x)$ which are `probability mass functions'. What is
\begin{align}
    \probability(R_1 = x \mid R_2 = y)
\end{align}
The probability $p(x \mid y)$ equals
\begin{align}
    \frac{\probability(R_1 = x, R_2 = y)}{\probability(R_2 = y)} = p_{12}(x,y) / p_2(y)
\end{align}
which is the quotient of the joint probability function and the individual probability function.
\begin{aside}
    For a given, fixed $y$, if $p_2(y) > 0$, then
    \begin{align}
        p(x \mid y)
    \end{align}
    is a probability function for $x$, i.e.
    \begin{align}
        \sum_x p(x \mid y) = 1
    \end{align}
    which is true by extending the law of total probability.
\end{aside}


    \section{October 30, 2024}

\subsection{...}
\subsection{...}
\subsection{...}
\subsection{...}



    \section{November 1, 2024}
This lecture is in a discrete $\varepsilon$-neighborhood of the set of all lectures contained on Midterm 2.

\subsection{Conditional Probabilities}
Recall the conditional probability for $R_1,R_2$ r.v.
\begin{align}
    \probability(R_2 = y \mid R_1 = x) = \frac{\probability(R_1 = x \cap R_2 = y)}{\probability(R_1 = x)}
\end{align}
assuming that $\probability(R_1 = x) \ne 0$. This can be worked into the language of (discrete) probability mass functions:
\begin{align}
    p(x,y) &= \probability(R_1 = x, R_2 = y)\\
    p_1(x) &= \probability(R_1 = x) = \sum_y p(x,y)
\end{align}
This yields the Theorem of Total Probability, stated in a previous section, that states that
\begin{align}
    \probability(R_2 \in B) = \sum_{x} \probability(R_2 \in B \mid R_1 = x) p_1(x)
\end{align}
which generalizes many statements from before. This also equals, for absolutely continuous r.v. $R_1,R_2$ with joint density $f(x,y)$,
\begin{align}
    h(y \mid x) = \frac{f(x,y)}{f_1(x)} \qsp f_1(x) = \int_\real f(x,y) \dd{y} %= f_{R_2 \mid R_1}(y \mid x)
\end{align}
which is the \textit{conditional density of $R_2$ given $R_1$}. For this case, the Theorem of Total Probability is
\begin{align}
    \probability(R_2 \in B) = \int_\real \probability(R_2 \in B \mid R_1 = x) f_1(x) \dd{x}
\end{align}

\subsubsection{Conditional Distribution Function}
The conditional distribution may be given by
\begin{align*}
    F_{R_2 \mid R_1}(y_0 \mid x) = \probability(R_2 \le y_0 \mid R_1 = x) = \int_{-\infty}^{y_0} h(y \mid x) \dd{y} \qsp \text{$x$ constant}
\end{align*}

\subsubsection{Examples}
\begin{enumerate}
    \item $(R_1,R_2)$ has joint density
    \begin{align}
        f(x,y) = \begin{cases}
            e^{-y/x - x}/x & 0 < x,y < \infty\\
            0 & \text{else}
        \end{cases}
    \end{align}
    Find $\probability(R_2 > 1 \mid R_1 = x) = \int_1^\infty h(y \mid x) \dd{y}$.
    \begin{solution}
        To find $h$,
        \begin{align}
            h(y \mid x) = \frac{f(x,y)}{\int_\real f(x,y) \dd{y}} = \cdots = e^{-y/x}/x
        \end{align}
        Then, $\probability(R_2 > 1 \mid R_1 = x) = \int_1^\infty h(y \mid x) \dd{y}$, which equals
        \begin{align}
            \probability(R_2 > 1 \mid R_1 = x) = \int_1^\infty e^{-y/x}/x \dd{x} = \cdots = e^{\qty(-\frac{1}{x})}
        \end{align}
    \end{solution}
    \item $R_1$ is the outcome $n$ of rolling a fair, six-sided dice. Given $R_1 = n$, define $R_2 \sim \exp(n)$, i.e. has density
    \begin{align}
        f_2(y) = \begin{cases}
            n \cdot e^{-n x} & x \ge 0\\
            0 & \text{else}
        \end{cases}
    \end{align}
    Find $\probability(R_2 \le y)$.
    \begin{solution}
        Using the Theorem of Total Probability, we get
        \begin{align}
            F_2(y) = \probability(R_2 \le y_0) = \sum_x \probability(R_2 \le y_0 \mid R_1 = x) \probability(R_1 = x)
        \end{align}
        for $x \in \{ 1, 2, \ldots, 6 \}$. So we have
        \begin{align}
            F_2(y_0) = \probability(R_2 \le y_0) &= \frac{1}{6} \sum_{n=1}^6 \probability(R_2 \le y_0 \mid R_1 = n)\\
            &= \frac{1}{6} \sum_{n=1}^6 \int_{-\infty}^{y_0} n e^{-n y} \dd{y}\\
            F_2(y) &= \frac{1}{6} \sum_{n=1}^6 \qty( 1 - e^{-ny} )
        \end{align}
        The density is just the derivative, i.e. $f_2(y) = \dv{y} F_2(y)$.
    \end{solution}
\end{enumerate}

\subsection{Conditional Expectation}
Recall that if $R$ is an absolutely continuous random variable with density $f(x)$, then the expectation of $R$ is
\begin{align}
    E(R) = \int_\real x f(x) \dd{x} \qsp E(g(R)) = \int_\real g(x) f(x) \dd{x}
\end{align}
\begin{definition}
    The conditional expectation of $R_2$ \textit{given} $R_1 = x$ is
    \begin{align}
        E(R_2 \mid R_1 = x) = \int_\real y h(y \mid x) \dd{y}
    \end{align}
    where $h(y \mid x)$ is the conditional density of $R_2$ given $R_1$.
\end{definition}
The same thing applies to the conditional expectation of a function on $R_2$, where
\begin{align}
    E(g(R_2) \mid R_1 = x) = \int_\real g(y) h(y \mid x) \dd{y}
\end{align}
Now, suppose $R$ is discrete; then, replace the integral by a sum:
\begin{align}
    E(R_2 \mid R_1 = x) &= \sum_y y p(y \mid x)\\
    E(g(R_2) \mid R_1 = x) &= \sum_y g(y) p(y \mid x)
\end{align}
\begin{example}
    Take $R_1 \sim \exp(1)$. Given $R_1 = x$, let $R_2$ be uniformly distributed on $[0,x]$. Find $E(R_2 \mid R_1 = x)$ and $E(R_2^4 \mid R_1 = x)$.
\end{example}
\begin{solution}
    \begin{align}
        h(y \mid x) = \begin{cases}
            1/x & 0 \le y \le x\\
            0 & \text{else}
        \end{cases}
    \end{align}
    Then, $E(R_2 \mid R_1 = x) = \int_\real y h(y \mid x) \dd{y} = \cdots = \frac{x}{2}$. For $R_2^4$, a similar process yields $\frac{x^4}{5}$. 
\end{solution}

\begin{example}
    A die is tossed $n$ times. Let $R_1$ be the number of $1$, and $R_2$ be the number of $2$. Find $E(R_2 \mid R_1 = k)$. This equals
    \begin{align}
        E(R_2 \mid R_1 = k) = \sum_y y \cdot p(y \mid k)
    \end{align}
\end{example}
\begin{solution}
    \begin{align}
        \probability(R_2 = l \mid R_1 = k) = \frac{\probability(R_2 = l \cap R_1 = k)}{\probability(R_1 = k)}
    \end{align}
    This is a multinomial distribution. Somewhere on the previous pages gives a formula for this. We get
    \begin{align}
        \probability(\cdots) &= \dfrac{\frac{n!}{k!l!(n-k-l)!}\qty(\frac{1}{6})^k \qty(\frac{1}{6})^l \qty(\frac{4}{6})^{n-k-l}}{\binom{n}{k} \cdot \qty(\frac{1}{6})^k \cdot \qty(\frac{5}{6})^{n-k}}\\
        &= \frac{(n-k)!}{l!(n-k-l)!} \qty(\frac{1}{6})^l \qty(\frac{4}{6})^{n-k-l} \qty(\frac{6}{5})^{n-k}
    \end{align}
    Then, the total probability equals
    \begin{align}
        \sum_{k=0}^{n-k} l \cdot \probability(\cdots)
    \end{align}
    which can be summed if enough time is spent.
\end{solution}

\begin{proof}[An easier way]
    If $k$ rolls are $1$, then $n-k$ rolls are not $1$. Then, the probability of getting $l$ rolls that are $2$ in these $n-k$ rolls is
    \begin{align}
        \binom{n-k}{l} \qty(\frac{1}{5})^l \qty(\frac{4}{5})^{n-k-l}
    \end{align}
    It turns out that these are the same.
\end{proof}

The expectation of this binomial distribution is $np$, where $n$ is the number of rolls and $p$ is the probability of success. In this case, given $k$ rolls that are $1$, this means the expectation is $(n-k) \cdot \frac{1}{5}$.


    \section{November 4, 2024}

\subsection{Conditional Expectation}
Recall for $R_1,R_2$ absolutely continuous r.v.,
\begin{align}
    E(R_2 \mid R_1 = x) = \int_\real y h(y \mid x) \dd{y}
\end{align}
where $h(y \mid x)$ equals the quotient of the densities
\begin{align}
    h(y \mid x) = f(x,y) / f_1(x)
\end{align}
In general, for some function on a random variable $g(R_2)$,
\begin{align}
    E(g(R_2) \mid R_1 = x) = \int_\real g(y) h(y \mid x) \dd{y}
\end{align}
\begin{proposition}
    We can do this with $N$ r.v., $R_1, \ldots, R_N$.
    \begin{align}
        E(g(R_{k+1}, \ldots, R_N) \mid R_1 = x_1, R_2 = x_2, \ldots, R_k = x_k)
    \end{align}
    would equal
    \begin{align}
        \int_S g(x_{k+1}, \ldots, x_N) h(x_{k+1}, \ldots, x_N \mid x_1, \ldots, x_k) \dd{S}
    \end{align}
    where $S \equiv x_{k+1} \times \cdots \times x_N$.
\end{proposition}
\begin{example}
    Suppose $R_0 \sim \exp(1)$. For $R_0 = \lambda$, define $R_1, \ldots, R_N$ as independent\footnote{i.i.d means independent and identically distributed, but this phraseology is not used in this text.} each with distribution $\exp(\lambda)$. We previously computed
    \begin{align}
        h(\lambda \mid x_1, \ldots, x_N) = \frac{1}{N!}\lambda^N e^{-\lambda(1+x_1 + \cdots + x_N)} \cdot (1 + x_1 + \cdots + x_N)^{N+1}
    \end{align}
    Find the conditional expectation of $R_0^{-N}$ given that $(R_1, \ldots, R_N)$ equals $(x_1, \ldots, x_N)$.
\end{example}
\begin{solution}
    We can write
    \begin{align}
        E(R_0^{-N} \mid R_1 = x_1, \ldots, R_N = x_N) &= \int_0^\infty \lambda^{-N} h(y \mid x_1, \ldots, x_N) \dd{\lambda}\\
        &= \int_0^\infty \frac{1}{N!}e^{-\lambda(1+x_1+\cdots+x_N)}(1+x_1+\cdots+x_N) \dd{\lambda}
    \end{align}
    which is an easy integral that eventually simplifies into
    \begin{align}
        \frac{1}{N!}(1+x_1+\cdots+x_N)^{N}
    \end{align}
\end{solution}

\subsection{Theorem of Total Expectation}
\begin{proposition}
    For $R_1$ absolutely continuous and $R_2$ r.v.,
    \begin{align}
        E(R_2) = \int_\real E(R_2 \mid R_1 = x) f_1(x) \dd{x}
    \end{align}
    This formula suggests that $E(R_2 \mid R_1 = x) = E(R_2)$, which suggests that for $g(x) = E(R_2 \mid R_1 = x)$, $E(R_2) = E(g(R_1))$.
\end{proposition}
\begin{proof}
    The right hand side of the equation equals
    \begin{align}
        \iint_{x \times y} y h(y \mid x) \dd{y} f_1(x) \dd{x}
    \end{align}
    This equals
    \begin{align}
        \iint_{x \times y} y f(x,y)/\cancel{f_1(x)} \dd{y} \cancel{f_1(x)} \dd{x}
    \end{align}
    which equals the left hand side.
    \begin{align}
        \iint_{x \times y} y f(x,y) \dd{y} \dd{x} = E(R_2)
    \end{align}
\end{proof}
\begin{lemma}
    There is a discrete version of this. Suppose $R_1$ is discrete with probability mass function $p_1(x)$. Then $E(R_2)$ equals
    \begin{align}
        \sum_x E(R_2 \mid R_1 = x) p_1(x)
    \end{align}
\end{lemma}
\begin{example}
    Person stuck in a mine (poor guy!). There are three doors labelled 1, 2, and 3 and one of them leads to safety. Suppose Door 1 takes 3 hours to lead to safety. Door 2 takes 5 hours and ends up at the starting point. Door 3 takes 7 hours and also ends up at the starting point. Once he gets back, he forgets which door he just picked. Every time he gets into the mine, he randomly picks a door with no memory. How long does it take to get out?
\end{example}
\begin{solution}
    Set $R$ as the number of hours to get to safety. The trick is to let $S$ be the first door picked. The expectation of $R$ equals
    \begin{align}
        E(R) = \sum_{i \in S} E(R \mid S = i) \probability(S = i)
    \end{align}
    This equals
    \begin{align}
        E(R) &= E(R \mid S = 1) \probability(S = 1)\\\
        &+ E(R \mid S = 2) \probability(S = 2) \notag\\
        &+ E(R \mid S = 3) \probability(S = 3) \notag
    \end{align}
    This is effectively a `dynamic' programming problem where previous... adventures are the recursive relation. So this reduces to
    \begin{align}
        E(R) = \frac{1}{3}\left( 3 + 5 + E(R) + 7 + E(R) \right) \implies E(R) = 15
    \end{align}
\end{solution}

\subsection{Expectation Conditional on an Event}
We sometimes care about expectations like $E(R \mid R \in B)$ where $B \subseteq \real$.
\begin{proposition}
    The expectation simplifies into
    \begin{align}
        E(R \mid R \in B) \ce{->[indicators]} \frac{E(R I_B(R))}{\probability(R \in B)}
    \end{align}
\end{proposition}
\begin{proof}
    \begin{enumerate}
        \item The discrete case. Then
            \begin{align}
                E(R \mid R \in B) &= \sum_x x \probability(R = x \mid R \in B)\\
                &= \sum_{x \ne 0} x \frac{\probability(R = x, R \in B)}{\probability(R \in B)}\\
                &= \sum_{x \ne 0} x \frac{\probability\qty(R I_B(R) = x)}{\probability(R \in B)}
            \end{align}
            This just equals $$\frac{E(R I_B(R))}{\probability(R \in B)}$$
        \item The absolutely continuous case, i.e. $(R_1,R_2)$ absolutely continuous with joint density $f(x,y)$. Then, the conditional distribution function equals
            \begin{align}
                F_R(R \le x_0 \mid R \in B)
            \end{align}
            That is equivalent to
            \begin{align}
                \frac{\probability(R \le x_0 \mid R \in B)}{\probability(R \in B)}
            \end{align}
            which integrates to
            \begin{align}
                \frac{\int_{x \in B \cap x \le x_0} f(x) \dd{x}}{\probability(R \in B)}
            \end{align}
            which, using indicators, becomes
            \begin{align}
                \frac{\int_{-\infty}^{x_0} f(x) I_B(x) \dd{x}}{\probability(R \in B)}
            \end{align}
            We can define the conditional density of $R$ given $R \in B$ to be
            \begin{align}
                f_R(x \mid R \in B) = \frac{f(x) I_B(x)}{\probability(R \in B)}
            \end{align}
            so naturally, $E(R \mid R \in B)$ equals
            \begin{align}
                \frac{\int_\real x f_R(x) I_B(x) \dd{x}}{\probability(R \in B)} = \frac{E(RI_B(R))}{\probability(R \in B)}
            \end{align}
    \end{enumerate}
\end{proof}
\begin{aside}
    Suppose $\real = \bigcup B_i$ where $B_i$ are all disjoint. Then,
    \begin{align}
        E(R) = \sum_i E(R \mid R \in B_i) \probability(R \in B_i)
    \end{align}
    This is an alternative Theorem of Total Expectation using the fact that the $B_i$ span the entirety of $\real$. This is because $I_B(R)$ reduces into $1$ on all reals when $B \equiv \real$.
\end{aside}

\begin{example}
    Roll a die $n$ times and define $R$ as the number of 1s. Find $E(R \mid R \ge 2)$.
\end{example}
\begin{solution}
    The expectation of $R$ equals
    \begin{align}
        E(R) = E(R \mid R = 0) \probability(R = 0) + E(R \mid R = 1) \probability(R = 1) + \chi \probability(R \ge 2)
    \end{align}
    where $\chi$, the sum of the rest of the terms, is what we want to solve for.
    \begin{align}
        \frac{n}{6} &= 0 + E(R \mid R = 1) \probability(R = 1) + \chi\\
        &= n \qty(\frac{5}{6})^{n-1} \qty(\frac{1}{6}) + \chi \qty(1 - \frac{5}{6}^n - n\qty(\frac{5}{6})^{n-1}\qty(\frac{1}{6}))
    \end{align}
    and it can be computed $E(R \mid R \ge 2)$ (i.e. $\chi$) using some algebra.
\end{solution}

    \section{November 6, 2024}


\subsection{Complex Space}

\subsubsection{Complex Valued Random Variables}
Take $\psp$ as some probability space. We say that a complex valued function
\begin{align}
    T: \Omega \to \complex
\end{align}
is a random variable if both $\Re(T)$ and $\Im(T)$ are `normal' (real) random variables. Specifically,
\begin{align}
    T = \Re(T) + i\Im(T)
\end{align}
where $\Re(T): \Omega \to \real$ and $\Im(T): \Omega \to \real$. Then, the expectation of $T$ equals
\begin{align}
    E(T) = E(\Re T) + iE(\Im T)
\end{align}

\subsubsection{Characteristic Functions}
Take $R$ as a r.v. Its characteristic function is some map
\begin{align}
    M_R : \real \to \complex
\end{align}
given by $M(u \in \real) = E\qty(e^{-iuR})$ where $R$ is some random variable, and $M$ is some function of a random variable that just so happens to be complex valued.

\begin{proposition}
    If $R$ is absolutely continuous with density $f_R(x)$, then
    \begin{align}
        M(u) = \int_{-\infty}^\infty e^{-iux} f_R(x) \dd{x}
    \end{align}
    \textbf{Interestingly, this is just the Fourier Transform of $f_R$. (note that this is well defined because it absolutely converges)}
\end{proposition}

\subsubsection{Generalized Characteristic Function}
The generalized characteristic function $N_R(s \in \complex)$ is given by
\begin{align}
    N_R(s) = E(e^{-sR}) \ce{->[if $R$ abs cont][density $f_R(x)$]} \int_{-\infty}^\infty e^{-sx} f_R(x) \dd{x}
\end{align}
This is not necessarily well-defined in general. In the case where $s = iu$, then $N_R(s) = N_R(iu) = \cdots = M_R(u)$. \textbf{In general, this is the (2-sided) Laplace transform of $f_R(x)$.}

\begin{lemma}
    Suppose $R$ is uniformly distributed on $[-1,1]$. Then, $N_R(s) = \int_\real e^{-sx} f_R(x) \dd{x}$. After some integration this becomes
    \begin{align}
        N_R(s) = \frac{1}{2s}(e^s - e^{-s})
    \end{align}
    It can be shown that $N_R(0) = 1$. This is \textit{always true} because $\int_\real f_R(x) \cdot 1 \dd{x} \equiv 1$.
\end{lemma}

\begin{example}
    Take the exponential function $R \sim \exp(1)$. Then,
    \begin{align}
        N_R(s) = \mathcal{L}(e^{-x}) = \frac{1}{s+1} && \Re(s+1) > 0
    \end{align}
\end{example}

\subsubsection{Laplace Transform Aside}
\begin{aside}
    The 2-sided Laplace transform $\mathcal{L}_2$ is on $(-\infty,\infty)$ whereas the `normal' Laplace transform is on $[0,\infty)$.
\end{aside}
\begin{theorem}
    \textbf{Sums of random variables correspond to products of characteristic functions.} Take $R_1, \ldots, R_N$ as independent random variables with $N_{R_i}(s)$ finite for all $i$. Define
    \begin{align}
       R_0 = \sum_i R_i
    \end{align}
    Then, $N_{R_0}(s)$ is finite and equals
    \begin{align}
        N_{R_0}(s) = \prod_i N_{R_i}(s)
    \end{align}
    and for $s = iu$, this applies to $M_R(u)$ as well.
\end{theorem}
\begin{proof}
    $N_{R_0}$ is equal to $E(e^{-sR_0})$. This factors into $E(e^{-sR_1} \cdots e^{-sR_N})$ which, by independence, yields
    \begin{align}
        N_{R_0} = N_{R_1} \cdots N_{R_N}
    \end{align}
\end{proof}

\begin{example}
    Suppose $R_1,R_2$ independent. Let $R_1$ be uniformly distributed on $[-1,1]$ and $R_2 \sim \exp(1)$. Set $R_0 = R_1 + R_2$. We can compute the generalized characteristic functions.
    \begin{align}
        N_{R_0}(s) = N_{R_1}(s)\cdot N_{R_2}(s) = \frac{1}{2s}\qty(e^s - e^{-s}) + \frac{1}{s+1}
    \end{align}
    How would we find $R_0$ then? By taking the inverse Laplace transform of $N_{R_0}$. Taking this inverse may require use of the Residue Theorem.
\end{example}

\begin{theorem}
    Suppose $R$ is absolutely continuous and $N_R(s)$ is given by $\int_{-\infty}^\infty e^{-sx} h(x) \dd{x}$ for some piecewise continuous $h(x)$ for $\{ s \mid \Re(s) = a\}$ on some $a$. Then, $R$ has density
    \begin{align}
        f_R(x) = h(x)
    \end{align}
    (if a function $h(x)$ whose Laplace transform is $N_R(s)$ can be found on some $s$ for which $\Re(s) = a$, then that function is the inverse Laplace transform of $N_R(s)$)
\end{theorem}

\subsubsection{Laplace Transform Properties}
Set $\mathcal{L}_f(s) = \int_\real e^{-sx} f(x) \dd{x}$ for general $f: \mathcal{F}$ (any function; not necessarily a density) as the Laplace transform of $f$.
\begin{aside}
    Take $f \equiv 1$. Then $\mathcal{L}_f(s) = \eval{\frac{-e^{-sx}}{s}}_{-\infty}^\infty$. This is not necessarily defined for \textit{any} $s \in \complex$.
\end{aside}
\begin{lemma}
    If $\abs{f(t)}$ is bounded by some constant $M_1 e^{a_1t}$ for $t \ge 0$ and $f(t) \le M_2 e^{a_2 t}$ for $t \le 0$, then $\mathcal{L}_f(s)$ is finite f, then $\mathcal{L}_f(s)$ is finite for $a_1 < \Re(s) < a_2$.
\end{lemma}
\begin{proof}
    The absolute Laplace transform is given by
    \begin{align}
        \int_{-\infty}^\infty \abs{e^{-sx} f(x)} \dd{x}
    \end{align}
    which can be split into
    \begin{align}
        & \int_{0}^\infty e^{-\Re(s)x} \abs{f(x)} \dd{x} + \int_{-\infty}^0 e^{-\Re(s)x} \abs{f(x)} \dd{x}\\
        \le & \int_0^\infty M_1 e^{-\Re(s) + a_1}t \dd{t} + \int_{-\infty}^0 M_2 e^{-\Re(s) + a_2}t \dd{t}
    \end{align}
    which only holds on $t \in (a_1,a_2)$.
\end{proof}

\begin{enumerate}
    \item it's linear
    \item $f(x - a) \ce{->[$\mathcal{L}_f$]} e^{-sa} \mathcal{L}_f(s)$
    \begin{proof}
        highly trivial
    \end{proof}
    \item $f(-x) \ce{->[$\mathcal{L}_f$]} \mathcal{L}_f(-s)$
    \begin{proof}
        again... trivial
    \end{proof}
    \item $e^{-ax} f(x) \ce{->[$\mathcal{L}_f$]} \mathcal{L}_f(s+a)$
    \begin{proof}
        obvious
    \end{proof}
    \item Heaviside function
    \begin{align}
        u(x) = \begin{cases}
            1 & x \ge 0\\
            0 & \text{else}
        \end{cases}
    \end{align}
    Then, $\mathcal{L}_u(s) = 1/s$ for $\Re(s) > 0$.
    \begin{proof}
        not as \textit{trivial}, but still easy
    \end{proof}
\end{enumerate}



    \section{November 8, 2024}

\subsection{Characteristic Functions Recap}
Recall $M_R(u \in \real) \equiv E(e^{-iuR})$. In the case where $R$ has density $f(x)$, then $M_R(u)$ just equals the Fourier transform $f(x) \ce{->[$\mathcal{F}$]} \hat{f}(\xi)$.\\

\noindent In general, for $N_R(s \in \complex) \equiv E(e^{-sR})$. In the case where $R$ has density $f(x)$, then $N_R(s)$ just equals the Laplace transform $f(x) \ce{->[$\mathcal{L}$]} \hat{f}(\xi)$. The Laplace transform is finite when $f(x)$ is bounded by $Me^{at}$ for some constants $M$ and $a$ both above and below $x = 0$.

\subsubsection{Properties of Characteristic Functions}
Some properties were listed for Laplace transforms.
\begin{proposition}
    \begin{align}
        x^\alpha e^{-ax} u(x) \ce{->[$\mathcal{L}$]} \frac{\Gamma(\alpha+1)}{(s+a)^{\alpha+1}} && \Re s > -a
    \end{align}
\end{proposition}
\begin{proof}
    This can be `proven' via
    \begin{align}
        \mathcal{L}_f(s)
        &= \int_\real e^{-sx} x^{\alpha} e^{-ax} u(x) \dd{x}\\
        &= \int_\real e^{-x(a+s)} x^{\alpha} u(x) \dd{x}\\
        &= \int_\real \frac{y^\alpha}{(a+s)^\alpha}e^{-y} \frac{\dd{y}}{(a+s)}\\
        &= \frac{1}{(a+s)^{\alpha+1}} \int_0^\infty y^\alpha e^{-y} \dd{y}\\
        &= \frac{1}{(a+s)^{\alpha+1}} \Gamma(\alpha+1)
    \end{align}
    However, this is not entirely true because $y = (a+s)x$ as a substitution assumes $s$ is real. Due to $\mathcal{L}_f(s)$ being analytic, it suffices to evaluate this integral on a line.
\end{proof}

\subsection{Examples}
These examples are not worth copying the solution to, but the techniques are still useful.
\begin{enumerate}
    \item Suppose $R_1,R_2$ are independent; $R_1$ is uniformly distributed on $[-1,1]$ and $R_2 \sim \exp(1)$. Take $R_0 = R_1 + R_2$. Find the density of $R_0$.
    \begin{solution}
        use characteristic functions and convolution theorem. he spent like 10 minutes doing this example by hand. it's crazy.
    \end{solution}
    \item Suppose $R_1,\ldots, R_N$ are independent and each are Gaussian distributed with $R_i \sim N(m_i, \sigma_i^2)$. Find the density of $R_0 = R_1 + \cdots R_N$. 
    \begin{solution}
        Let $R \sim N(m,\sigma^2)$. Compute $N_R(s)$, i.e. the Laplace transform of the Gaussian distribution. 
        \begin{align}
            N(m,\sigma^2) \ce{->[$\mathcal{L}$]} e^{-sm + \sigma^2s^2/2}
        \end{align}
        The answer is that the density stays normally distributed after transformation. If we have $N$ of these, then we can use the convolution theorem to find
        \begin{align}
            N_{R_0}(s) = e^{-s(m_1 + \cdots + m_N) + \frac{(\sigma_1^2 + \cdots + \sigma_N^2)s^2}{2}}
        \end{align}
        This is a normal distribution that linearly sums the means and variances of its components.
    \end{solution}

    \begin{aside}
        For $R$ absolutely continuous, if there is an $h(x)$ such that $\mathcal{L}_h(s) = N_R(s)$ on $\Re s = a$ for some $a$, then that implies that $R$ has density $h(x)$.
        \begin{proof}
            Student: ``did you ever prove this?''\\

            \noindent Prof: ``no but we stated it''
        \end{proof}
    \end{aside}
    
    \item Let $R \sim \text{pois}(\lambda)$. Find $N_R(\lambda)$. If $R_1, \ldots, R_N$ independennt and $R_i \sim \text{pois}(\lambda_i)$, then find $R_0 = R_1 + \cdots + R_N$
    \begin{solution}
        Here, we cannot integrate because the Poisson distribution is discrete. Instead, for some Poisson distribution $R$ with parameter $lambda$,
        \begin{align}
            N_R(s) = E(e^{-sR}) &= \sum_k e^{-sk} p_R(k)
            &= \sum_k e^{-sk} \frac{\lambda^k e^{-\lambda}}{k!}\\
            &= e^{-\lambda} \cdot \sum_k \frac{(\lambda e^{-s})^k}{k!}\\
            &= e^{-\lambda} e^{e^{-s}\lambda}
        \end{align}
    \end{solution}
    For $R_0 = R_1 + \cdots + R_N$, the characteristic function is the product of the components,
    \begin{align}
        e^{-\lambda_1} e^{e^{-s}\lambda_1} \cdots e^{-\lambda_N} e^{e^{-s}\lambda_N} 
    \end{align}
    which obviously simplfiies into a Poisson distribution with parameter $\lambda_1 + \cdots + \lambda_N$.

    \item Take $R \sim \exp(\lambda)$. Find $N_R(s)$. Then, find $R_0 = R_1 + \cdots + R_N$.
    \begin{solution}
        Using the table of Laplace transforms,
        \begin{align}
            N_R(s) = \frac{\lambda}{s+\lambda} && \Re s > -\lambda
        \end{align}
        Clearly, $N_{R_0}(s)$ equals
        \begin{align}
            N_{R_0}(s) = \frac{\lambda^n}{(s+\lambda)^n} \implies f_{R_0}(x) = x^{n-1} e^{\lambda x} \frac{\lambda^n}{(n-1)!} u(x)
        \end{align}
        where $u(x)$ is the Heaviside function. This density is that of a Gamma distribution $\Gamma(\alpha = n, \beta = 1/\lambda)$.
    \end{solution}

    \begin{definition}
        A Gama r.v. with parameters $\alpha, \beta$ has density
        \begin{align}
            f(x) = \begin{cases}
                \frac{x^{\alpha-1} e^{-x/\beta}}{\Gamma(\alpha) \beta^\alpha} & x \ge 0\\
                0 & x < 0
            \end{cases}
        \end{align}
    \end{definition}

\end{enumerate}






    \section{November 13, 2024}

\subsection{MORE Properties of Characteristic Functions}
Recap. What are characteristic functions? Given $R$ r.v.,
\begin{definition}
    The characteristic function of $R$, $M_R(u)$, equals
    \begin{align}
        M_R(u) = E(e^{-iuR}) = \int_\real e^{-iux} f(x) \dd{x} = \mathcal{F}_R(u)
    \end{align}
    for $u \in \real$. The generalized characteristic of $R$, $N_R(u)$, equals
    \begin{align}
        N_R(s) = E(e^{-sR}) = \int_\real e^{-sx} f(x) \dd{x} = \mathcal{L}_R(u)
    \end{align}
    for $s \in \complex$ (does not exist for all $s \in \complex$). Note that $N_R(iu) = M_R(u)$.
\end{definition}
More properties...
\begin{enumerate}
    \item $N_R(0) = M_R(0) = E(e^0) = 1$
    \item $\abs{M_R(u)} \le 1$ for all $u \in \real$.
    \begin{proof}
        In the case that $R$ has density $f_R(x)$,
        \begin{align}
            \abs{M_R(u)} = \abs{\int_\real e^{-iux} f(x) \dd{x}}
        \end{align}
        which is in general a complex number. This is bounded by
        \begin{align}
            \abs{\int_\real \ldots} &\le \int_\real \abs{e^{-iux} f_R(x)} \dd{x}\\
            &\le \int_\real \abs{f_R(x)} \dd{x}
        \end{align}
        but because $f_R$ is a density, this integrates to $1$.
    \end{proof}
    \item If $R$ has density $f$ and $f$ is an even function, then
    \begin{align}
        M_R(u) \in \real
    \end{align}
    the characteristic function is necessarily real valued.
    \begin{proof}
        Clearly,
        \begin{align}
            M_R(u) &= \int_{-\infty}^\infty e^{-iux} f(x) \dd{x}\\
            &= \int_{-\infty}^\infty \qty[\cos(-ux) + i\sin(-ux)] f(x) \dd{x}\\
            &= \int_{-\infty}^\infty \qty[\cos(ux) - i\sin(ux)] f(x) \dd{x}
        \end{align}
        Since $\sin(ux)$ is an odd function, this means that $\Im(M_R(u)) \equiv 0$, so
        \begin{align}
            M_R(u) = \int_\real \cos(ux) f(x) \dd{x} \in \real
        \end{align}
    \end{proof}
    \item If $R$ is a discrete r.v. on $\integer$, then $M_R$ is periodic with period $2\pi$, i.e. $M_R(u + 2\pi) = M_R(u)$.
    \begin{proof}
        \begin{align}
            M_R(u) = E(e^{-iuR}) = \sum_{x \in R} e^{-iux} \probability(R = x)
        \end{align}
        If $R$ takes only integral values, then $M_R(u + 2\pi)$ equals
        \begin{align}
            M_R(u + 2\pi) = \sum_{x \in R} e^{-iux} e^{2\pi i u} p_x
        \end{align}
        Since $u \in \integer$, this becomes
        \begin{align}
            M_R(u + 2\pi) = \sum_{x \in R} e^{-iux} p_x = M_R(u)
        \end{align}
    \end{proof}
    \begin{aside}
        For $R$ discrete r.v., the coefficients $p_n$ are the Fourier coefficients of the function $M_R(u)$.
        \begin{align}
            p_n = \frac{1}{2\pi} \int_0^{2\pi} e^{iun} M_R(u) \dd{u}
        \end{align}
    \end{aside}
    \item \textbf{Moment Generating Property} If $N_R$ is analytic in a ball around $0$ of radius $\varepsilon > 0$, then the generalized characteristic function can be written as
    \begin{align}
        N_R(s) = \sum_{k=0}^\infty \frac{(-1)^k}{k!} \underbracket{E(R^k)}_{\text{$k$-th moment}} s^K
    \end{align}
    This tells us that once we compute the characteristic function, we can read off the $k$-th moments.
    \begin{proof}[Sketch of Proof]
        For the case where $R$ has density $f(x)$,
        \begin{align}
            N_R(s)
            &= \int_\real e^{-sx} f(x) \dd{x} = \int_\real \qty[\sum_k \frac{(-1 \cdot s \cdot x)^k}{k!}] f(x) \dd{x}\\
            &= \sum_k \int_\real \qty(\frac{(-1)^k x^k}{k!} f(x) \dd{x}) s^k\\
            &= \sum_k \frac{(-1)^k}{k!} E(R^k) s^k & \qedhere
        \end{align}
    \end{proof}
    \begin{example}
        If $R \sim N(0,1)$, then $N_R(s) = e^{-s^2/2}$. Use this to find $E(R^k)$ for all $k$.
    \end{example}
    \begin{solution}
        \begin{align}
            N_R(s) = e^{-s^2/2} = \sum_\lambda \frac{s^{2\lambda}}{2^\lambda \lambda!} \leftrightarrow \sum_k \frac{(-1)^k}{k!} E(R^k) s^k
        \end{align}
        This has odd moments equal to zero because $s^{2\lambda}$ does not include odd exponents. So
        \begin{align}
            \sum_\lambda \frac{E(R^{2\lambda}}{(2\lambda)!}s^{2\lambda}
        \end{align}
        Then this converts into
        \begin{align}
            \frac{E(R^{2\lambda})}{(2\lambda)!} = \frac{1}{2^\lambda \lambda!} \implies \boxed{E(R^{2k}) = \frac{(2k)!}{2^{k}k!}}
        \end{align}
    \end{solution}
    \begin{aside}
        If $\sum_k a_k s^k = \sum_k b_k s^k$ for all $s$ in an open set of $\complex$, then $a_k = b_k$.
    \end{aside}
    (The core lemmas behind this, from complex analysis, were assumed)
\end{enumerate}

\subsection{Notions of Convergence of Random Variables}
\begin{definition}
    Let $R_1, \ldots$ be random variables on the same probability space. We say that $R_n \ce{->[$\probability$]} R$ ($R_n$ converges to $R$ in probability) if for all $\varepsilon > 0$,
    \begin{align}
        \probability(R_n - R > \varepsilon) \ce{->[$n \to \infty$]} 0
    \end{align}
\end{definition}
This connects back to the Weak Law of Large Numbers.
\begin{definition}
    Let $R_1, \ldots$ be random variables on the same probability space. We say that $R_n \ce{->[$d$]} R$ (convergence in distribution) if $F_{R_n}(x) \to F_R(x)$ for all points $x$ at which these functions are continuous.
\end{definition}

\begin{example}
    Suppose $R_n \sim \exp(n)$ are independent. Show that (i) $R_n \ce{->[$p$]} 0$ and (ii) $R_n \ce{->[$d$]} 0$.
\end{example}
\begin{solution}
    \begin{enumerate}
        \item[(i)] Let $\varepsilon > 0$. Trivially,
        \begin{align}
            \probability(R_n \ge \varepsilon) = \int_\varepsilon^\infty f_{R_n}(x) \dd{x} \ce{->[$n \to \infty$]} 0
        \end{align}
        \item[(ii)] Let $x \ge 0$. Obviously, $F_{R_n}(x) = 1 - e^{-nx}$. For $x > 0$, as $n \to \infty$, $F_{R_n}(x) \to 1$. For $x \le 0$, $F_{R_n}(x) \to 0$. This suggests that the distribution approaches the Heaviside function, and the density approaches $\delta(x)$, i.e. only has density at $x = 0$. Thus, as $n \to \infty$, $R_n$ converges to effectively zero. This also shows that $F_{R_n}(x) \to F_R(x)$ on all $x$ where there is continuity.
    \end{enumerate}
\end{solution}



    \section{November 15, 2024}

\subsection{Convergence (cont)}
Recap. Random variables $R_n \ce{->[$\probability$]} R$ converge in probability means that for all $\varepsilon > 0$, $\probability(\abs{R_n - R} \ge \varepsilon) \ce{->[$n \to \infty$]} 0$. Random variables $R_n \ce{->[$d$]} R$ converge in distribution means that $F_{R_n}(x) \to F_{R}(x)$ as $n \to \infty$ for all $x$ where $F_R$ is continuous.

\begin{proposition}
    If a sequence of r.v. converges in probability, then it also converges in distribution.
    \begin{align}
        R_n \ce{->[$\probability$]} R \implies R_n \ce{->[$d$]} R
    \end{align}
\end{proposition}
\begin{proof}
    Claim. For $\varepsilon > 0$,
    \begin{align}
        F_R(x - \varepsilon) - \probability(\abs{R_n - R} \ge \varepsilon) \le F_{R_n}(x) \le F_R(x + \varepsilon) + \probability(\abs{R_n - R} \ge \varepsilon)
    \end{align}
    Then, as $n \to \infty$,
    \begin{align}
        F_R(x - \varepsilon) \le \lim_{n\to\infty} F_{R_n}(x) \le F_R(x + \varepsilon) && \varepsilon > 0
    \end{align}
    If $F_R(x)$ is continuous at $x$, then as $\varepsilon \to 0$, this implies $F_R(x \pm \varepsilon) \to \lim_{n\to\infty} F_{R_n}(x)$, which implies
    \begin{align}
        F_R(x) \le \lim_{n\to \infty} F_{R_n}(x) \le F_R(x)
    \end{align}
    which implies that $R_n \ce{->[$d$]} R$ for points where $F_R$ is continuous.
    \begin{proof}[Proof of Claim]
        First,
        \begin{align}
            F_{R_n}(x) &= \probability(R_n \le x)\\
            &= \probability(R_n \le x \cap R - R_n < \varepsilon) + \probability(R_n \le x \cap R_n - R \ge \varepsilon)
        \end{align}
        Notably, this is less than or equal to
        \begin{align}
            F_{R_n}(x) &\le \probability(R \le x + \varepsilon) + \probability(\abs{R_n - R} \ge \varepsilon)\\
            = F_R(x + \varepsilon) + \probability(\abs{R_n - R} \ge \varepsilon)
        \end{align}
        Similarly,
        \begin{align}
            F_R(x - \varepsilon) &= \probability(R \le x - \varepsilon)\\
            &= \probability(R \le x - \varepsilon \cap R - R_n < \varepsilon) + \probability(R \le x - \varepsilon \cap R_n - R \le \varepsilon)\\
            &\le \probability(R_n \le x) + \probability(\abs{R_n - R} \ge \varepsilon)\\
            F_{R_n}(x) &\le F_R(x - \varepsilon) - \probability(\abs{R_n - R} \ge \varepsilon)
        \end{align}
        which constructs the inequality
        \begin{align}
            F_R(x - \varepsilon) - \probability(\abs{R_n - R} \ge \varepsilon) \le F_{R_n}(x) \le F_R(x + \varepsilon) + \probability(\abs{R_n - R} \ge \varepsilon)
        \end{align}
    \end{proof}
\end{proof}

\begin{lemma}
    The converse of this is not true, i.e. convergence in distribution does not imply convergence in probabiliity.
\end{lemma}

As a silly example,
\begin{example}
    Take $R_n$ as the number of heads when a coin is flipped once. Let $R$ be the same. Let $R_i$ all be independent.
\end{example}
\begin{solution}
    \begin{align}
        F_{R_n}(x) = \begin{cases}
            1 & x \ge 1\\
            1/2 & x \in [0,1)\\
            0 & \text{else}
        \end{cases}
    \end{align}
    This obviously converges to $F_R(x)$. However, the probabilities do not converge for all $\varepsilon > 0$. Take $0 < \varepsilon < 1$. Then,
    \begin{align}
        \probability(\abs{R_n - R} \ge \varepsilon = 1/2) = 1/2 \ne 0
    \end{align}
    which does not tend to zero as $n\to\infty$.
\end{solution}

Convergence in probability has a rather strange aspect that a random variable doesn't converge to itself.

\begin{lemma}
    $R_n \ce{->[$d$]} R$ is not the same as $R_n - R \ce{->[$d$]} 0$.
    \begin{proof}
        This is an example of $R_n$ converging to $R$ in distribution. However, $R_n - R \ne 0$, as it is not effectively zero (has non-zero density elsewhere).
    \end{proof}
\end{lemma}


\subsection{A Theorem to State but Not Prove}
\begin{theorem}
    \begin{align}
        R_n \ce{->[$d$]} R \iff \forall u \in \real, M_{R_n}(u) \to M_R(u)
    \end{align}
    \begin{proof}
        This was not proven.
    \end{proof}
\end{theorem}

\subsection{Central Limit Theorem}
\begin{theorem}
    \textbf{(Central Limit Theorem)} Let $R_1, R_2, \ldots$ be i.i.d (and absolutely continuous) random variables with mean $m$, finite variance $\sigma^2$, (and finite $E(\abs{R}^3) < \infty$) (the last condition is not strictly necessary but is used in this proof?). Define the r.v.
    \begin{align}
        T_n \equiv \frac{1}{\sigma \sqrt{n}} \qty(\sum_{i=1}^n R_i - nm)
    \end{align}
    Then,
    \begin{align}
        T_n \ce{->[$d$]} N(0,1)
    \end{align}
    $T_n$ converges to the normal distribution with mean $0$ and variance $1$.
\end{theorem}

\begin{aside}
    Some remarks.
    \begin{enumerate}
        \item The first moment for $T_n$ equals
        \begin{align}
            E(T_n) = \frac{1}{\sigma \sqrt{n}} \qty(n E(R_i) - nm) = 0
        \end{align}
        \item The variance for $T_n$ equals
        \begin{align}
            \frac{1}{n\sigma^2} \qty(n \sigma^2 - 0) = 1
        \end{align}
    \end{enumerate}
    This suggests that $T_n$ necessarily has mean and variance of $0$ and $1$, and \textit{converge to a normal distribution}.
\end{aside}

\begin{example}
    Flip a coin many times. Let $R_i$ be the number of heads in the $i$-th pair of even flips (e.g. 1 is first 2, 2 is next 2, etc.)
\end{example}
\begin{solution}
    We would expect $E(R_i) = 1$. The weak law of large numbers says that
    \begin{align}
        \frac{1}{n} \sum_{i=1}^n R_i - 1 \ce{->[$\probability$][n \to \infty]} 0
    \end{align}
    The density as $n \to \infty$ approaches $\delta(x - 1)$. The central limit theorem tells us that this distribution is, in fact, a normal distribution. 
\end{solution}

\subsubsection{Proof}
\begin{proof}

It suffices to show that the characteristic function of $T_n$ converges to the characteristic function of $N(0,1)$, i.e.
\begin{align}
    M_{T_n}(u) \ce{->[$n \to \infty$]} M_{N(0,1)}(u) \equiv e^{-u^2/2} && \forall u
\end{align}
Assume WLOG that $E(R_n) = 0$. Then,
\begin{align}
    M_{T_n}(u) &= E\qty(\exp\qty(-\frac{iu}{\sigma \sqrt{n}}\sum_{j=1}^n R_j)) 
    = \prod_{j=1}^n E\qty(\exp\qty(-\frac{iuR_j}{\sigma \sqrt{n}})) \\
    &= \qty[M_{R_1}\qty(\frac{u}{\sigma\sqrt{n}})]^n
\end{align}
$M_{R_1}$ thus must be computed.
\begin{align}
    M_{R_1}\qty(\frac{u}{\sigma\sqrt{n}}) = \int_{-\infty}^\infty e^{-\frac{iu}{\sigma\sqrt{n}}x} f(x) \dd{x}
\end{align}
Now we do a Taylor Expansion.
\begin{aside}
    \textbf{(Taylor Expansions)} For $z \in \complex$,
    \begin{align}
        e^z = 1 + z + \frac{z^2}{2} + O\qty(\abs{z}^3)
    \end{align}
    \begin{aside}
        $g(x) = O(f(x))$ means $g(x)$ is on the order of $f(x)$, i.e. there exists some $C$ such that
        \begin{align}
            \abs{g(x)} \le C \abs{f(x)}
        \end{align}
    \end{aside}
\end{aside}
Thus, the integral from above equals
\begin{align*}
    M_{R_1}\qty(\frac{u}{\sigma\sqrt{n}}) &= \int_{-\infty}^\infty \qty[1 - \frac{iux}{\sigma\sqrt{n}} + \frac{1}{2}\qty(\frac{iux}{\sigma\sqrt{n}})^2 + O\qty(\left| \frac{-iux}{\sigma\sqrt{n}} \right|^3)] f(x) \dd{x}\\
    &= 1 - \frac{iu}{\sigma\sqrt{n}} E(R_1) - \int_\real \frac{1}{2} \frac{u^2x^2}{n\sigma^2} f(x) \dd{x} + \int_\real O\qty(\frac{\abs{u}^3\abs{x}^3}{n^{3/2}\sigma^3}) f(x) \dd{x}\\
    &= 1 - 0 - \frac{u^2}{2n\sigma^2}\sigma^2 + O\qty(\frac{\abs{u}^3}{n^{3/2}})E(R^3)
\end{align*}
As $n \to \infty$, we can take the log:
\begin{align}
    \log\qty(M_{R_1}\qty(\frac{u}{\sigma \sqrt{n}}))^n
\end{align}
which equals
\begin{align}
    n\log\qty(1 - \frac{u^2}{2n} + O\qty(\frac{\abs{u}^3}{n^{3/2}}))
\end{align}
which equals, because $\log(1 + z) = z + O(\abs{z}^2)$,
\begin{align}
    n \qty(-\frac{1}{2}\frac{u^2}{n} + O(1/n^{3/2}))
\end{align}
    which as $n \to \infty$, tends to, because $E(R^3)$ is bounded,
\begin{align}
    -\frac{1}{2}u^2 + 0
\end{align}
Recall that this is logarithmic. When we exponentiate both sides,
\begin{align}
    \qty(M_{R_1})^n \to e^{-u^2/2}
\end{align}

\end{proof}









    \section{November 18, 2024}

\subsection{Limits of Sets}
\begin{definition}
    Take $A_1, A_2, \ldots \in \Omega$ as some sets. Then,
    \begin{align}
        \limsup_n A_n = \bigcap_{n=1}^\infty \bigcup_{k=n}^\infty A_k
    \end{align}
    and
    \begin{align}
        \liminf_n A_n = \bigcup_{n=1}^\infty \bigcap_{k=n}^\infty A_k
    \end{align}
\end{definition}

\begin{proposition}
    \begin{enumerate}
        \item[(i)] The limit
        \begin{align}
            \limsup_n A_n = \{ \omega \in \Omega \mid \omega \text{ is in infinitely many } A_n \}
        \end{align}
        \item[(ii)] the limit
        \begin{align}
            \liminf_n A_n = \{ \omega \in \Omega \mid \omega \text{ is in all, but finitely many } A_n \}
        \end{align}
        Equivalently, $\liminf \cdots$ means that $\omega$ is \textit{eventually} in $A_n$, i.e. for some $k$, $\omega \in A_n$ for all $k \ge n$
    \end{enumerate}
\end{proposition}
\begin{proof}
    \begin{enumerate}
        \item[(i)] By definition,
        \begin{align}
            \omega \in \limsup_n A_n
            &\iff \forall n, \omega = \bigcup_{k=n}^\infty A_k\\
            &\iff \forall n, \exists k \ge n, \omega \in A_k\\
            &\iff \omega \text{ is in infinitely many } A_n
        \end{align}
        because if this was not the case, then there would be some largest $n$, but that is a contradiction with there existing $k \ge n$ \textbf{for all} $n$.

        \item[(ii)] By definition,
        \begin{align}
            \omega \in \liminf_n A_n
            &\iff \omega \in \bigcap_{k=n}^\infty \text{ for some } n\\
            &\iff \text{for some } n, \forall k \ge n, \omega \in A_k
        \end{align}
        Beyond a certain $k$, $\omega$ must be in all $A_n$.
    \end{enumerate}
\end{proof}

As an example...
\begin{example}
    Take $A_n$ defined as
    \begin{align}
        A_n = \begin{cases}
            [-1,1] \subseteq \real & n \text{ even}\\
            [-2,2] \subseteq \real & n \text{ odd}
        \end{cases}
    \end{align}
    Then, $\limsup_n A_n = [-2,2]$ and $\liminf_n A_n = [-1,1]$.
\end{example}

\subsection{Almost Sure Convergence}
\begin{definition}
    Let $R_1, R_2, \ldots$ be random variables on a given probability space $\psp$. Define the set
    \begin{align}
        \{ R_n \to R \} \equiv \{ \omega \in \Omega \mid R_n(\omega) \ce{->[$n \to \infty$]} R(\omega) \}
    \end{align}
    We say that $R_n$ ``converges almost surely'' to $R$, or $R_n \ce{->[\text{a.s.}]} R$, if
    \begin{align}
        \probability(\{ R_n \to R \}) = 1
    \end{align}
\end{definition}

\begin{example}
    Take $\Omega = [0,1]$, $\curlyf$ as Borel sets of $[0,1]$, and $\probability([a,b]) = b-a$ where $0 \le a \le b \le 1$. Define $R_n(\omega)$ as
    \begin{align}
        R_n(\omega) = \begin{cases}
            1 & \omega = 0\\
            1/n & \omega \in (0,1]
        \end{cases}
    \end{align}
    Show that $R_n \ce{->[\text{a.s.}]} 0$
\end{example}
\begin{solution}
    The set $\{ R_n \to R \}$ is the set $\{ \omega \in [0,1] \mid R_n(\omega) \ce{->[$n \to \infty$]} 0 \}$. This set is just $(0,1]$. The probability of this set is 1 by definition, so this set obviously ``converges almost surely''.
\end{solution}

This definition is slightly difficult to work with.
\begin{proposition}
    \begin{align}
        \{ R_n \to R \} = \bigcap_{m=1}^\infty \liminf_n A_{nm}
    \end{align}
    where $A_{nm}$ is defined as
    \begin{align}
        A_{nm} \equiv \{ \omega \in \Omega \mid \abs{R_n(\omega) - R(\omega)} \le 1/m \}
    \end{align}
\end{proposition}
\begin{proof}
    This proof can be done with equivalences again
    \begin{align}
        \omega \in \{ R_n \to R \}
        &\iff R_n(\omega) \ce{->[$n \to \infty$]} R_\omega\\
        &\iff \forall \varepsilon > 0, \abs{R_n(\omega) - R(\omega)} < \varepsilon \text{ for $n$ suf. large}\\
        &\iff \forall m \in \natural, \abs{R_n(\omega) - R(\omega)} < 1/m \text{ for $n$ suf. large}\\
        &\iff \forall m, \omega \in A_{nm} \text{ (eventually)}\\
        &\iff \forall m, \omega \in \liminf_n A_{nm}
    \end{align}
\end{proof}

\begin{proposition}
    \begin{align*}
        R_n \ce{->[\text{a.s.}]} R \iff \forall \varepsilon > 0, \probability\qty(\{ \abs{R_k - R} \ge \varepsilon \text{ for some $k \ge n$}\}) \ce{->[$n \to \infty$]} 0
    \end{align*}
\end{proposition}
We can first prove a lemma!
\begin{lemma}
    If $R_n \ce{->[\text{a.s.}]} R$, then $R_n \ce{->[$\probability$]} R$.
\end{lemma}
\begin{proof}
    Let $\varepsilon > 0$. Then, for some $k \ge n$,
    \begin{align}
        &\probability(\abs{R_n - R} \ge \varepsilon)\\
        \le &\probability(\{ \abs{R_k - R} \ge \varepsilon \text{ for some $k \ge n$} \})
    \end{align}
    as $n\to\infty$, this reduces to
    \begin{align}
        \probability(\abs{R_n - R} \ge \varepsilon) \ce{->[$n \to \infty$]} 0
    \end{align}
    which is equivalent to
    \begin{align}
        R_n \ce{->[$\probability$]} R
    \end{align}
\end{proof}

\begin{lemma}
    The converse is not true, i.e. convergence in probability does not imply convergence in ``almost sure''.
\end{lemma}
\begin{proof}
    Take $R_n \sim \text{Bernoulli}(1/n)$ where $\text{Bernoulli}(x)$ is the result of a single Bernoulli trial with probability of success $x$. By definition, for $1 > \varepsilon > 0$,
    \begin{align}
        \probability(\abs{R_n - 0} \ge \varepsilon)
        &= \probability(R_n = 1)\\
        &= 1/n \ce{->[$n \to \infty$]} 0
    \end{align}
    However, $R_n$ does not ``almost surely'' converge.
    \begin{align}
        \probability(\{ \abs{R_k - R} \ge \varepsilon \text{ for some $k \ge n$} \})
        &= \probability(\{R_k = 1 \text{ for some $k \ge n$}\})\\
        &\ge \probability(\{ R_k = 1 \text{ for some $k \in [1,N]$, $N > n$} \}
    \end{align}
    This is a smaller set, so proving the statement for this finite case proves the converse not being true.
    \begin{align}
        &\ge \cdots\\
        &= 1 - \probability(R_k = 0 \text{ for all } k = n, n+1, \ldots, N)\\
        &= 1 - \qty(\frac{n-1}{n})\qty(\frac{n}{n+1})\cdots\qty(\frac{N-1}{N})\\
        &= 1 - \frac{n-1}{N} \ne 0
    \end{align}
    As $N \to \infty$, then
    \begin{align}
        \probability(\abs{R_k - 0} \ge \varepsilon \text{ for some $k \ge n$}) \ge 1 = 1
    \end{align}
    Thus, the probability does not go to zero, so the sets do not ``almost surely'' converge.
\end{proof}

\subsubsection{Proof of Proposition}
Here comes another lemma...
\begin{lemma}
    \begin{align}
        R_n \ce{->[\text{a.s.}]} R \iff \forall m \in \natural, \probability\qty(\liminf_n A_{nm}) = 1
    \end{align}
\end{lemma}
\begin{proof}
    \begin{enumerate}
        \item[$\implies$] This is direct
            \begin{align}
                R_n \ce{->[\text{a.s.}]} R
                &\implies \probability(\{ R_n \to R \}) = 1\\
                &\implies \probability\qty(\bigcap_{m=1}^\infty \liminf_n A_{nm}) = 1\\
                &\implies \forall m \in \natural, \probability\qty(\liminf_n A_{nm}) = 1
            \end{align}
        \item[$\impliedby$] This uses the last problem from the first midterm, which is in one of the Appendixes.
            \begin{align}
                \forall n, \probability(B_n) = 1 \implies \probability\qty(\bigcap_{n=1}^\infty B_n) = 1
            \end{align}
            Using this,
            \begin{align}
                \probability\qty(\bigcap_{m=1}^\infty \liminf A_{nm}) = 1
                &\implies \probability(\{ R_n \to R \}) = 1\\
                &\implies R_n \ce{->[\text{a.s.}]} R
            \end{align}
    \end{enumerate}
\end{proof}

Now we can prove the proposition.
\begin{align}
    R_n \ce{->[\text{a.s.}]} R &\iff \forall m, \probability\qty(\liminf_n A_{nm}) = 1\\
    &\iff \forall m, \lim_{n\to\infty} \probability\qty(\bigcap_{k=n}^\infty A_k) = 1 \qsp \text{expanding seq.}\\
    &\iff \probability\qty(\forall k \ge n \qsp \abs{R_k - R} < 1/m) \ce{->[$n \to \infty$]} 1\\
    &\iff \probability\qty(\exists k \ge n \qsp \abs{R_k - R} \ge 1/m) \ce{->[$n \to \infty$]} 0\\
    &\iff \probability\qty(\abs{R_k - R} \ge \varepsilon \text{ for some $k \ge n$}) \ce{->[$n \to \infty$]} 0
\end{align}


    \section{November 20, 2024}

\subsection{Some Recap}
Recall for $A_n \in \Omega$,
\begin{align}
    \limsup_n A_n &= \bigcap_{n=1}^\infty \bigcup_{k=n}^\infty A_k \\
    \liminf_n A_n &= \bigcup_{n=1}^\infty \bigcap_{k=n}^\infty A_k
\end{align}
$\omega \in \limsup$ refers to $\omega$ being in infinitely many $A_k$, while $\omega \in \liminf$ refers to $\omega$ eventually being in all $A_k$, i.e. for all $n \ge k$ (some constant $k$).


\subsection{Borel-Cantelli Lemma}
\begin{theorem}
    (\textbf{Borel-Cantelli Lemma}) Take events $A_1, A_2, \ldots$ on some probability space with converging $\sum_k \probability(A_k) < \infty$. Then,
    \begin{align}
        \probability\qty(\limsup_n A_n) = 0
    \end{align}
\end{theorem}

\begin{proof}
    The probability of $\limsup$ equals
    \begin{align}
        \probability\qty(\limsup_n A_n) &= \probability\qty(\bigcap_{n=1} \bigcup_{k=n} A_k)\\
        &\le \probability\qty(\bigcup_{k=n}^\infty A_k) \qsp \forall n \in \natural\\
        &\le \sum_{k=n}^\infty \probability(A_k) \ce{->[$n\to\infty$]} 0
    \end{align}
    because if the limit was not zero, then convergence would be a contradiction.
\end{proof}

\begin{aside}
    For a sum to converge, i.e.
    \begin{align}
        \sum_{n=1}^\infty a_n = s \implies s_N \equiv \sum_{n=1}^N a_n \ce{->[$n\to\infty$]} s
    \end{align}
    This means $s - s_N \ce{->[$N\to\infty$]} 0$, which implies that
    \begin{align}
        \sum_{n = N+1}^\infty a_n \ce{->[$N \to\infty$]} 0
    \end{align}
\end{aside}

\subsubsection{Consequences of Lemma}
Recall a sequence of random variables $R_1, \ldots$ converges almost surely to $R$ means
\begin{align}
    \probability(\{ R_n \to R \}) = 1
\end{align}
which, recall, is equivalent to
\begin{align}
    \probability\qty(\bigcap_{m=1}^\infty \liminf_n \{ \abs{R_n - R} < 1/m \}) = 1
\end{align}
which is also equivalent to
\begin{align}
    \forall m \in \natural, \probability\qty(\liminf_n \{ \abs{R_n - R} < 1/m \})
\end{align}

\begin{proposition}
    Suppose
    \begin{align}
        \sum_{n=1}^\infty \probability\qty(\abs{R_n - R} \ge \varepsilon) < \infty
    \end{align}
    for all $\varepsilon > 0$. Then, $R_n \ce{->[\text{a.s.}]} R$. Note the converse is not necessarily true.
\end{proposition}
\begin{proof}
    The finite sum
    \begin{align}
        \sum_{n=1}^\infty \probability\qty(\abs{R_n - R} \ge \varepsilon) < \infty
    \end{align}
    implies, via the Borel-Cantelli Lemma, that
    \begin{align}
        \probability\qty(\limsup_n \{ \abs{R_n - R} \ge \varepsilon \}) = 0
    \end{align}
    This is equivalent to
    \begin{align}
        \probability\qty[\qty(\limsup_n \{ \abs{R_n - R} \ge \varepsilon \})^C] = 1
    \end{align}
    
    \begin{aside}
        \begin{align}
            \qty(\limsup_n A_n)^C = \liminf_n A_n^C
        \end{align}
        because of De Morgan's Laws.
    \end{aside}

    Thus, for all $\varepsilon$,
    \begin{align}
        \probability\qty[\liminf_n \{ \abs{R_n - R} < \varepsilon \}] = 1
    \end{align}
    From this, for $\varepsilon = 1/m$ on all $m \in \natural$, it follows that $R_n \ce{->[\text{a.s.}]} R$.
\end{proof}

\begin{example}
    Take $R_1, R_2, \ldots$ independent with $R_i \sim \text{Bernoulli}(1/n)$, i.e. a single trial with probability of success $0 \le p \le 1$. Show that $R_n$ does not almost surely converge to $0$.
\end{example}
\begin{solution}
    Let $\varepsilon > 0$. Then,
    \begin{align}
        \sum_{n=1}^\infty \probability\qty(\abs{R_n - 0} > \varepsilon) = \sum_n 1/n = \infty
    \end{align}
    so the above-proved proposition does not apply.
\end{solution}

\begin{aside}
    \begin{align}
        \sum_n 1/n^\alpha
    \end{align}
    converges when $\alpha > 1$ and diverges when $\alpha \le 1$.
\end{aside}

\begin{example}
    Take $R_1, R_2, \ldots$ independent with $R_i \sim \text{Bernoulli}(1/n)$, i.e. a single trial with probability of success $0 \le p \le 1$. Define $S_n \equiv R_n R_{n+1}$. Show that $S_n \ce{->[\text{a.s.}]} 0$.
\end{example}
\begin{solution}
    Let $\varepsilon > 0$. Then,
    \begin{align}
        \sum_{n=1}^\infty \probability\qty(\abs{R_n - 0} > \varepsilon) = \sum_n 1/n(n+1) < \infty
    \end{align}
    Thus, the above-proved proposition shows that $S_n \ce{->[\text{a.s.}]} 0$.
\end{solution}

\begin{proposition}
    For some sequence of random variables $R_1, \ldots$, if
    \begin{align}
        \sum_{n=1}^\infty E\qty((R_n - R)^k) < \infty
    \end{align}
    for some $k > 0$, then $R_n \ce{->[\text{a.s.}]} R$.
\end{proposition}
\begin{proof}
    We can apply Chebyshev's Inequality
    \begin{align}
        \sum_{n=1}^\infty \probability(\abs{R_n - R} \ge \varepsilon) \le \sum_{n=1}^\infty \frac{E(R_n - R)^k}{\varepsilon^k} < \infty
    \end{align}
    and then the previous proposition.
\end{proof}


\subsection{Strong Law of Large Numbers}
\begin{aside}
    Weak LLN is convergence in probability. Strong LLN is almost sure convergence.
\end{aside}

\begin{theorem}
    (\textbf{Strong Law of Large Numbers}) For $R_1, R_2, \ldots$ independent r.v. and $E((R_i - E(R_i))^4) < M$, i.e. 4th central moments are bounded for all $i$. Let
    \begin{align}
        S_n = R_1 + \cdots + R_n
    \end{align}
    Then,
    \begin{align}
        \frac{S_n - E(S_n)}{n} \ce{->[\text{a.s.}]} 0
    \end{align}
\end{theorem}

\begin{proof}
    First, assume WLOG that $E(R_i) = 0$. It suffices to show that
    \begin{align}
        \sum_{n=1}^\infty E\qty[\qty(\frac{S_n}{n})^k] < \infty
    \end{align}
    for some $k$, as then by the previous propositions, there is almost sure convergence. It just so happens that this works for $k = 4$, i.e.
    \begin{align}
        \sum_{n=1}^\infty E\qty[\qty(\frac{S_n}{n})^4] < \infty
    \end{align}
    What is $S_n^4$?
    \begin{align}
        S_n^4 &= \qty(R_1 + \cdots + R_n)^4\\
        &= \sum_{j} R_j^4 + \binom{4}{2} \sum_{j,k} R_j^2 R_k^2 + \frac{4!}{2!1!1!} \sum_{j \ne k,l} R_j^2 R_k R_l\\
        &+ 4! \sum_{j,k,l,m} R_jR_kR_lR_m + \binom{4}{3} \sum_{j,k} R_j^3 R_k
    \end{align}
    The expectation of $S_n^4$ thus equals
    \begin{align}
        \sum_{j}^n E(R_j^4) + \binom{4}{2} \sum_{j,k} E(R_j^2) E(R_k^2)
    \end{align}
    because all first-order expectations equal zero, or $E(R_i) = 0$ for all $i$. This $\le$
    \begin{align}
        nM + \binom{4}{2} \sum_{j,k} E(R_j^2) E(R_k^2)
    \end{align}
    By Cauchy Schwarz, this $\le$
    \begin{align}
        nM + \kappa \sum_{j,k} \sqrt{E(R_j^4)} \sqrt{E(R_k^4)}
    \end{align}
    which is again bounded by $M$, i.e. converges. Thus,
    \begin{align}
        E\qty[\qty(\frac{S_n}{n})^4] = \frac{1}{n^4} E\qty(S_n^4) \le nM + 6\frac{n(n-1)}{2}M
    \end{align}
    Thus,
    \begin{align}
        \sum_{n=1}^\infty E\qty[\qty(\frac{S_n}{n})^4] \le M \sum_{n=1}^\infty \qty(\frac{1}{n^3} + \frac{3}{n^2}) < \infty
    \end{align}
    which using previous propositions, shows almost sure convergence.
\end{proof}


    \section{November 22, 2024}

\subsection{Cauchy Distribution}
There is a case where LLN fails for $R_1, \ldots, R_n$ independent, i.e.
\begin{align}
    \frac{R_1 + \cdots + R_n}{n} \ce{->[$d$]} \text{constant}
\end{align}
does not hold.

\begin{definition}
    \textbf{(Cauchy Distribution)} A random variable $R$ that is Cauchy distributed has density
    \begin{align}
        f_R(x) = \frac{1}{\pi(1+x^2)}
    \end{align}
    The second moment $E(R^2) = \infty$.
\end{definition}

The characteristic function of this equals
\begin{align}
    M_R(u) = \int_{-\infty}^\infty e^{-iux} \frac{1}{\pi(1 + x^2)} = e^{-\abs{u}}
\end{align}
by doing a standard contour integral and maybe using the residue theorem.

\begin{example}
    Take $R_1, R_2, \ldots$ i.i.d. Cauchy. Then,
    \begin{align}
        M_{R_1 + \cdots + R_n}(u) = e^{-n\abs{u}}
    \end{align}
    So,
    \begin{align}
        M_{\frac{R_1 + \cdots + R_n}{n}}(u) = M_{R_1 + \cdots + R_n}\qty(\frac{u}{n}) = e^{-n\abs{u/n}} = e^{-\abs{u}}
    \end{align}
    This indicates that
    \begin{align}
        \frac{R_1 + \cdots + R_n}{n} \ce{->[$d$]} R_1 \qsp \text{it is Cauchy, not constant}
    \end{align}
    There is no law of large numbers for Cauchy because it goes to infinity very slowly.
\end{example}

\subsection{An Interesting Problem}
\begin{proposition}
    Take $R_1, R_2, \ldots$ i.i.d. a sequence of random variables, where $R_i \sim \text{Bernoulli}(1/2)$. Take
    \begin{align}
        S_n \equiv \sum_{j=1}^n \frac{R_j}{2^j}
    \end{align}
    Then, 
    \begin{align}
        S_n \ce{->[$d$]} Z \equiv \text{Unif}[0,1]
    \end{align}
\end{proposition}
\begin{solution}
    First, compute $M_Z(u)$.
    \begin{align}
        M_Z(u) &= E\qty(e^{-iuZ}) \int_{0}^1 e^{-iux} \dd{x}\\
        &= \frac{1 - e^{-iu}}{iu}
    \end{align}
    We want to show that $M_{S_n}(u) \longrightarrow \frac{1 - e^{-iu}}{iu}$ for all $u \in \real$.
    \begin{align}
        M_{S_n}(u) = M_{\Sigma}(u) &= \prod_{j=1}^n M_{R_j/2^j}(u)\\
        &= \sum_x e^{-iux} \probability(x)\\
        &= \frac{1}{2}\qty(1 + e^{-iu/2^j})
    \end{align}
    Thus,
    \begin{align}
        M_{S_n}(u) = \frac{1}{2^n} \qty(1 + e^{-iu/2}) \qty(1 + e^{-iu/2^2}) \cdots \qty(1 + e^{-iu/2^n})
    \end{align}

    \begin{lemma}
        As a convenient hint,
        \begin{align}
            1 + e^{-iu/2^j} = \frac{e^{-iu/2^{j-1}} - 1}{e^{-iu/2^j} - 1}
        \end{align}
        \begin{proof}
            Ben proved it on the board but it is not particularly interesting enough to copy into here.
        \end{proof}
    \end{lemma}

    Thus,
    \begin{align}
        M_{S_n}(u) = \frac{1}{2^n} \qty(\frac{e^{-iu} - 1}{e^{-iu/2} - 2})\qty(\frac{e^{-iu/2^{n-1}} - 1}{e^{-iu/2^n} - 2})
    \end{align}
    These cascadingly cancel into
    \begin{align}
        M_{S_n}(u) = \frac{1}{2^n} \frac{e^{-iu} - 1}{e^{-iu/2^n} - 1}
    \end{align}
    We want to show that this converges to $(1 - e^{-iu})/iu$ as $n \to \infty$. This can be achieved with a Taylor expansion:
    \begin{align}
        \frac{1}{2^n}\frac{e^{-iu} - 1}{e^{-iu/2^n} - 1} = \frac{e^{-iu} - 1}{2^n\qty(1 - iu/2^n + \cdots - 1)}
    \end{align}
    which equals
    \begin{align}
        \frac{e^{-iu} - 1}{-iu + 2^nO\qty[\qty(\frac{iu}{2^n})^2]} \ce{->[$n \to \infty$]} \frac{1 - e^{-iu}}{iu}
    \end{align}
\end{solution}

\subsection{Banach's Matchbox Problem}
(smoked pipe)
\begin{proposition}
    Banach has two matchboxes: one in his left pocket and one in his right pocket; both have $N$ matches. Every time Banach wants to smoke, he takes a match randomly and independently from one of the two boxes. At some point when Banach tries to take a match, the box he picks will be empty. What is the probability that the non-empty box has exactly $k \in \natural_0$ matches?
\end{proposition}

\begin{solution}
    WLOG, to get to the premise of the problem, Banach has taken a total of $N$ matches from the right and $N - k$ from the left for a total of $2N - k$ matches. The next match then must be the right (which has probability $1/2$). Now this reduces to two Binomial distributions:
    \begin{align}
        \binom{2N - k}{N} \qty(\frac{1}{2})^N \qty(\frac{1}{2})^{N-k} \qty(\frac{1}{2})\qty(2)
    \end{align}
    Thus, the probability equals
    \begin{align}
        \binom{2N - k}{N}\qty(\frac{1}{2})^{2N - k}
    \end{align}
    which is the \textbf{negative binomial distribution}.
\end{solution}

\begin{definition}
    \textbf{(Negative Binomial Distribution)} Take independent trials with probability of success $p$. Let $R$ be the number of trials until $r$ successes. Then,
    \begin{align}
        R \equiv \text{NB}(r,p) \implies \probability(R = n) &= \binom{n-1}{r-1} p^{r-1} (1-p)^{n-1-(r-1)}\\
        &= \binom{n-1}{r-1}p^r(1-p)^{n-r}
    \end{align}
\end{definition}

\subsection{Exponential Example}
Take $R,S \sim \exp(1)$. Show that the distribution of $R$ given that $R + S = z$ is $\text{Unif}[0,z]$. We want
\begin{align}
    f_{R \mid R + S}(R = x \mid R + S = z)
\end{align}
but if we flip this we get
\begin{align}
    f_{R + S \mid R}(R + S = z \mid R = x)
\end{align}
which is trivial and just equals
\begin{align}
    f_S(z - x) = \begin{cases}
        e^{-(z-x)} & z-x \ge 0\\
        0 & \text{else}
    \end{cases}
\end{align}

To solve the rest of this problem, we can use the continuous Bayes' Theorem:
\begin{proposition}
    \textbf{(Continuous Bayes' Theorem)}
    \begin{align}
        f_{Y \mid X}(y \mid x) f_X(x) = f_{X,Y}(x,y) = f_Y(y)f_{X \mid Y}(x \mid y)
    \end{align}
\end{proposition}

The rest of this problem is left as an exercise for the reader. It is quite trivial, but some densities may need to be computed. Eventually, we get
\begin{align}
    f_{R \mid R + S}(R = x \mid R + S = z) = \frac{e^{-(z-x)e^{-x}}}{ze^{-z}} = \frac{1}{z}
\end{align}
which is just the density of $\text{Unif}[0,z]$.


    \section{November 25, 2024}
More examples...

\subsection{Sticks}
\begin{proposition}
    Randomly and uniformly break a unit stick (length 1) into two pieces: a long piece and a short piece. Find the expected value of the ratio of the two pieces.
\end{proposition}
\begin{solution}
    Let $R$ be the short side. Then, $R \sim \text{Unif}[0,1/2]$. We then want to compute the ratio
    \begin{align}
        \frac{R}{1-R}
    \end{align}
    which equals
    \begin{align}
        \int_0^1 \frac{x}{1-x} f_R(x) \dd{x} &= 2 \int_0^{1/2} \frac{x}{1-x} \dd{x}\\
        &= 2 \int_0^{1/2} \qty(\frac{1}{1-x} - 1) \dd{x}\\
        &= 2 \qty[ \eval{ -x - \ln(1-x)}_0^{1/2} ] = \boxed{\ln(4) - 1}
    \end{align}
\end{solution}

\subsection{Another Example}
\begin{proposition}
    Take $Y \sim \text{Unif}[0,1]$. Given $Y = y$, let $X \sim \text{Binomial}(n,y)$. Find $E(X)$ and $\text{var}(X)$.
\end{proposition}
\begin{solution}
    The Theorem of Total Expectation can be used.
    \begin{align}
        E(X) &= \int_Y E(X \mid Y = y) f_Y(y) \dd{y}\\
        &= \int_0^1 ny \cdot 1 \dd{y} = \boxed{\frac{n}{2}}
    \end{align}
    What about the variance? We can compute $E\qty(X^2)$ which equals
    \begin{align}
        E\qty(X^2) = \int_0^1 E(X^2 \mid Y = y) \cdot 1 \dd{y}
    \end{align}
    Note that $\text{var}(X) = E\qty(X^2) - E(X)^2 = np(1-p)$. Thus, $E\qty(X^2) = (np)^2 + np(1-p)$. Hence,
    \begin{align}
        E\qty(X^2) = \int_0^1 n^2y^2 + ny - ny^2 \dd{y} = \boxed{\frac{n^2}{3} + \frac{n}{2} - \frac{n}{3}}
    \end{align}
    It follows that $\var(X)$ equals
    \begin{align}
        \text{var}(X) &= \frac{n^2}{3} + \frac{n}{2} - \frac{n}{3} - \qty(\frac{n}{2})^2\\
        &= \frac{n(n + 2)}{12}  
    \end{align}
\end{solution}


\subsection{Yet Another Example}
\begin{proposition}
    Let $R \sim \text{Geom}(p)$, i.e. the number of independent trials until a success given probability of success $p$. Show that $E(R) = 1/p$ and $\text{var}(R) = (1-p)/p^2$ by using conditional expectation on $S \equiv I_{\text{1st trial success}}$.
\end{proposition}
\begin{solution}
    We can again use the Theorem of Total Expectation.
    \begin{align}
        E(R) &= E(R \mid S = 0) \probability(S = 0) + E(R \mid S = 1) \probability(S = 1)\\
        &= (E(R) + 1)(1-p) + (1)(p)\\
        &= p + E(R) - p + 1 - pE(R)\\
        &= E(R) (1 - p) + 1
    \end{align}
    It follows that
    \begin{align}
        pE(R) = 1 \implies \boxed{E(R) = 1/p}
    \end{align}
    In a similar manner comes the variance. First,
    \begin{align}
        E(R^2) &= E(R^2 \mid S = 0) \probability(S = 0) + E(R^2 \mid S = 1) \probability(S = 1)\\
        &= E(R^2 \mid S = 0) (1-p) + p\\
        &= E\qty((R+1)^2)(1-p) + p\\
        &= E(R^2 + 2R + 1)(1-p) + p\\
        &= \qty[E(R^2) + E(2R) + 1](1-p) + p
    \end{align}
    Thus,
    \begin{align}
        p E(R^2) = \frac{2(1-p)}{p} + 1 \implies \boxed{E(R^2) = \frac{2(1-p)}{p^2} + \frac{1}{p}}
    \end{align}
    From this,
    \begin{align}
        \text{var}(R) = \frac{2}{p^2} - \frac{2}{p} + \frac{1}{p} - \frac{1}{p^2} = \boxed{\frac{1-p}{p^2}}
    \end{align}
\end{solution}

\subsection{ANOTHER Example}
\begin{proposition}
    Let $X_1, X_2, X_3 \sim \text{Bernoulil}(p)$ (independent). Define $Y_1 = \text{max}(X_1, X_2)$, $Y_2 = \text{max}(X_1, X_3)$, and $Y_3 = \text{max}(X_2, X_3)$. Set $Y = Y_1 + Y_2 + Y_3$. Find $E(Y)$ and $\text{var}(Y)$.
\end{proposition}
\begin{solution}
    Clearly,
    \begin{align}
        E(Y) = E(Y_1) + E(Y_2) + E(Y_3) \ce{->[symmetry]} 3E(Y_1)
    \end{align}
    That expectation equals
    \begin{align}
        E(Y_1) = 0P_{Y_1}(0) + 1\qty[p^2 + 2p(1-p)] = p(2-p)
    \end{align}
    It follows that $E(Y) = 3p(2-p)$. For variance,
    \begin{align}
        E(Y^2) &= E((Y_1 + Y_2 + Y_3)^2) = E(Y_1^2 + Y_2^2 + Y_3^2 + 2Y_1Y_2 + 2Y_2Y_3 + 2Y_1Y_3)\\
        &= E(Y_1) + E(Y_2) + E(Y_3) + E(2Y_1Y_2 + 2Y_2Y_3 + 2Y_1Y_3)\\
        \ce{->[symmetry]} &= 3p(2-p) + 6E(Y_1Y_2)
    \end{align}
    which can eventually solve for the variance. The solution is straightforward but annoying and Ben kind of just... gave up.
\end{solution}


\subsection{An Example with Graphs}
\begin{proposition}
    Take a fully connected graph with $n$ nodes. Given $k = 1, 2, \ldots, n$, can the edges be denoted $A$ or $B$ such that no set of $k$ vertices has all its faces with edges of all the same color?
\end{proposition}
\begin{solution}
    Let $S$ be a set of all $\binom{n}{k}$ subsets of $k$ vertices. Let $E_i$ be the event where all of the $\binom{k}{2}$ edges connecting the $i$-th subset of vertices have the same color. What is $\probability(E_i)$?
    \begin{align}
        \probability(E_i) = 2 \cdot \qty(\frac{1}{2})^{\binom{k}{2}} = \qty(\frac{1}{2})^{\binom{k}{2} - 1}
    \end{align}
    We want the probability
    \begin{align}
        \probability\qty(\bigcup_{i=1}^{\binom{n}{k}} E_i) \le \sum_{i=1}^{\binom{n}{k}} \qty(\frac{1}{2})^{\binom{k}{2} - 1} \stackrel{?}{\le} 1
    \end{align}
    If $n \gg k$, then this inequality holds, i.e. there can exist some permutation of colors such that the desired coloring is possible. The answer is that it depends.
\end{solution}



    % \section{Appendix C: Practice Final}
Below are the practice problems that were provided for the final exam.

\subsection{Problem 1}
For each of the following, give a precise definition of the term, or formally state the result (whichever is applicable):
\begin{enumerate}
    \item Probability space
    \item Random variable
    \item Three independent random variables
    \item Chebyshev's inequality
    \item Convergence in distribution
    \item Central Limit Theorem
    \item Convergence in probability
\end{enumerate}
\begin{solution}
    \begin{enumerate}
        \item A probability space is a triplet $\psp$ consisting of a sample space $\Omega$ of all possible outcomes of an experiment, an event space $\curlyf$ which is a sigma algebra of subsets of $\Omega$ (i.e. events to consider), and a probability measure $\probability: \curlyf \to [0,1]$ that measures the probabilities of events.
        \item A random variable is a function $R: \Omega \to \real$ such that the inverse image $R^{-1}(B) \in \curlyf$ for all Borel sets $B \in \mathcal{B}$.
        \item Three independent random variables on some probability space $\psp$ are $R_1,R_2,R_3$ such that
            \begin{align}
                \probability\qty(R_1 \in F_1 \qand R_2 \in F_2 \qand R_3 \in F_3) = \prod_{i=1}^3 \probability\qty(R_i \in F_i)
            \end{align}
            for elements $F_1,F_2,F_3 \in \curlyf$.
        \item Chebyshev's inequality has three parts
            \begin{enumerate}
                \item For $R$ r.v., $b \in \real$,
                    \begin{align}
                        \probability(R \ge b) \le E(R)/b
                    \end{align}
                \item For $R$ r.v., $c$ and $l$ constants,
                    \begin{align}
                        \probability(\abs{R - c} \ge \varepsilon) \le \frac{E\qty(\abs{R-c}^l)}{\varepsilon^l}
                    \end{align}
                    is true for all $\varepsilon > 0$
                \item For $R$ r.v. with finite variance $\sigma^2 > 0$, finite mean $m$, and $k$ constant,
                    \begin{align}
                        \probability(\abs{R-m} \ge k\sigma) \le \frac{1}{\sigma^2}
                    \end{align}
            \end{enumerate}
        \item Convergence in distribution for a sequence of random variables ($R_n \ce{->[$d$]} R$) if
            \begin{align}
                F_{R_n}(x) \to F_R(x)
            \end{align}
            for all $x$ on which $F_R$ is continuous.
        \item Central Limit Theorem. For i.i.d. random variables $R_1, \ldots$ with finite mean $m$, variance $\sigma^2$, and third moment,
            \begin{align}
                \sum_{i=1}^n \frac{R_i - nm}{\sigma\sqrt{n}} \ce{->[$d$]} N(0,1)
            \end{align}
        \item Convergence in probability for a sequence of random variables ($R_n \ce{->[$\probability$]} R$) if for all $\varepsilon > 0$
            \begin{align}
                \probability(\abs{R - R_n} \ge \varepsilon) \ce{->[$n \to \infty$]} 0
            \end{align}
    \end{enumerate}
\end{solution}

\subsection{Problem 2}
There are $k$ biased coins in a box, labelled $1$ through $k$. The probability that the $i$-th coin lands on heads is $i/k$. A coin is randomly selected from the box and is repeatedly flipped. If the first $n$ flips all result in heads, what is the conditional probability that the $(n+1)$-th flip will also land on heads?
\begin{solution}
    The conditional probability
    \begin{align}
        \probability\qty((n+1)\text{-th is heads} \mid \text{first $n$ all heads})
    \end{align}
    equals
    \begin{align}
        \frac{\probability(\text{first $n+1$ heads})}{\probability(\text{first $n$ heads})}
    \end{align}
    The law of total probability gives the probability of $n$ heads for some coin as
    \begin{align}
        \probability(A_n) = \sum_{i=1}^k \qty(\frac{i}{k})^n \frac{1}{k}
    \end{align}
    Thus, the conditional probability equals
    \begin{align}
        \frac{\sum_{i=1}^k \qty(\frac{i}{k})^{n+1}}{\sum_{i=1}^k \qty(\frac{i}{k})^n}
    \end{align}
\end{solution}

\subsection{Problem 3}
Let $n \in N$ and consider the set $S$ of all permutations of the integers $\{1, 2, \ldots , n\}$. For $k \in \{1, 2, \ldots , n\}$, we say a permutation $f \in S$ has fixed point $k$ if and only if $f(k) = k$. Choose a permutation uniformly at random from $S$ and consider the random variable $R$ that gives the number of its fixed points. Compute $E(R)$ and $\text{Var}(R)$.
\begin{solution}
    We can write using indicators:
    \begin{align}
        R = I_1 + I_2 + \cdots + I_n
    \end{align}
    each has a probability $1/n$ of being a fixed point by independence. Thus,
    \begin{align}
        E(R) = 1 && \text{Var}(R) = E(R^2) - 1
    \end{align}
    Then,
    \begin{align}
        E(R^2) = n(1/n) + (n^2-n)\qty(\frac{1}{n-1}\frac{1}{n}) = 2
    \end{align}
    Thus, Var$(R) = 1$.
\end{solution}

\subsection{Problem 4}
Let $R_1, R_2, \ldots$ be i.i.d. random variables with finite mean and finite variance. State the Weak Law of Large Numbers for these conditions and prove it using Chebyshev’s inequality.
\begin{solution}
    Define $S_n \equiv \sum_i^n R_i$. Then,
    \begin{align}
        \probability\qty[\frac{S_n - E(S_n)}{n} \ge \varepsilon] \ce{->[$n \to \infty$]} 0
    \end{align}
    The proof is in Section 15.
\end{solution}

\subsection{Problem 5}
Let $R$ and $S$ be independent Bernouilli random variables with parameter $p = 1/2$. Define $Z = R + S$ and $W = \abs{R - S}$. Prove that $\text{Cov}(Z, W) = 0$ but $Z$ and $W$ are not independent.
\begin{solution}
    For $Z = R+S$, the event space is $0$, $1$, and $2$ with probabilities $1/4$, $1/2$, $1/4$. For $W = \abs{R - S}$, the event space is $0$ and $1$ with probabiltiies $1/2$ and $1/2$/ It can be shown that
    \begin{align}
        \text{Cov}(Z,W) = 0
    \end{align}
    but $\probability(Z = 0, W = 0) = 1/4$ while $\probability(Z = 0)\cdot \probability(W = 0) = 1/4 \cdot 1/2 = 1/8$.
\end{solution}

\subsection{Problem 6}
Let $X$ have the exponential distribution with parameter 3 and let $Y = e^X$.
\begin{enumerate}
    \item[(a)] Find the expectation and variance of $Y$.
    \item[(b)] For $Y_1, Y_2, \ldots$ i.i.d. with the same distribution as $Y$, what is the approximate distribution of $(Y_1 + Y_2 + \cdots + Y_n)/\sqrt{n}$ when $n$ is large?
    \item[(c)] Is there a random variable $S$ such that
    \begin{align}
        \frac{Y_1 + Y_2 + \cdots + Y_k}{k} \ce{->[$\probability$]} S
    \end{align}
    as $k \rightarrow \infty$? Justify your answer. If you answer is yes, specify $S$.
\end{enumerate}
\begin{solution}
    
\end{solution}

\subsection{Problem 7}
Let $R$ have density $f_R(r) = 1/r^2$ for $r \ge 1$. Given $R$, let $U$ be uniformly distributed on $[0,R]$.
\begin{enumerate}
    \item[(a)] Find $E(U \mid R = r)$.
    \item[(b)] Find the marginal density $f_U$ of $U$.
    \item[(c)] Compute the conditional density of $R$ given $U = u$.
\end{enumerate}
\begin{solution}
    
\end{solution}

\subsection{Problem 8}
Let $U$ be uniformly distributed on $[0, 1]$. For $n = 1, 2, \ldots$, define random variables $R_n$ by
\begin{align}
    R_n = UI_{\{ U \le 1 - (1/n) \}}
\end{align}
\begin{enumerate}
    \item[(a)] Show that $R_n$ converges to $U$ almost surely.
    \item[(b)] Let $X_n$ be i.i.d. $X_n \sim \text{Bernouilli}(1 - (1/n))$, also independent of $U$. Define $Y_n = U X_n$. Does $Y_n$ have the same distribution as $R_n$? Why or why not?
    \item[(c)] Show that $Y_n$ converges in probability to $U$.
    \item[(d)] Show that $Y_n$ does not converge to $U$ almost surely.
\end{enumerate}
\begin{solution}

\end{solution}


\subsection{Problem 9}
You break a rod of length $1$ uniformly at random in two pieces. Then take the shorter of the two pieces and break it again uniformly at random. Let $S$ be the length of the shortest of the three pieces. What is the density of $S$?
\begin{solution}

\end{solution}



\end{document}
