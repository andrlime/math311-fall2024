\section{September 25, 2024}
Recall that a probability space is defined as a triplet $(\Omega, \mathcal{F}, \mathbb{P})$. Further recall the three conditions that define a $\sigma$-algebra. Finally, recall the definition of a probability measure. Using these, we can define some properties.

\subsection{Probability Space Properties}

\subsubsection{Properties of a Sigma Algebra}
Let $\mathcal{F}$ be a $\sigma$-algebra. Then,
\begin{enumerate}
    \item $\emptyset \in \mathcal{F}$ which is proven in Lemma 1.6.
    \item Closedness under union
    \begin{align}
        A_1, \ldots, A_N \in \curlyf \implies \bigcup_i^N A_i \in \curlyf
    \end{align}
    \begin{proof}
        Using $A_{N+1}, ... \equiv \emptyset$, the union of all of these must be an element of $\curlyf$
    \end{proof}
    \item Closedness under intersection
    \begin{align}
        A_1, \ldots, A_N \in \curlyf \implies \bigcap_i^N A_i \in \curlyf
    \end{align}
    \begin{proof}
        Take
        \begin{align}
            A_1, \ldots, A_N \in \curlyf 
        \end{align}
        Then, by definition,
        \begin{align}
            A_1^C, \ldots, A_N^C \in \curlyf 
        \end{align}
        By definition and then De Morgan's Laws,
        \begin{align}
            A_1^C, \ldots, A_N^C \in \curlyf  \implies  \bigcup A_i^C \in \curlyf \implies \left( \bigcap A_i \right)^C \in \curlyf \implies \bigcap A_i \in \curlyf & \qedhere
        \end{align}
    \end{proof}
\end{enumerate}

\subsubsection{Generated Sigma Algebras}
Let $\Omega \equiv \natural$. Then,
\begin{align}
    \curlyf \equiv \{ \emptyset, \{ 1 \}, \{ 1, 2 \}, ... \}
\end{align}
is not a $\sigma$-algebra because $\{ 1, 2 \} - \{ 1 \} = \{ 2 \}$ is not an element of $\curlyf$. In spite of this, we can define a $\sigma$-algebra $\widetilde{\curlyf}$ to be the intersections of all $\sigma$-algebras that contain $\curlyf$, i.e. for some non-sigma-algebra subset of sets $\curlyf$,
\begin{align}
    \widetilde{\curlyf} \equiv \bigcap \mathcal{G} \qsp \forall \mathcal{G}_\sigma \supset \mathcal{F}
\end{align}
\begin{proposition}
    $\widetilde{\curlyf}$ is a $\sigma$-algebra.
\end{proposition}
For the case in (2.5) specifically, we assert that $\widetilde{\curlyf} \equiv 2^\natural$, i.e. the power set of natural numbers. To show this, see that if $\{ 1 \} \in \curlyf$ then $\{ 2 \} \in \curlyf$ for $\{ 1, 2 \} \in \curlyf$. For similar reasons,
\begin{align}
    \{ n \} \in \curlyf \qsp \forall n \in \natural
\end{align}
and by taking the union of these sets, all subsets of $\natural$ can be composed of these singleton subsets.

\subsubsection{Properties of a Probability Measure}
\begin{enumerate}
    \item The probability of nothing... is nothing!
    \begin{align}
        \probability(\emptyset) = 0
    \end{align}
    \begin{proof}
        \begin{align}
            \probability(\emptyset \cup \Omega) = \probability(\Omega) = \probability(\emptyset) + \probability(\Omega) \implies \probability(\emptyset) = 0
        \end{align}
    \end{proof}
    \item Probability of unions
    \begin{align}
        \probability(A \cup B) = \probability(A) + \probability(B) - \probability(A \cap B)
    \end{align}
    To prove this without using pictures, we can express $A$, $B$, and $A \cup B$ as disjoint sets
    \begin{align}
        A \cup B &= (A - B) \cup (A \cap B) \cup (B - A)\\
        A &= (A - B) \cup (A \cap B)\\
        B &= (B - A) \cup (A \cap B)
    \end{align}
    This means that
    \begin{align}
        \probability(A) + \probability(B) = \probability(A - B) + \probability(B - A) + 2 \probability(A \cap B)
    \end{align}
    and clearly, there is one extra $\probability(A \cap B)$.
    \item Subset probability. Given $B \subseteq A$, $\probability(A - B) = \probability(A) - \probability(B)$ which implies $\probability(B) \le \probability(A)$ due to non-negative probabilities being not possible by definition. We can write $A$ and $B$ as disjoint sets,
    \begin{align}
        A &= B \cup (A-B)\\
        B &= B
    \end{align}
    and then
    \begin{align}
        \probability(A) &= \probability(B) + \probability(A - B)\\
        \probability(B) &= \probability(B)\\
        \therefore \probability(A) - \probability(B) &= \probability(A - B) (\ge 0) & \qedhere
    \end{align}
    \item Union probability less than sum
    \begin{align}
        \probability\left( \bigcup A_i \right) \le \sum \probability(A_i)
    \end{align}
    \begin{proof}
        We can again write these in a different way
        \begin{align}
            \bigcup A_i = (A_1) \cup (A_1^C \cap A_2) \cup (A_1^C \cap A_2^C \cap A_3) \cup \cdots
        \end{align}
        Note that by Property 3, $\probability(A_1^C \cap A_2) \le \probability(A_2)$ and the same applies to all of the other ones. So,
        \begin{align}
            \probability\left( \bigcup A_i \right) \le \probability(A_1) + \probability(A_2) + \cdots & \qedhere
        \end{align}
    \end{proof}
\end{enumerate}


\subsection{Combinatorics and Counting}
Take $\Omega = \{ a_1, \cdots, a_N \}$ as some sample space with $\abs{\Omega} = N$. Take $\curlyf \equiv 2^\Omega$ as all subsets of $\Omega$ and define
\begin{align}
    \probability(a_i) = 1/N \qsp \forall i
\end{align}
which generalizes to
\begin{align}
    \probability(A) = \dfrac{\abs{A}}{\abs{\Omega}}
\end{align}
To compute $\probability$, we have to be able to count $\abs{A}$, which requires an overview of combinatorics. 

\subsubsection{Ordered with Replacement}
Suppose we have a license plate with five letters. Then, there are $26^5$ possible combinations because we can reuse letters, and the order matters. In general, for a set of size $N$ and $R$ repeats, there are
\begin{align}
    N^R \qsp \text{permutations}
\end{align}

\subsubsection{Ordered without Replacement}
If we do not replace then we cannot reuse letters. So, for the license plate we have $26 \cdot 25 \cdots 22$ combinations. In general, for $N$ and $R$, we have
\begin{align}
    \frac{N!}{(N-R)!} = \, ^NP_R = (N)_R \qsp \text{permutations}
\end{align}

% \subsubsection{Unordered with Replacement}
% Unordered means there are $N!$ permutations with the same set of characters, i.e. we are overcounting by $N!$. So, we have
% \begin{align}
%     \frac{N^R}{R!} \qsp \text{permutations}
% \end{align}

\subsubsection{Unordered without Replacement}
By the same logic, we now have
\begin{align}
    \frac{N!}{R!(N-R)!} = \binom{N}{R} \qsp \text{permutations}
\end{align}

