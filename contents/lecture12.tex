\section{October 18, 2024}

\subsection{Recap: Random Variables}
Recall that a random variable $R$ on some probability space $\psp$ is a function
\begin{align}
    R: \Omega \to \real
\end{align}
\begin{example}
    Flip one coin. Then
    \begin{align}
        R = \begin{cases}
            1 & \text{heads}\\
            0 & \text{tails}
        \end{cases}
    \end{align}
    Then, the probability distribution is
    \begin{align}
        F_R(x) = \probability(R \le x) = \begin{cases}
            0 & x < 0\\
            1/2 & x \in [0,1)\\
            1 & x \ge 1
        \end{cases}
    \end{align}
    This can also be thought of as $\Omega = \{ H, T \}$, and
    \begin{align}
        R(\Omega) = \{ H \to 1, T \to 0 \}
    \end{align}
\end{example}

\begin{example}
    Roll a dice. Let
    \begin{align}
        R = \begin{cases}
            1 & \text{even}\\
            0 & \text{odd}
        \end{cases}
    \end{align}
    Then, $R$ can also be expressed as
    \begin{align}
        R = \{ 1 \to 0, 2 \to 1, \cdots \}
    \end{align}
    The probability distribution is the same as in Equation 12.3.
\end{example}

\noindent Take $R$ as a random variable with a density $f(x)$. Note that the probability space here is not specified, because there are many random variables with the same density $f$. \textbf{Choose} the probability space $(\real, \mathcal{B}, \probability)$ such that
\begin{align}
    \probability(B \in \mathcal{B}) = \int_B f(x) \dd{x}
\end{align}
Then we can define
\begin{align}
    R: \real \to \real \equiv R(x) = x
\end{align}
Then the probability
\begin{align}
    \probability(R \le z) = \int_{x \le z} f(x) \dd{x}
\end{align}

\begin{aside}
    If a random variable has a density function, it is assumed to be absolutely continuous.
\end{aside}

Let $R_1, \ldots, R_n$ be independent, absolutely continuous random variables with densities $f_1(x), \ldots, f_n(x)$. Take the probability space $(\real^n, \mathcal{B}, \probability)$ where
\begin{align}
    \probability(B) = \int\cdots\int_B f_1(x_1)\cdots f_n(x_n) \dd{x_1}\cdots\dd{x_n}
\end{align}
Then,
\begin{align}
    R_1(x_1, \ldots, x_n) &= x_1\\
    R_n(x_1, \ldots, x_n) &= x_n
\end{align}

\subsection{Median}
\begin{definition}
    Let $R$ be a random variable with distribution function $F_R(x)$. Assume absolute continuity. Then, $F_R(x)$ is continuous. Then, the median is defined as
    \begin{align}
        m \in \real \mid f_R(m) = \frac{1}{2}
    \end{align}
    which is equivalent to
    \begin{align}
        \probability(R \le m) = \frac{1}{2} = \probability(R > m)
    \end{align}
\end{definition}

\noindent For general, not-necessarily-absolutely-continuous $R$,
\begin{definition}
    The median $m \in \real$ of a random variable $R$ satisfies
    \begin{align}
        F_R(m) \ge 1/2 \qsp F_R(\ce{$m$-}) \le 1/2
    \end{align}
\end{definition}
\begin{example}
    Take Example 12.1. The median is any number $m \in [0,1]$.
    \begin{proof}
        Let $x \in [0,1]$. Then,
        \begin{align}
            F_R(x) \ge 1/2 \qand F_R(x^-) \le 1/2 & \qedhere
        \end{align}
    \end{proof}
\end{example}


\subsection{Properties of Expectation}
\begin{aside}
    Assume expectations all exist and are finite, i.e. do not diverge.
\end{aside}
\begin{enumerate}
    \item Closure under addition. Let $R_1, \ldots, R_n$ be random variables. Then,
    \begin{align}
        E\qty(\sum_i R_i) = \sum_i E(R_i)
    \end{align}
    \item For $a \in \real$,
    \begin{align}
        E(aR) = a \cdot E(R)
    \end{align}
    \item If $R_1 \le R_2$ for all $\omega \in \Omega$, then $E(R_1) \le E(R_2)$.
    \item If $R \ge 0$ and $E(R) = 0$, then $\probability(R = 0) = 1$, i.e. $R$ is essentially zero.
    \begin{proof}
        It's in the textbook.
    \end{proof}
    \begin{aside}
        If for some random variable $R$, $\text{Var}(R)$, then $R$ is essentially constant.
    \end{aside}
    \item Let $R_1, \ldots, R_n$ be independent random variables. Then,
    \begin{align}
        E(R_1\cdots R_n) = E(R_1) \cdots E(R_n)
    \end{align}
    \item Let $R$ be a random variable with mean $\mu$ and variance $\sigma^2$. Then,
    \begin{align}
        \text{var}(aR + b) = a^2\sigma^2
    \end{align}
    \begin{proof}
        It's in the textbook.
    \end{proof}
    \item Variance is closed under addition, i.e. for independent random variables $R_1, \ldots, R_n$,
    \begin{align}
        \text{var}\qty(\sum_i R_i) = \sum_i \text{var}\qty(R_i)
    \end{align}
    \begin{proof}
        It's in the textbook.
    \end{proof}
\end{enumerate}

