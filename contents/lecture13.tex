\section{October 23, 2024}

\subsection{Review of Moments}
Let $X$ be a random variable. Recall that
\begin{align}
    \alpha_k = E(X^k) && k > 0
\end{align}
is the $k$-th moment. It follows that $E(X) = \text{mean}(X)$, so recall that
\begin{align}
    \beta_k = E((X - m)^k) && k > 0
\end{align}
is the $k$-th central moment (for $m < \infty$).

\subsection{Properties of Expectation Continued}
\begin{enumerate}
    \item[8.] The central moments $\beta_1, \beta_2, \ldots$ of a random variable can be obtained from the moments $\alpha_1, \alpha_2, \ldots$, assuming $\alpha_i < \infty$ for $i < n$ and $\alpha_n$ exists. The result is
    \begin{align}
        \beta_n = E\qty[(R - m)^n] = E \qty[\sum_{k = 0}^n \binom{n}{k} (-m)^{n-k} R^k]
    \end{align}
    assuming $m < \infty$. This simplifies into
    \begin{align}
        \sum_{k=0}^n \binom{n}{k}(-m)^{n-k} \alpha_k
    \end{align}
    because $E(R^k) \equiv \alpha_k$. It follows that for $n = 2$,
    \begin{align}
        \text{var}(R) = E(R^2) - 2mE(R) + m^2 \implies \boxed{\sigma_R^2 = E(R^2) - \qty[E(R)]^2}
    \end{align}
    \item[9.] If $0 < j < k$, then
    \begin{align}
        E(\abs{R}^j) \le 1 + E(\abs{R}^k)
    \end{align}
    \begin{proof}
        For $\omega \in \Omega$,
        \begin{align}
            \abs{R(\omega)}^j \le \begin{cases}
                \abs{R(\omega)}^k & \abs{R(\omega)} \ge 1\\
                1 & \text{else}
            \end{cases}
        \end{align}
        i.e. finiteness is a logical consequence of $0 < j < k$. Thus,
        \begin{align}
            \abs{R(\omega)}^j \le 1 + \abs{R(\omega)}^k
        \end{align}
        for all $\omega \in \Omega$. This extends to the expectation:
        \begin{align}
            E(\abs{R}^j) \le 1 + E(\abs{R}^k)
        \end{align}
    \end{proof}
    It follows that if some higher order expectation is finite, then lower order expectations are also finite.
\end{enumerate}

\begin{example}
    Let $R \sim \exp(\lambda)$ be
    \begin{align}
        f_R(x) = \begin{cases}
            \lambda e^{-\lambda x} & x \ge 0\\
            0 & \text{else}
        \end{cases}
    \end{align}
    Find $\alpha_i$ and $\text{var}(R)$.
\end{example}
\begin{solution}
    By definition,
    \begin{align}
        \alpha_j = E\qty(R^j) = \int_0^\infty x^j \lambda e^{-\lambda x} \dd{x}
    \end{align}
    This integration could clearly be done by parts. But this can also be done through a substitution. Set $y = \lambda x$. Then, $\dd{y} = \lambda \dd{x}$. We get
    \begin{align}
        \dfrac{1}{\lambda^j} \cdot \int_0^\infty y^j \cdot e^{-y} \dd{y}
    \end{align}
    \begin{aside}
        Define the Gamma function
        \begin{align}
            \Gamma\qty(j) \equiv \int_0^\infty y^{j-1} \cdot e^{-y} \dd{y}
        \end{align}
        We can do some induction on $j$:
        \begin{align}
            \Gamma(1) = \int_0^\infty e^{-x} = 1
        \end{align}
        For $\Gamma(r)$,
        \begin{align}
            \int_0^\infty x^{r-1} e^{-x} \dd{x}
        \end{align}
        By parts, this becomes
        \begin{align}
            \cancel{\eval{\frac{x^r}{r} e^{-x}}_0^\infty} + \frac{1}{r} \int_0^\infty x^r e^{-x} \dd{x}
        \end{align}
        This indicates
        \begin{align}
            \Gamma(r) = \dfrac{1}{r} \Gamma(r+1)
        \end{align}
        Inductively, for $n \in \natural$, $\Gamma(n + 1) = n!$.
    \end{aside}
    It follows that Equation 13.12 simplifies into
    \begin{align}
        a_j = \dfrac{j!}{\lambda^j}
    \end{align}
    So it follows that
    \begin{align}
        \text{var}\qty(\exp(\lambda)) = \alpha_2 - \alpha_1^2 = \dfrac{2}{\lambda^2} - \qty(\dfrac{1}{\lambda})^2 = \boxed{\dfrac{1}{\lambda^2}} & \qedhere
    \end{align}
\end{solution}

\subsection{Covariance}
With multiple random variables, we can define covariance for two variables at a time. Let $R_1,R_2$ be two random variables.
\begin{definition}
     The $(i,j)$-th joint moment for the pair of random variables is
    \begin{align}
        \alpha_{i,j} = E\qty[R_1^i \cdot R_2^j] && i,j \ge 1
    \end{align}
\end{definition}
Assume that the moments are finite, i.e. $m_1 = E(R_1) < \infty$ and $m_2 = E(R_2) < \infty$.
\begin{definition}
    The $(i,j)$-th joint central moment for the pair of random variables is
    \begin{align}
        \beta_{i,j} = E\qty[(R_1 - m_1)^i \cdot (R_2 - m_2)^j]
    \end{align}
\end{definition}
\begin{definition}
    The \textbf{covariance} of a pair of random variables $R_1,R_2$ is defined as
    \begin{align}
        \beta_{1,1} = \text{cov}(R_1, R_2) = E((R_1 - E(R_1)) (R_2 - E(R_2)))
    \end{align}
    Note that if $R_1 = R_2$, then
    \begin{align}
        \text{cov}(R_1,R_2) = \text{var}(R_1)
    \end{align}
\end{definition}
\begin{proposition}
    An easier way to compute covariance is given by
    \begin{align}
        \text{cov}(R_1,R_2) &= E(R_1R_2 - R_2m_1 - R_1m_2 + m_1m_2)\\
        &= E(R_1R_2) - m_1 \cdot E(R_2) - m_2 \cdot E(R_1) + m_1m_2\\
        &= E(R_1R_2) - m_1 m_2 - m_1 m_2 + m_1m_2\\
        &= E(R_1R_2) - m_1 m_2
    \end{align}
\end{proposition}

\subsection{Preview of Correlation}
Covariance is not normalized and can take any value $r \in \real$. However, that may be unintuitive, so we want a normalized correlation constant $R \in [-1,1]$
% \begin{aside}
%     If a plot looks like
%     % figure
%     then we say the two random variables are positively correlated and has positive covariance. Similarly,
%     % figure
%     are negatively correlated and has negative covariance. And,
%     % figure
%     are not correlated and has zero covariance.
% \end{aside}
\begin{proposition}
    If $R_1,R_2$ are independent, then $\text{cov}(R_1,R_2) = 0$ and the two variables are uncorrelated.
    \begin{proof}
        $R_1,R_2$ are independent, so
        \begin{align}
            E\qty[(R_1-m_1)(R_2-m_2)] &= E\qty[(R_1-m_1)] \cdot E\qty[(R_2-m_2)]\\
            &= 0 \cdot 0 = 0 && \qedhere
        \end{align}
    \end{proof}
    This is not an iff; functions may be uncorrelated, but not independent.
\end{proposition}
This will be completed in Lecture 14.
