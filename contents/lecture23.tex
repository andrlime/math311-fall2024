\section{November 18, 2024}

\subsection{Limits of Sets}
\begin{definition}
    Take $A_1, A_2, \ldots \in \Omega$ as some sets. Then,
    \begin{align}
        \limsup_n A_n = \bigcap_{n=1}^\infty \bigcup_{k=n}^\infty A_k
    \end{align}
    and
    \begin{align}
        \liminf_n A_n = \bigcup_{n=1}^\infty \bigcap_{k=n}^\infty A_k
    \end{align}
\end{definition}

\begin{proposition}
    \begin{enumerate}
        \item[(i)] The limit
        \begin{align}
            \limsup_n A_n = \{ \omega \in \Omega \mid \omega \text{ is in infinitely many } A_n \}
        \end{align}
        \item[(ii)] the limit
        \begin{align}
            \liminf_n A_n = \{ \omega \in \Omega \mid \omega \text{ is in all, but finitely many } A_n \}
        \end{align}
        Equivalently, $\liminf \cdots$ means that $\omega$ is \textit{eventually} in $A_n$, i.e. for some $k$, $\omega \in A_n$ for all $k \ge n$
    \end{enumerate}
\end{proposition}
\begin{proof}
    \begin{enumerate}
        \item[(i)] By definition,
        \begin{align}
            \omega \in \limsup_n A_n
            &\iff \forall n, \omega \in \bigcup_{k=n}^\infty A_k\\
            &\iff \forall n, \exists k \ge n, \omega \in A_k\\
            &\iff \omega \text{ is in infinitely many } A_n
        \end{align}
        because if this was not the case, then there would be some largest $n$, but that is a contradiction with there existing $k \ge n$ \textbf{for all} $n$.

        \item[(ii)] By definition,
        \begin{align}
            \omega \in \liminf_n A_n
            &\iff \omega \in \bigcap_{k=n}^\infty \text{ for some } n\\
            &\iff \text{for some } n, \forall k \ge n, \omega \in A_k
        \end{align}
        Beyond a certain $k$, $\omega$ must be in all $A_n$.
    \end{enumerate}
\end{proof}

As an example...
\begin{example}
    Take $A_n$ defined as
    \begin{align}
        A_n = \begin{cases}
            [-1,1] \subseteq \real & n \text{ even}\\
            [-2,2] \subseteq \real & n \text{ odd}
        \end{cases}
    \end{align}
    Then, $\limsup_n A_n = [-2,2]$ and $\liminf_n A_n = [-1,1]$.
\end{example}

\subsection{Almost Sure Convergence}
\begin{definition}
    Let $R_1, R_2, \ldots$ be random variables on a given probability space $\psp$. Define the set
    \begin{align}
        \{ R_n \to R \} \equiv \{ \omega \in \Omega \mid R_n(\omega) \ce{->[$n \to \infty$]} R(\omega) \}
    \end{align}
    We say that $R_n$ ``converges almost surely'' to $R$, or $R_n \ce{->[\text{a.s.}]} R$, if
    \begin{align}
        \probability(\{ R_n \to R \}) = 1
    \end{align}
\end{definition}

\begin{example}
    Take $\Omega = [0,1]$, $\curlyf$ as Borel sets of $[0,1]$, and $\probability([a,b]) = b-a$ where $0 \le a \le b \le 1$. Define $R_n(\omega)$ as
    \begin{align}
        R_n(\omega) = \begin{cases}
            1 & \omega = 0\\
            1/n & \omega \in (0,1]
        \end{cases}
    \end{align}
    Show that $R_n \ce{->[\text{a.s.}]} 0$
\end{example}
\begin{solution}
    The set $\{ R_n \to R \}$ is the set $\{ \omega \in [0,1] \mid R_n(\omega) \ce{->[$n \to \infty$]} 0 \}$. This set is just $(0,1]$. The probability of this set is 1 by definition, so this set obviously ``converges almost surely''.
\end{solution}

This definition is slightly difficult to work with.
\begin{proposition}
    \begin{align}
        \{ R_n \to R \} = \bigcap_{m=1}^\infty \liminf_n A_{nm}
    \end{align}
    where $A_{nm}$ is defined as
    \begin{align}
        A_{nm} \equiv \{ \omega \in \Omega \mid \abs{R_n(\omega) - R(\omega)} \le 1/m \}
    \end{align}
\end{proposition}
\begin{proof}
    This proof can be done with equivalences again
    \begin{align}
        \omega \in \{ R_n \to R \}
        &\iff R_n(\omega) \ce{->[$n \to \infty$]} R_\omega\\
        &\iff \forall \varepsilon > 0, \abs{R_n(\omega) - R(\omega)} < \varepsilon \text{ for $n$ suf. large}\\
        &\iff \forall m \in \natural, \abs{R_n(\omega) - R(\omega)} < 1/m \text{ for $n$ suf. large}\\
        &\iff \forall m, \omega \in A_{nm} \text{ (eventually)}\\
        &\iff \forall m, \omega \in \liminf_n A_{nm}
    \end{align}
\end{proof}

\begin{proposition}
    \begin{align*}
        R_n \ce{->[\text{a.s.}]} R \iff \forall \varepsilon > 0, \probability\qty(\{ \abs{R_k - R} \ge \varepsilon \text{ for some $k \ge n$}\}) \ce{->[$n \to \infty$]} 0
    \end{align*}
\end{proposition}
We can first prove a lemma!
\begin{lemma}
    If $R_n \ce{->[\text{a.s.}]} R$, then $R_n \ce{->[$\probability$]} R$.
\end{lemma}
\begin{proof}
    Let $\varepsilon > 0$. Then, for some $k \ge n$,
    \begin{align}
        &\probability(\abs{R_n - R} \ge \varepsilon)\\
        \le &\probability(\{ \abs{R_k - R} \ge \varepsilon \text{ for some $k \ge n$} \})
    \end{align}
    as $n\to\infty$, this reduces to
    \begin{align}
        \probability(\abs{R_n - R} \ge \varepsilon) \ce{->[$n \to \infty$]} 0
    \end{align}
    which is equivalent to
    \begin{align}
        R_n \ce{->[$\probability$]} R
    \end{align}
\end{proof}

\begin{lemma}
    The converse is not true, i.e. convergence in probability does not imply convergence in ``almost sure''.
\end{lemma}
\begin{proof}
    Take $R_n \sim \text{Bernoulli}(1/n)$ where $\text{Bernoulli}(x)$ is the result of a single Bernoulli trial with probability of success $x$. By definition, for $1 > \varepsilon > 0$,
    \begin{align}
        \probability(\abs{R_n - 0} \ge \varepsilon)
        &= \probability(R_n = 1)\\
        &= 1/n \ce{->[$n \to \infty$]} 0
    \end{align}
    However, $R_n$ does not ``almost surely'' converge.
    \begin{align}
        \probability(\{ \abs{R_k - R} \ge \varepsilon \text{ for some $k \ge n$} \})
        &= \probability(\{R_k = 1 \text{ for some $k \ge n$}\})\\
        &\ge \probability(\{ R_k = 1 \text{ for some $k \in [1,N]$, $N > n$} \})
    \end{align}
    This is a smaller set, so proving the statement for this finite case proves the converse not being true.
    \begin{align}
        &\ge \cdots\\
        &= 1 - \probability(R_k = 0 \text{ for all } k = n, n+1, \ldots, N)\\
        &= 1 - \qty(\frac{n-1}{n})\qty(\frac{n}{n+1})\cdots\qty(\frac{N-1}{N})\\
        &= 1 - \frac{n-1}{N} \ne 0
    \end{align}
    As $N \to \infty$, then
    \begin{align}
        \probability(\abs{R_k - 0} \ge \varepsilon \text{ for some $k \ge n$}) \ge 1 = 1
    \end{align}
    Thus, the probability does not go to zero, so the sets do not ``almost surely'' converge.
\end{proof}

\subsubsection{Proof of Proposition}
Here comes another lemma...
\begin{lemma}
    \begin{align}
        R_n \ce{->[\text{a.s.}]} R \iff \forall m \in \natural, \probability\qty(\liminf_n A_{nm}) = 1
    \end{align}
\end{lemma}
\begin{proof}
    \begin{enumerate}
        \item[$\implies$] This is direct
            \begin{align}
                R_n \ce{->[\text{a.s.}]} R
                &\implies \probability(\{ R_n \to R \}) = 1\\
                &\implies \probability\qty(\bigcap_{m=1}^\infty \liminf_n A_{nm}) = 1\\
                &\implies \forall m \in \natural, \probability\qty(\liminf_n A_{nm}) = 1
            \end{align}
        \item[$\impliedby$] This uses the last problem from the first midterm, which is in one of the Appendixes.
            \begin{align}
                \forall n, \probability(B_n) = 1 \implies \probability\qty(\bigcap_{n=1}^\infty B_n) = 1
            \end{align}
            Using this,
            \begin{align}
                \probability\qty(\bigcap_{m=1}^\infty \liminf A_{nm}) = 1
                &\implies \probability(\{ R_n \to R \}) = 1\\
                &\implies R_n \ce{->[\text{a.s.}]} R
            \end{align}
    \end{enumerate}
\end{proof}

Now we can prove the proposition.
\begin{align}
    R_n \ce{->[\text{a.s.}]} R &\iff \forall m, \probability\qty(\liminf_n A_{nm}) = 1\\
    &\iff \forall m, \lim_{n\to\infty} \probability\qty(\bigcap_{k=n}^\infty A_k) = 1 \qsp \text{expanding seq.}\\
    &\iff \probability\qty(\forall k \ge n \qsp \abs{R_k - R} < 1/m) \ce{->[$n \to \infty$]} 1\\
    &\iff \probability\qty(\exists k \ge n \qsp \abs{R_k - R} \ge 1/m) \ce{->[$n \to \infty$]} 0\\
    &\iff \probability\qty(\abs{R_k - R} \ge \varepsilon \text{ for some $k \ge n$}) \ce{->[$n \to \infty$]} 0
\end{align}

