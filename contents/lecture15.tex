\section{October 28, 2024}

\subsection{Chebyshev's Inequality}
Stating this theorem is harder than proving it.
\begin{theorem}
    \textbf{Chebyshev's Inequality}
    \begin{enumerate}
        \item[a.] $R \ge 0$ r.v.; $b > 0$ random number
        \begin{align}
            \probability(R \ge b) \le E(R)/b && \text{assume expectation exists}
        \end{align}
        \item[b.] $R$ r.v.; $c$ constant, $l, \varepsilon > 0$ constants
        \begin{align}
            \probability(\abs{R - c} \ge \varepsilon) \le \frac{E(\abs{R-c})^l}{\varepsilon^l}
        \end{align}
        \item[c.] $R$ r.v. finite mean $m$, finite variance $\sigma^2 > 0$, $k > 0$
        \begin{align}
            \probability(\abs{R-m} \ge k\sigma) \le 1/k^2
        \end{align}
    \end{enumerate}
\end{theorem}

\subsubsection{Proof of (a)}
First, assuming $R$ is absolutely continuous with density $f(x)$. Then,
\begin{align}
    E(R) = \int_{-\infty}^\infty x f(x) \dd{x}
\end{align}
Since $R \ge 0$, this is equivalent to
\begin{align}
    E(R) = \int_{0}^\infty x f(x) \dd{x}
\end{align}
We can shove an inequality at this
\begin{align}
    E(R) = \int_{0}^\infty x f(x) \dd{x}\ge \int_b^\infty x f(x) \dd{x}
\end{align}
Note that we are throwing away $\int_0^b$, so this is not a very sharp estimate. Note that within this integral, $x \ge b$, so this previous line is greater than
\begin{align}
    E(R) \ge \int_b^\infty x f(x) \dd{x} \ge b\int_b^\infty f(x) \dd{x} = b \probability(R \ge b)
\end{align}
Rearranging,
\begin{align}
    \probability(R \ge b) \le E(R)/b
\end{align}
For the general case, we claim
\begin{align}
    R \ge b I_{A_b}
\end{align}
where $A_b \equiv \{ R \ge b \} \subset \Omega$. Recall that $I$ is the indicator function, i.e.
\begin{align}
    I_{A_b}(\omega) = \begin{cases}
        1 & \omega \in A_b\\
        0 & \omega \notin A_b
    \end{cases}
\end{align}
To prove this, let $\omega \in \Omega$. We want $R(\omega) \ge b I_{A_b}(\omega)$. We can split into cases
\begin{enumerate}
    \item Case 1: $\omega \notin A_b$. Then, we want to show that $R(\omega) \ge 0$, which is true by assumption that $R \ge 0$.
    \item Case 2: $\omega \in A_b$. Then $b I_{A_b}(\omega) = b$. We need to show that $R(\omega) \ge b$. This is true by how $A_b$ is defined, i.e. as $\{ R \ge b \}$.
\end{enumerate}
This proves the claim, i.e. for all $\omega$, this statement is true:
\begin{align}
    R \ge b I_{A_b}
\end{align}
This implies that $E(R) \ge E(b I_{A_b})$. By linearity,
\begin{align}
    E(R) \ge b E\qty(I_{A_b}) = b\probability(A_b) = b \probability(R \ge b)
\end{align}
which can be rearranged to complete the proof for the general, non-absolutely-continuous case.
\qed

\subsubsection{Proof of (b)}
This follows from $A$. The probability
\begin{align}
    \probability(\abs{R-c} \ge \varepsilon) = \probability(\abs{R-c}^l \ge \varepsilon^l)
\end{align}
Then we apply (a) where $\abs{R - c}^l$ is our new, non-negative random variable and $\varepsilon^l$ is our number. Thus,
\begin{align}
    \probability(\abs{R-c}^l \ge \varepsilon^l) \le \frac{E\qty(\abs{R-c}^l)}{\varepsilon^l}
\end{align}
\qed

\subsubsection{Proof of (c)}
The probability $\probability(\abs{R-m} \ge k\sigma)$, by (b),
\begin{align}
    \probability(\abs{R-m} \ge k\sigma) \le \frac{E(\abs{R-m}^2)}{(k\sigma)^2} = \frac{\sigma^2}{k^2\sigma^2} = 1/k^2
\end{align}
\qed

\begin{aside}
    This is not a very sharp inequality. Take $R \sim \exp(1)$ (mean and variance are both $1$). By (c), (assume $k$ large)
    \begin{align}
        \probability(\abs{R-m} \ge k\sigma) \le 1/k^2
    \end{align}
    For our specific distribution, $m= \sigma = 1$, so we get 
    \begin{align}
        \probability(\abs{R - m} \ge k\sigma) = \probability(\abs{R - 1} \ge k) = \probability(R \ge k + 1) = e^{-(k+1)}
    \end{align}
    For $k$ large,
    \begin{align}
        e^{-(k+1)} \ll 1/k^2
    \end{align}
    so the inequality is not sharp, and quite wasteful.
\end{aside}


\subsection{Weak Law of Large Numbers}
\begin{theorem}
    Let $R_1, R_2, \ldots$ be a whole bunch of independent random variables on a given probability space. Assume the means and variances are finite, and shared (equal) across all $R_{1 \cdots N}$. And assume $\sigma_i^2 \le M$, i.e. the variances are bounded by some constant $M \in \real$ for all $i$. Define
    \begin{align}
        S \equiv \sum_{i} R_i
    \end{align}
    and let $n$ be the number of random variables being summed over. Then, for all $\varepsilon > 0$,
    \begin{align}
        \probability\qty[\frac{\abs{S - E(S)}}{n} \ge \varepsilon] \ce{->[$n \to \infty$]} 0
    \end{align}
    or
    \begin{align}
        \probability\qty[ \abs{\frac{R_1 + \cdots + R_n}{n} - m} \ge \varepsilon ] \ce{->[$n \to \infty$]} 0
    \end{align}
    i.e. the mean of the random variables tends towards the actual mean as $n \to \infty$
\end{theorem}
\begin{example}
    Flip $N$ coins. As $N \to \infty$, the expected number of heads is $N/2$. As $N \to \infty$, the probability of deviation from the expected value tends towards zero.
\end{example}
\begin{proof}[Solution]
    Obvious.
\end{proof}
Why is this law of large numbers \textit{weak}? Because it is stating that the probability of being greater than $\varepsilon$ goes to zero, as opposed to an actual probability going to zero.

\subsubsection{Proof}
Take
\begin{align}
    \probability\qty(\frac{\abs{S - E(S)}}{n} \ge \varepsilon)
\end{align}
By Part (b) of Chebyshev, we can write
\begin{align}
    \probability\qty(\frac{\abs{S - E(S)}}{n} \ge \varepsilon) &\le \frac{E\qty(\abs{\frac{S-E(S)}{n}}^2)}{\varepsilon^2}\\
    &\le \frac{1}{n^2\varepsilon^2} E\qty[(S - E(S))^2]\\
    &\le \frac{\text{var}(S)}{n^2\varepsilon^2}
\end{align}
By linearity on the independent variables,
\begin{align} 
    \probability\qty(\frac{\abs{S - E(S)}}{n} \ge \varepsilon)
    &\le \frac{\sum_i^n \text{var}(R_i)}{n^2\varepsilon^2}\\
    &\le \frac{Mn}{n^2\varepsilon^2} = \frac{M}{n\varepsilon^2}\\
    &\le 0\\
    \implies \probability\qty(\frac{\abs{S - E(S)}}{n} \ge \varepsilon) &= 0
\end{align}


\subsection{Conditional Probability}
Let $R_1,R_2$ be discrete r.v. with probability functions $P_{R_1}(x)$ and $P_{R_2}(x)$ which are `probability mass functions'. What is
\begin{align}
    \probability(R_1 = x \mid R_2 = y)
\end{align}
The probability $p(x \mid y)$ equals
\begin{align}
    \frac{\probability(R_1 = x, R_2 = y)}{\probability(R_2 = y)} = p_{12}(x,y) / p_2(y)
\end{align}
which is the quotient of the joint probability function and the individual probability function.
\begin{aside}
    For a given, fixed $y$, if $p_2(y) > 0$, then
    \begin{align}
        p(x \mid y)
    \end{align}
    is a probability function for $x$, i.e.
    \begin{align}
        \sum_x p(x \mid y) = 1
    \end{align}
    which is true by extending the law of total probability.
\end{aside}

