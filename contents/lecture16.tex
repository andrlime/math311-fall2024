\section{October 30, 2024}

\subsection{Conditional Probabilities}
Quick recap: for two events $A$ and $B$,
\begin{align}
    \probability(B \mid A) = \frac{\probability(A \cap B)}{\probability(A)}
\end{align}
On absolutely continuous random variables, $(R_1,R_2)$ with some joint density $f(x, y)$, write $f_1(x)$ marginal density of $R_1$, i.e.
\begin{align}
    f_1(x) = \int_\real f(x,y) \dd{y}
\end{align}
We can define the conditional density of $R_2$ \textit{given} $R_1 = x$ to be
\begin{align}
    h(y \mid x) = \frac{f(x,y)}{f_1(x)}
\end{align}

\begin{aside}
    A remark about notation
    \begin{enumerate}
        \item $\probability(R_1 = x) = 0$ because of absolute continuity
        \item This can also be denoted as $f_{R_2 \mid R_1}(y \mid x)$ as opposed to $h(y \mid x)$
    \end{enumerate}
\end{aside}

Let $R$ be a random variable with continuous density $f(x)$. Then, let $\varepsilon > 0$:
\begin{align}
    \probability(x \le R \le x + \varepsilon) \approx f(x) \Delta x
\end{align}
Applying this to the conditional probability $h(x,y)$ yields
\begin{align}
    \frac{f(x,y) \varepsilon_x \varepsilon_y}{f_1(x) \varepsilon_x} \approx \dfrac{\probability(x \le R_1 \le x + \varepsilon_x \cap y \le R_2 \le y + \varepsilon_y)}{\probability(x \le R_1 \le x + \varepsilon_x)}
\end{align}

\begin{aside}
    Another remark. Fix $x$. Then, $h(y \mid x)$ is a density in $y$, so
    \begin{align}
        \int_\real h(y \mid x) \dd{y} = \frac{\int_\real f(x,y)\dd{y}}{f_1(x)} = 1
    \end{align}
\end{aside}

\begin{lemma}
    Take $B \in \mathcal{B}$. Then,
    \begin{align}
        \probability(R_2 \in B \mid R_1 = x) = \int_B h(y \mid x) \dd{y}
    \end{align}
\end{lemma}
\begin{proof}
    Quite obvious.
\end{proof}

\begin{example}
    $(R_1,R_2)$ have some joint density
    \begin{align}
        f(x,y) = \begin{cases}
            e^{-y} & 0 \le x \le y\\
            0 & \text{else}
        \end{cases}
    \end{align}
    Find the conditional density of $R_2$ given $R_1 = x$ and $\probability(R_2 \le y \mid R_1 = x)$, which is the \textit{conditional distribution function} of $R_2$ given $R_1$.
\end{example}
\begin{solution}
    The denominator equals
    \begin{align}
        f_1(x) &= \int_\real f(x,y) \dd{y}\\
        &= \int_x^\infty e^{-y} \dd{y}\\
        &= \cdots = e^{-x}
    \end{align}
    Thus, $h(y \mid x)$ equals
    \begin{align}
        h(y \mid x) = f(x,y) / f_1(x) = e^{x - y}
    \end{align}
    on $y \ge x \ge 0$ and $0$ everywhere else. Now, assume $y \ge x \ge 0$.
    \begin{align}
        \probability(R_2 \le y \mid R_1 = x) = \int_{-\infty}^y h(u \mid x) \dd{u} = \int_x^y e^{x-u} \dd{u} = \cdots = -e^{x-y} + 1
    \end{align}
\end{solution}

\begin{aside}
    If $R_1,R_2$ are independent, then $h(y \mid x) = f_2(y)$.
\end{aside}

\subsection{Higher Dimensions}
Let $R_1, \ldots, R_N$ be random variables with joint density $f(x_1, \ldots x_N)$. Let the joint density $(R_1, \ldots, R_k)$ be $$f_{1\cdots k}(x_1, \ldots, x_k)$$ for $k \in \{ 1, \ldots, n - 1 \}$. Then, the conditional density of $(R_{k+1}, \ldots, R_N)$ given $(R_1, \ldots R_k)$ is
\begin{align}
    h(x_{k+1}, \ldots, x_n \mid x_1, \ldots x_k) = \frac{f(x_1, \ldots, x_n)}{f_{1\cdots k}(x_1, \ldots, x_k)}
\end{align}
\begin{example}
    Let $R_0 \sim \exp{1}$. If $R_0$ has value $\lambda$, random variables $R_1, \ldots, R_N$ are independent, and $R_i \sim \exp{\lambda}$ for $i = 1, \ldots, N$, then find the conditional density of $R_0$ given $(R_1, \ldots, R_N) = (x_1, \ldots, x_N)$ (where $x_1, \ldots, x_N > 0$).
\end{example}
\begin{solution}
    We want
    \begin{align}
        h(\lambda \mid x_1, \ldots, x_N) = \frac{f(\lambda, x_1, \ldots, x_N)}{f_{1\cdots n}(x_1, \ldots, x_N)}
    \end{align}
    We know the density $h(x_1, \ldots, x_N \mid \lambda)$. We want the density of $R_1, \ldots, R_N$ given $R_0 = \lambda$. That just equals
    \begin{align}
        \lambda^N e^{-\lambda(x_1 + \cdots + x_N)}
    \end{align}
    From the formula, $h(x_1, \ldots, x_N \mid \lambda) \cdot f_0(\lambda) = f(\lambda, x_1, \ldots, x_N)$. We eventually get that
    \begin{align}
        f(\lambda, x_1, \ldots, x_N) &= \lambda^N e^{-\lambda(x_1 + \cdots + x_N)}e^{-\lambda}\\
        &= \lambda^N e^{-\lambda(1 + x_1 + \cdots + x_N)}
    \end{align}
    The denominator of the conditional density is the integral of the joint density on $\lambda \in \real$. That equals
    \begin{align}
        \int_0^\infty \lambda^N e^{-\lambda(1 + x_1 + \cdots + x_N)} \dd{\lambda}
    \end{align}
    and after integrating that (possibly asking $\Gamma$, Mathematica, or some other -a's for help), we can quite trivially find the quotient that equals the conditional density. The final expression is
    \begin{align}
        \frac{\lambda^N e^{-\lambda(1 + x_1 + \cdots + x_N)}(1 + x_1 + \cdots + x_N)}{N!}
    \end{align}
\end{solution}

\begin{aside}
    $$\Gamma(n+1) = \int_0^\infty x^n e^{-x} \dd{x} = n!$$
\end{aside}

\subsection{Theorem of Total Probability}
Take $\{ A_i \}_I$ are mutually exclusive and exhaustive events in some probability space. Recall that
\begin{align}
    \probability(B) = \sum_{i \in I} \probability(B \mid A_i) \cdot \probability(A_i)
\end{align}
What about for a continuous case?
\begin{definition}
    Take $R_1,R_2$ as random variables with joint density $f(x,y)$. Let $R_1$ have density $f_1(x)$. Write $h(y \mid x) = f(x,y)/f_1(x)$. Then, given some event $B$,
    \begin{align}
        \probability(R_2 \in B) = \int_\real \probability(R_2 \in B \mid R_1 = x) f_1(x) \dd{x}
    \end{align}
    This is exactly the same thing as the discrete case.
\end{definition}
This is consistent with what we already have.
\begin{align}
    \probability(R_2 \in B \mid R_1 = x) = \probability(B_x) = \int_{y \in B} h(y \mid x) \dd{y}
\end{align}
and
\begin{align}
    \probability(R_2 \in B) = \int_\real \int_{y \in B} f(x,y) \dd{y} \dd{x}
\end{align}
\begin{aside}
    The theorem of total probability for a continuous random variable (see above) holds even when $R_2$ is discrete. As an example, suppose there is some number $x \in [0, 1]$ uniformly and randomly. Then, flip a coin with probability of heads is $x^2$. Find the probability of heads.
\end{aside}
\begin{solution}
    Let $R_1$ be $x \in [0,1]$. Let $R_2 = I_\text{heads}$ ($1$ is heads, $0$ if tails). Then, $\probability(R_2 = 1)$ equals
    \begin{align}
        \probability(R_2 = 1) = \int_0^1 \probability(R_2 = 1 \mid R_1 = x) f_1(x) \dd{x}
    \end{align}
    In this case, this reduces to
    \begin{align}
        \probability(R_2 = 1) = \int_0^1 x^2 \dd{x} = 1/3
    \end{align}
\end{solution}

