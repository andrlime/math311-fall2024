\section{October 4, 2024}
\subsection{Random Variables}
Take $(\Omega, \curlyf, \probability)$ as a probability space. Recall

\begin{definition}
    A random variable is some function
    \begin{align}
        R : \Omega \to \real
    \end{align}
    such that $R^{-1}([a,b]) \in \curlyf$. Note that
    \begin{align}
        R^{-1}([a,b])
    \end{align}
    describes all points $\chi \in \Omega$ such that $R(\chi)$ maps into $[a,b] \in \real$.
\end{definition}
\begin{definition}
    A random variables is some function
    \begin{align}
        R: \Omega \to \real
    \end{align}
    such that $R^{-1}(B) \in \curlyf$ for all Borel sets $B \in \mathcal{B}$ on $\real$.
\end{definition}

\begin{proposition}
    The above two definitions are equivalent.
\end{proposition}
\begin{proof}
    Suppose $R$ is a random variable as defined in Definition 6.1. Define
    \begin{align}
        \mathcal{G} = \{ G\subset \real \mid R^{-1}(G) \in \curlyf \}
    \end{align}
    By Definition 6.1, all closed intervals are in $\mathcal{G}$. We now claim that $\mathcal{G}$ is a $\sigma$-algebra on $\real$. If this is true, then $\mathcal{G}$ being a $\sigma$-algebra containing all closed intervals means it contains all Borel sets $B \in \mathcal{B}$, i.e. $\mathcal{B} \subseteq \mathcal{G}$. This means Definition 6.2 $\subseteq$ Definition 6.1.
    \begin{lemma}
        \textbf{The Claim} $\mathcal{G}$ is a $\sigma$-algebra on $\real$
    \end{lemma}
    \begin{proof}[Proof of Lemma 6.4]
        \begin{align}
            R^{-1}(\real) &= \Omega \in \curlyf \implies \real \in \mathcal{G}\\
            G_i \in \mathcal{G} &\implies R^{-1}(G_i) \in \curlyf \implies \bigcup_i R^{-1}(G_i) \in \curlyf\\
            &\implies R^{-1}\qty(\bigcup_i G_i) \in \curlyf \implies \qty(\bigcup_i G_i) \in \mathcal{G}\\
            G \in \mathcal{G} &\implies R^{-1}(G) \in \curlyf \implies \qty(R^{-1}(G))^C \in \curlyf \implies G^C \in G
        \end{align}
    \end{proof}
    On the converse, if we have a random variable as defined in Definition 6.2, then that implies that
    \begin{align}
        R^{-1}([a,b]) \in \curlyf
    \end{align}
    for all $a \le b$ because all closed intervals are Borel sets. Then that trivially proves the definition, i.e. Definition 6.1 $\subseteq$ Definition 6.2. Thus,
    \begin{align}
        \text{Definition 6.1} \equiv \text{Definition 6.2} & \qedhere
    \end{align}
\end{proof}

\subsection{Discrete Random Variables}
Take some probability space. Take some random variable
\begin{align}
    R: \Omega \to \real
\end{align}
\begin{definition}
    A random variable $R: \Omega \to \real$ is discrete if $\Im(R)$ is a finite or countably infinite set of points, i.e. $R$ hits a countable number of points.
\end{definition}
\begin{example}
    Flip a coin six times. $\Omega$ is the set of all possible outcomes. Take $R: \Omega \to \natural$ as defined as the number of heads. The image of $R$ is
    \begin{align}
        \kappa := \{ 0, \ldots, 6 \}
    \end{align}
    which is discrete.
\end{example}
\noindent Define a probability function $\probability_R(x)$ as
\begin{align}
    \probability(R = x)
\end{align}
Then, $\probability(R \in B)$ where $B \in \mathcal{B}$ is the sum
\begin{align}
    \sum_{x \in B} \probability_R(x)
\end{align}
Define a distribution function $F_R(x)$ as
\begin{align}
    F_R(x) = \probability(\{ R \le x \}) = \sum_{t \le x} \probability_R(t)
\end{align}
In Example 6.6,
\begin{align}
    \probability_R(k \in \kappa) = \binom{6}{k}\left( \frac{1}{2} \right)^6
\end{align}
For $k \notin \kappa$, $\probability(k) = 0$. The distribution function is then
\begin{align}
    F_R(x) = \sum_{t=0}^x \probability_R(t) \dd{t}
\end{align}

\subsection{Absolutely Continuous Random Variables}
Take a probability space $(\real, \mathcal{B}, \probability)$. Define
\begin{align}
    \probability(B \in \mathcal{B}) = \int_B f(x) \dd{x}
\end{align}
such that $f$ is a given probability density function which is \textbf{non-negative}, \textbf{integrable}, and
\begin{align}
    \int_\real f(x) \dd{x} = 1
\end{align}
Define a random variable $R(\omega) \equiv \omega$. As in the same way we defined a distribution function,
\begin{align}
    F_R(x) \equiv \probability(R \le x)
\end{align}
Based on the integral above,
\begin{align}
    F_R(x) = \int_{-\infty}^x f(\omega) \dd{\omega}
\end{align}
\begin{definition}
    Given a probability space $(\Omega, \curlyf, \probability)$ and a random variable $R: \Omega \to \real$. $R$ is absolutely continuous if there exists an integrable density function $f_R \ge 0$ such that $F_R(x) = \int_{-\infty}^x f_R(t) \dd{t}$
\end{definition}
\begin{proposition}
    Let $R$ be absolutely continuous with a density function $f_R$. Then,
    \begin{align}
        \probability(R \in (a,b]) = \int_{a}^b f_R(t) \dd{t}
    \end{align}
    \begin{proof}
        Trivial
    \end{proof}
    This is equivalent to
    \begin{align}
        \probability(R \in B) = \int_B f(x) \dd{x}
    \end{align}
    for Borel set $B \in \mathcal{B}$.
\end{proposition}
\begin{lemma}
    For an \textbf{absolutely continuous} random variable $R$ with density function $f_R$, $\probability\qty(R = \chi)$ for some fixed value $\chi$ equals zero. This does not mean $\chi$ \textit{never} occurs, but the probability is zero.
    \begin{proof}
        Trivial
    \end{proof}
\end{lemma}
\begin{example}
    \textbf{Uniform Distribution}
    Take probability space $(\real, \mathcal{B}, \probability)$ where $\probability$ is some density function.
    \begin{align}
        \probability(B) = \int_B f(x) \dd{x}
    \end{align}
    such that the density function is defined as
    \begin{align}
        f_R(x) = \begin{cases}
            \frac{1}{b-a} & a \le x \le b\\
            0 & \text{else}
        \end{cases}
    \end{align}
    for some constants $a,b$. This has uniform probability for a subset of $\real$. Then, the cumulative distribution function equals
    \begin{align}
        F_R(x) = \probability(R \le x) = \begin{cases}
            0 & x < a\\
            \int_a^{x} \frac{1}{b-a} \dd{t} = \frac{x-a}{b-a} & x \in [a,b]\\
            1 & x > b
        \end{cases}
    \end{align}
\end{example}

% \subsection{Functions of a Random Variable}


